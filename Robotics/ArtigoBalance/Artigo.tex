%*******************************************************************************************%
%            	DYNAMIC MODELLING AND CONTROL OF BALANCED PARALLEL MECHANISMS           	%
% 																					   		%
% April 26, 2015														 						%
% Authors: Andre G. Coutinho, Tarcisio A. H. Coelho					%
% bash adaptive.sh							 												%
% 																							%
%*******************************************************************************************%



%%%%%%%%%%%%%%%%%%%%%%
\documentclass[a4paper,11pt,brazil,fleqn]{article}
\synctex=1
%%%%%%%%%%%%%%%%%%%%%%

% \usepackage{natbib}
\usepackage[english]{babel}
% \usepackage{amsmath,amssymb,amsthm,amsfonts,textcomp}
% \usepackage{eucal,eufrak,mathrsfs,bbm,stmaryrd}
\usepackage{color}
\usepackage{amsthm}
\usepackage{array,hhline,supertabular}
\usepackage[colorlinks,citecolor=black,urlcolor=black,linkcolor=black]{hyperref}
\usepackage[pdftex]{graphicx}
\usepackage{multicol}
\usepackage[symbol]{footmisc}
\usepackage{enumitem}
\usepackage{float}
\usepackage{titlesec}
\usepackage{nomencl}
\usepackage{EXTRAS/special-char}
\usepackage{EXTRAS/special-conf}
\usepackage{subfigure}

\graphicspath{{FIGURES/}{../FIGURES/}}
\makenomenclature


%%%%%%%%%%%%%%%%%%%%%%
\begin{document}
%%%%%%%%%%%%%%%%%%%%%%

\noindent
{\bf \huge A new approach for designing dynamic balanced serial mechanisms}\\

\noindent
{\Large 		Andr\'e Garnier Coutinho$\,{}^\text{a}$,
			Tarcisio Antonio Hess Coelho$\,{}^\text{b}$
}\\

\noindent
{${}^\text{a}$ \it Department of Mechatronics and Mechanical Systems Engineering, Escola Politecnica, 
University of Sao Paulo, Brazil. E-mail: andre.garnier.coutinho@usp.br}

\noindent
{${}^\text{b}$ \it Department of Mechatronics and Mechanical Systems Engineering, Escola Politecnica, 
University of Sao Paulo, Brazil. E-mail: tarchess@usp.br}

\vspace{24pt}

% \begin{multicols}{1}

\begin{abstract}
Balancing is an important issue related to the design of mechanical systems in general,
and also parallel mechanisms, in particular. In fact, the performance of parallel mechanisms
associated to specific applications depends on the choice of the balancing method, namely, 
either static or dynamic, either passive or active, whether it is valid for a given trajectory
or even for any motion.
The main contribution of this work is to highlight the importance of the dynamic modelling 
process in order to achieve the compensation conditions associated to the chosen 
balancing technique. Due to the fact that parallel mechanisms have highly complex structures,
the use of dynamic formalisms that employ redundant generalized coordinates,
in association with the successive coupling of additional balancing elements 
to the original system model, can bring remarkable benefits.
Additionally, this book chapter also discusses the impact of
the dynamic model, developed in accordance with the methodology shown here, for the 
control strategy of parallel mechanisms.
Finally, the simulation results demonstrates
how effective is the presented methodology for the planar 5-bar with revolute joints (5R). 

\vspace{10pt}

\noindent
KEYWORDS: {Dynamic balancing, serial mechanisms}
\end{abstract}



\printnomenclature[5em]

% basic mathematical alphabets

\nomenclature[A001]{$a,b, \ldots$}{Scalars, components of column-matrices, components of matrices or indexes}
\nomenclature[A002]{$A,B, \ldots$}{Scalars, components of column-matrices or components of matrices}

\nomenclature[A012]{$\ma, \mb, \ldots$}{Column-matrices}
\nomenclature[A013]{$\mA, \mB, \ldots$}{Matrices}

\nomenclature[A021]{$\va, \vb, \ldots$}{Vectors}
%\nomenclature[A022]{$\vA, \vB, \ldots$}{Tensors}

%\nomenclature[A031]{$\tta, \ttb, \ldots$}{Points}
\nomenclature[A032]{$\ttA, \ttB, \ldots$}{Coordinate systems}

%\nomenclature[A041]{$\llA, \llB, \ldots$}{Rigid bodies or reference frames}

\nomenclature[A051]{$\ssA, \ssB, \ldots$}{Sets or multibody mechanical systems\footnote{
	A multibody mechanical system will be conceived as a set whose elements are 
	material bodies, joints, actuators, energy storage, dissipation and transformation elements
	and a mathematical model (which includes physical parameters, model variables and 
	constitutive, constraint and dynamic equations).
	}}


% special char

%\nomenclature[CAA01]{$a_{n,l}$}{Arbitrary physical parameter}
% \nomenclature[CAA02]{$\nb a_{n,l}$}{Fixed physical parameter}
%\nomenclature[CAA11]{$\mA_{n}$}{Jacobian matrix of kinematic invariants ($\mc_n$) with respect to 
%	quasi-accelerations ($\dot\mp_n$)}
%\nomenclature[CAA21]{$\va_{\ttp \rl \llE}$}{Acceleration of point $\ttp$ measured relatively to reference frame $\llE$}
\nomenclature[CAB01]{$\ttB_i$}{Coordinate system fixed in the i\ts{th} rigid body of the mechanical system}

\nomenclature[CAC01]{$\mC$}{Kinematic constraints matrix}
%\nomenclature[CAC11]{$\mc_{n}$}{Kinematic invariants (constraints) column-matrix}
%\nomenclature[CAC51]{$\ssC^s$}{Differentiability class}
\nomenclature[CAC91]{$\ccos(.)$}{Shorthand notation for $\cos(.)$}

%\nomenclature[CAD11]{$\md_{n}$}{Dynamic invariants column-matrix}
%\nomenclature[CAD91]{$\dd$}{Differential operator}
%\nomenclature[CGD91]{$\dl$}{Variation operator}

%\nomenclature[CAF01]{$f_{n,j}$}{Generalized force}
%\nomenclature[CAF11]{$\mf_{n}$}{Generalized forces column-matrix}
%\nomenclature[CAF21]{$\vf_{\llB}$}{Resultant force acting on body $\llB$ (excluding constraint forces)}

%\nomenclature[CAG01]{$g_{n,j}$}{Generalized gyroscopic inertia force}
%\nomenclature[CAG11]{$\mg_{n}$}{Generalized gyroscopic inertia forces column-matrix}

%\nomenclature[CAI21]{$\vI_{\llB \rl \ttp}$}{Inertia tensor of rigid body $\llB$ relative to point $\ttp$}
%\nomenclature[CAI51]{$\ssI_{x}(\ssS_{n})$}{Set of indexes of variables $x_{n,r}$ defined in the model 
%	of system $\ssS_{n}$, i.e., $\ssI_{x}(\ssS_{n}) = \{ r \,\vert\, x_{n,r} \in \ssS_{n} \}$ }
\nomenclature[CAG01]{$g$}{gravitational acceleration}
\nomenclature[CAG11]{$\mg\ssh$}{Generalized gravitational forces column-matrix of a serial mechanism}
\nomenclature[CAG12]{$\mg\ssh_i$}{Generalized gravitational forces column-matrix of a counter-rotating disc}
\nomenclature[CAG13]{$\mg'$}{Generalized uncoupled gravitational forces column-matrix of a serial mechanism coupled with counter-rotating discs}
\nomenclature[CAG14]{${\mg'}\ssh$}{Generalized gravitational forces column-matrix of a serial mechanism coupled with counter-rotating discs}

\nomenclature[CAJ01]{$J_{x_i}, J_{y_i}, J_{z_i}$}{Moments of inertia of the i\ts{th} rigid body of the mechanical system}

\nomenclature[CAL01]{$l_i$}{Length of the i\ts{th} bar of a serial mechanism}
\nomenclature[CAL01]{$l_{g_i}$}{Position of the mass center of the i\ts{th} bar relative to the i\ts{th} joint and of a serial mechanism}

\nomenclature[CAM01]{$m_i$}{Mass of the i\ts{th} rigid body of the mechanical system}
%\nomenclature[CAM21]{$\vm_{\llB \rl \ttp}$}{Resultant moment (torque) acting on body $\llB$ 
%	measured relatively to pole $\ttp$ (excluding constraint moments)}
\nomenclature[CAM11]{$\mM\ssh$}{Generalized inertia matrix of a serial mechanism}
\nomenclature[CAM12]{$\mM\ssh_i$}{Generalized inertia matrix of a counter-rotating disc}
\nomenclature[CAM13]{$\mM'$}{Generalized uncoupled inertia matrix of a serial mechanism coupled with counter-rotating discs}
\nomenclature[CAM14]{${\mM'}\ssh$}{Generalized inertia matrix of a serial mechanism coupled with counter-rotating discs}

\nomenclature[CAN41]{$\llN$}{Inertial reference frame}
%\nomenclature[CGN01]{$\nu_{x}(\ssS_{n})$}{Number of elements of the set $\ssI_{x}(\ssS_{n})$}
%\nomenclature[CGN02]{$\nu\ssh(\ssS_{n})$}{Number of degrees of freedom of the mechanical system $\ssS_{n}$}

%\nomenclature[CAP01]{$p_{n,j}$}{Quasi-velocity}
%\nomenclature[CAP02]{$\dot p_{n,j}$}{Quasi-acceleration}
\nomenclature[CAP11]{$\mp\ssh$}{Independent quasi-velocities column-matrix}
\nomenclature[CAP12]{$\mp^\circ$}{Redundant quasi-velocities column-matrix}
\nomenclature[CAP13]{$\mp$}{Quasi-velocities column-matrix}

\nomenclature[CAQ01]{$q_i$}{Generalized coordinate}
\nomenclature[CAQ11]{$\mq\ssh$}{Independent generalized coordinates column-matrix}

%\nomenclature[CAR21]{$\vr_{\ttp_2 \rl \ttp_1}$}{Position of point $\ttp_2$ relative to point $\ttp_1$}
% \nomenclature[CAR51]{$\ssR^s$}{Set of $s$-tuples of real numbers}

\nomenclature[CAS91]{$\ssin(.)$}{Shorthand notation for $\sin(.)$}

%\nomenclature[CAT01]{$t$}{Time}

\nomenclature[CAU01]{$u_i$}{Effort made by the i\ts{th} actuator of a serial mechanism}
\nomenclature[CAU11]{$\mu$}{Generalized actuators' efforts column-matrix}

%\nomenclature[CAV21]{$\vv_{\ttp \rl \llE}$}{Velocity of point $\ttp$ measured relatively to reference frame $\llE$}

\nomenclature[CAV11]{$\mv\ssh$}{Generalized coupled gyroscopic inertia forces column-matrix of a serial mechanism}
\nomenclature[CAV12]{$\mv\ssh_i$}{Generalized coupled gyroscopic inertia forces column-matrix of a counter-rotating disc}
\nomenclature[CAV13]{$\mv'$}{Generalized uncoupled gyroscopic inertia forces column-matrix of a serial mechanism coupled with counter-rotating discs}
\nomenclature[CAV14]{${\mv'}\ssh$}{Generalized coupled gyroscopic inertia forces column-matrix of a of a serial mechanism coupled with counter-rotating discs}

%\nomenclature[CAW01]{$w_{n,j}$}{Arbitrary term of generalized force or generalized gyroscopic inertia
%	force linear or bilinear with respect to quasi-velocities}
%\nomenclature[CAW11]{$\mw_{n}$}{Column-matrix whose entries are $w_{n,j}$}	
\nomenclature[CGW21]{$\nvct{\vomega_{\scriptscriptstyle i}}_{\ttB_j} $}{Angular velocity of the i\ts{th} rigid body of the mechanical system measured relatively to a inertial reference frame $\llN$, written in the basis of $\ttB_j$}
\nomenclature[CGW31]{$\omega_{x_i}$}{1\ts{st} component of $\nvct{\vomega_{\scriptscriptstyle i}}_{\ttB_i} $}
\nomenclature[CGW32]{$\omega_{y_i}$}{2\ts{nd} component of $\nvct{\vomega_{\scriptscriptstyle i}}_{\ttB_i} $}
\nomenclature[CGW33]{$\omega_{z_i}$}{3\ts{rd} component of $\nvct{\vomega_{\scriptscriptstyle i}}_{\ttB_i} $}

%\nomenclature[CAZ01]{$z_{n,j}$}{Arbitrary term of generalized force or generalized gyroscopic inertia
%	force independent of quasi-velocities}
%\nomenclature[CAZ11]{$\mz_{n}$}{Column-matrix whose entries are $z_{n,j}$}	

\nomenclature[CN011]{$\mzr$}{Null column-matrix or null matrix}
%\nomenclature[CN021]{$\vzr$}{Null vector or null tensor}

\nomenclature[CN111]{$\mone$}{Identity matrix}
\nomenclature[CN112]{$\nvct{\mone}_{\ttB_i \rl \ttB_j} $}{Change of basis matrix, i.e. $\nvct{\vv}_{\ttB_i} = \nvct{\mone}_{\ttB_i \rl \ttB_j} \cdot \nvct{\vv}_{\ttB_j} $ }
%\nomenclature[CN121]{$\vone$}{Identity tensor}


% matrix/vector notations

% \nomenclature[SM001]{$\nvct{x_{r}}$}{Column-matrix defined by the entries $x_{r}$}
% \nomenclature[SM011]{$\nmat{X_{rs}}$}{Matrix defined by the entries $X_{rs}$}
% \nomenclature[SM012]{$\ndmat{x_{r}}$}{Diagonal-matrix representation of $x_{r}$}

% \nomenclature[SM021]{$\nvct{\vw}_{\ttE}$}{Coordinate column-matrix of vector $\vw$
 	% in coordinate system $\ttE$}
% \nomenclature[SM022]{$\nvct{\ttp}_{\ttE}$}{Coordinate column-matrix of point $\ttp$
 	% in coordinate system $\ttE$}
% \nomenclature[SM023]{$\nhvct{\ttp}_{\ttE}$}{Homogeneous coordinates column-matrix of point $\ttp$
 	% in coordinate system $\ttE$} 

% \nomenclature[SM031]{$\nmat{\vZ}_{\ttE' \rl \ttE}$}{Matrix representing tensor $\vZ$
 	% in terms of coordinate systems $\ttE$ and $\ttE'$ (if $\vw'=\vZ \cdot \vw$, then 
 	% $\nvct{\vw'}_{\ttE'}= \nmat{\vZ}_{\ttE' \rl \ttE} \, \nvct{\vw}_{\ttE}$)}
% \nomenclature[SM032]{$\nsmat{\vw}_{\ttE \rl \ttE}$}{Skew-symmetric matrix representation
	% of $\nvct{\vw}_{\ttE}$} 	

\nomenclature[SM101]{$\ntmat{\cdot}$}{Matrix transposition}
%\nomenclature[SM102]{$\nimat{\cdot}$}{Matrix inversion}
%\nomenclature[SM103]{$\nomat{\cdot}$}{Matrix infinity norm}	




%--------------------INTRODUCTION--------------------%

\section{Introduction and literature review}\label{S01}


%--------------------METHODOLOGY--------------------%

\section{Mothodology}\label{S02}

\subsection{Dynamic Model}\label{S02-1}

O modelo din\^{a}mico de um mecanismo serial pode ser escrito da seguinte maneira:
\begin{equation}
\mM\ssh (\mq\ssh) \ddot{\mq}\ssh + \mv\ssh(\mq\ssh,\dot{\mq}\ssh) + \mg\ssh (\mq\ssh) = \mu
\end{equation}

Sendo $\mq\ssh$ um conjunto de coordenadas generalizadas independentes, cujos elementos s\~{a}o os deslocamentos relativos das juntas e $\mu$ os esfor\c{c}os generalizados nas dire\c{c}\~oes das quasi-velocidades independentes $\mp\ssh = \dot{\mq}\ssh$.

Para realizar o balanceamento din\^{a}mico de um mecanismo serial, utilizando abordagem proposta, \'{e} necess\'{a}rio primeiro obter o modelo do mecanismo desbalanceado. Como em um mecanismo serial \'{e} poss\'{i}vel de expressar todas as velocidades lineares absolutas dos centros de massa das barras e todas as velocidades angulares absolutas das barras em fun\c{c}\~ao de $\mq\ssh$ e  $\dot{\mq}\ssh$, o modelo din\^{a}mico pode ser obtido sem grandes dificuldades utilizando m\'{e}todos de mec\'{a}nica anal\'{i}tica, como Lagrange, Kane e Orsino, aliados a programas ou bibliotecas de linguagens de programa\c{c}\~{a}o que s\~{a}o capazes de utilizar manipula\c{c}\~{a}o simb\'{o}lica, como o Mathematica e o SymPy.


\subsection{Static Balancing}\label{S02-2}

Depois de obtido o modelo din\^{a}mico, realiza-se o balanceamento est\'{a}tico encontrando as posi\c{c}\~{o}es dos centros de massa das barras que fazem com que $\mg\ssh = \mzr$. Isso \'{e} poss\'{i}vel para mecanismos com juntas apenas rotativas e mecanismos com juntas prism\'{a}ticas cuja dire\c{c}\~{o}es s\~{a}o ortogonais \`{a} gravidade. O posicionamento dos centros de massa \'e realizado mecanicamente com o prologamento das barras do mecanismo e adi\c{c}\~ao de contra-pesos.

\subsection{Dynamic Balancing}\label{S02-3}

O balanceamento din\^{a}mico \'{e} obtido acoplando discos girantes ao modelo do mecanismo est\'{a}ticamente balanceado. Isso \'{e} feito utilizando a t\'{e}cnica de acoplamento de subsistemas do M\'{e}todo Orsino.


%--------------------EXAMPLE---------------------%

\section{Applying the technique}\label{S03}

\begin{equation}
\mM\ssh (\mq\ssh) \ddot{\mq}\ssh + \mv\ssh(\mq\ssh,\dot{\mq}\ssh) + \mg\ssh (\mq\ssh) = \mu
\end{equation}

For a 3-DOF serial mechanism:
\begin{equation}
\mM\ssh = 
\begin{bmatrix}
D_{11} & D_{12} & D_{13} \\
D_{12} & D_{22} & D_{23} \\
D_{13} & D_{23} & D_{33} \\
\end{bmatrix}
\end{equation}
\begin{equation}
\mv\ssh = 
\begin{bmatrix}
D_{111} & D_{122} & D_{133} \\
D_{211} & D_{222} & D_{233} \\
D_{311} & D_{322} & D_{333} \\
\end{bmatrix}
\begin{bmatrix}
\dot{q}_1^2 \\
\dot{q}_2^2 \\
\dot{q}_3^2 \\
\end{bmatrix}
+
2\begin{bmatrix}
D_{112} & D_{113} & D_{123} \\
D_{212} & D_{213} & D_{223} \\
D_{312} & D_{313} & D_{323} \\
\end{bmatrix}
\begin{bmatrix}
\dot{q}_1 \dot{q}_2 \\
\dot{q}_1 \dot{q}_3 \\
\dot{q}_2 \dot{q}_3 \\
\end{bmatrix}
\end{equation}
\begin{equation}
\mg\ssh =
\begin{bmatrix}
D_1 &
D_2 &
D_3
\end{bmatrix}^\top
\end{equation}
\begin{equation}
\mq\ssh =
\begin{bmatrix}
q_1 &
q_2 &
q_3
\end{bmatrix}^\top
\end{equation}
\begin{equation}
\mu =
\begin{bmatrix}
u_1 &
u_2 &
u_3
\end{bmatrix}^\top
\end{equation}

For a rotational joint we call $q_i = \theta_i$ and $u_i = \tau_i$, and for a prismatic joint we call $q_i = d_i$ and $u_i = f_i$.

\subsection{3-DOF RRR planar serial mechanism}\label{S03-1}

\begin{equation}
\begin{cases}
D_1 = g[ (m_1 l_{g_1} + m_2 l_1 + m_3 l_1 ) \ccos(\theta_1) + (m_2 l_{g_2} + m_3 l_2) \ccos(\theta_1 + \theta_2) + m_3 l_{g_3} \ccos(\theta_1 + \theta_2 + \theta_3) ] \\
D_2 = g[  (m_2 l_{g_2} + m_3 l_2) \ccos(\theta_1 + \theta_2) + m_3 l_{g_3} \ccos(\theta_1 + \theta_2 + \theta_3) ] \\
D_3 = g[   m_3 l_{g_3} \ccos(\theta_1 + \theta_2 + \theta_3) ] \\
\end{cases}
\end{equation} 

Static balancing:
\begin{equation}
\begin{cases}
D_1 = 0 \\
D_2 = 0 \\
D_3 = 0 \\
\end{cases}
\Rightarrow
\begin{cases}
l_{g_1} = -\frac{l_1(m_2 + m_3)}{m_1} \\
l_{g_2} = -\frac{l_2 m_3}{m_2} \\
l_{g_3} = 0 \\
\end{cases}
\end{equation}

Static balanced mechanism:
\begin{equation}
\begin{cases}
D_{11} = J_{x_1} + J_{x_2} + J_{x_3} + m_2 l_1^2 + m_3 (l_1^2 + l_2^2) + \frac{l_1^2 (m_2 + m_3)^2}{m_1} + \frac{l_2^2 m_3^2}{m_2} \\
D_{22} = J_{x_2} + J_{x_3} + m_3 l_2^2 + \frac{l_2^2 m_3^2}{m_2} \\
D_{33} = J_{x_3} \\
D_{12} = D_{22} \\
D_{13} = D_{23} = D_{33} \\
\mv\ssh = \mzr \\
\mg\ssh = \mzr
\end{cases}
\end{equation}

Coupling 4 discs: \\

Planar disc models: 
\begin{equation}
\mM\ssh_i = \begin{bmatrix} J_{x_{i+3}} \end{bmatrix}; \; \mv\ssh_i = \begin{bmatrix} 0 \end{bmatrix}; \; \mg\ssh_i = \begin{bmatrix} 0 \end{bmatrix}; \; \mu_i = \begin{bmatrix} 0 \end{bmatrix}; \; \mp\ssh_i = \begin{bmatrix} \omega_{x_{i+3}} \end{bmatrix}, \; i = 1, 2, 3, 4
\end{equation}

Quasi-velocities constraints:

\begin{equation}
\begin{cases}
\omega_{x_4} = \omega_{x_1} + \dot{\theta}_2 \\
\omega_{x_5} = \omega_{x_1} + \beta\dot{\theta}_2 \\
\omega_{x_6} = \omega_{x_2} + \dot{\theta}_3 \\
\omega_{x_7} = \omega_{x_2} + \gamma\dot{\theta}_3 \\
\end{cases}
\Rightarrow
\begin{cases}
\omega_{x_4} = \dot{\theta}_1 + \dot{\theta}_2 \\
\omega_{x_5} = \dot{\theta}_1 + \beta\dot{\theta}_2 \\
\omega_{x_6} = \dot{\theta}_1 + \dot{\theta}_2 + \dot{\theta}_3 \\
\omega_{x_7} = \dot{\theta}_1 + \dot{\theta}_2 + \gamma\dot{\theta}_3 \\
\end{cases}
\therefore
\overline{\mp} = 
\begin{bmatrix}
\omega_{x_4} - \dot{\theta}_1 - \dot{\theta}_2 \\
\omega_{x_5} - \dot{\theta}_1 - \beta\dot{\theta}_2 \\
\omega_{x_6} - \dot{\theta}_1 - \dot{\theta}_2 - \dot{\theta}_3 \\
\omega_{x_7} - \dot{\theta}_1 - \dot{\theta}_2 - \gamma\dot{\theta}_3 \\
\end{bmatrix}
=\mzr
\end{equation}

\begin{equation}
\mA\ssh = \frac{\partial \overline{\mp}}{\partial \mp\ssh} = -
\begin{bmatrix}
1 & 1     & 0 \\
1 & \beta & 0 \\
1 & 1     & 1 \\
1 & 1     & \gamma \\
\end{bmatrix}; \;
\mA^\circ = \frac{\partial \overline{\mp}}{\partial \mp^\circ} = \mone; \;
\mC =
\begin{bmatrix}
\mone \\
-(\mA^\circ)^{-1} \mA\ssh
\end{bmatrix}  =
\begin{bmatrix}
1 & 0 & 0 \\
0 & 1 & 0 \\
0 & 0 & 1 \\
1 & 1     & 0 \\
1 & \beta & 0 \\
1 & 1     & 1 \\
1 & 1     & \gamma \\
\end{bmatrix} 
\end{equation}

\begin{equation}
\begin{cases}
D'_{11} = D_{11} + J_{x_4} + J_{x_5} + J_{x_6} + J_{x_7} \\
D'_{22} = D_{22} + J_{x_4} + J_{x_5} \beta^2 + J_{x_6} + J_{x_7} \\
D'_{33} = D_{33} + J_{x_6} + J_{x_7} \gamma^2 \\
D'_{12} = D_{12} + J_{x_4} + J_{x_5} \beta + J_{x_6} + J_{x_7} \\
D'_{13} = D_{13} + J_{x_6} + J_{x_7} \gamma  \\
D'_{23} = D'_{13}\\
{\mv'}\ssh = \mzr \\
\end{cases}
\end{equation}

Dynamic balancing:
\begin{equation}
\begin{cases}
D'_{12} = 0 \\
D'_{13} = 0 \\
\end{cases}
\Rightarrow
\begin{cases}
\beta = -\frac{J_{x_2} + J_{x_3} + J_{x_4} + J_{x_6} + J_{x_7} + m_3 l_2^2 + \frac{m_3^2 l_2^2}{m_2}}{J_{x_5}} \\
\gamma = -\frac{J_{x_3} + J_{x_6}}{J_{x_7}} \\
\end{cases}
\end{equation}

Dynamic balanced mechanism:
\begin{equation}
\begin{cases}
\tau_1 = k_1 \ddot{\theta}_1 \\
\tau_2 = k_2 \ddot{\theta}_2 \\
\tau_3 = k_3 \ddot{\theta}_3 \\
\end{cases}
\end{equation}

Being:
\begin{equation}
\begin{cases}
k_1 = J_{x_1} + J_{x_2} + J_{x_3} + J_{x_4} + J_{x_5} + J_{x_6} + J_{x_7} + m_2 l_1^2 + m_3 (l_1^2 + l_2^2) + \frac{l_1^2 (m_2 + m_3)^2}{m_1} + \frac{l_2^2 m_3^2}{m_2} \\
k_2 = J_{x_2} + J_{x_3} + J_{x_4} + J_{x_6} + J_{x_7} + m_3 l_2^2 + \frac{l_2^2 m_3^2}{m_2} + \frac{\big(J_{x_2} + J_{x_3} + J_{x_4} + J_{x_6} + J_{x_7} + m_3 l_2^2 + \frac{m_3^2 l_2^2}{m_2}\big)^2}{J_{x_5}} \\
k_3 = \frac{(J_{x_3}+J_{x_6})(J_{x_3}+J_{x_6}+J_{x_7})}{J_{x_7}} \\
\end{cases}
\end{equation}

\subsection{3-DOF RRR spatial serial mechanism} \label{S03-2}

\begin{equation}
\begin{cases}
D_1 = 0 \\
D_2 = g[ (m_2 l_{g_2} + m_3 l_2  ) \ccos(\theta_2) + m_3 l_{g_3} \ccos(\theta_2 + \theta_3) ]  \\
D_3 = g[ m_3 l_{g_3} \ccos(\theta_2 + \theta_3)  ] \\
\end{cases}
\end{equation} 

Static balancing:
\begin{equation}
\begin{cases}
D_2 = 0 \\
D_3 = 0 \\
\end{cases}
\Rightarrow
\begin{cases}
l_{g_2} = -\frac{l_2 m_3}{m_2} \\
l_{g_3} = 0 \\
\end{cases}
\end{equation}

Static balanced mechanism:
\begin{equation}
\begin{cases}
D_{11} = J_{y_2}\ssin^2(\theta_2) + J_{y_3}\ssin^2(\theta_2+\theta_3) + J_{z_1} + J_{z_2}\ccos^2(\theta_2) + J_{z_3}\ccos^2(\theta_2+\theta_3) + m_3 (l_1 + l_2 \ccos(\theta_2))^2 + \frac{(m_2 l_1 - m_3 l_2 \ccos(\theta_2))^2}{m_2} \\
D_{22} = J_{x_2} + J_{x_3} + m_2 l_2^2 + \frac{l_2^2 m_3^2}{m_2} \\
D_{33} = J_{x_3} \\
D_{12} = D_{13} = 0 \\
D_{23} = D_{33} \\
D_{211} = -\frac{1}{2} \Big( \big( J_{y_2} - J_{z_2} \big) \ssin(2\theta_2) + \big(J_{y_3} - J_{z_3} - m_3 l_2^2 ( 1 + \frac{m_3}{m_2} )\big)\ssin(2\theta_2+2\theta_3)    \Big) \\
D_{311} = \frac{1}{2} \Big( \big(J_{z_3} - J_{y_3} \big)\ssin(2\theta_2+2\theta_3) \Big) \\
D_{111} = D_{122} = D_{133} = D_{222} = D_{233} = D_{322} = D_{333} = 0 \\
D_{112} = - D_{211} \\
D_{113} = - D_{311} \\
D_{123} = D_{212} = D_{213} = D_{223} = D_{312} = D_{313} = D_{323} = 0 \\
\mg = \mzr
\end{cases}
\end{equation}

Coupling 2 discs: \\

Spatial disc models: 
\begin{equation}
\mM\ssh_i = \begin{bmatrix} J_{x_{i+3}} & 0 & 0 \\ 0 & J_{y_{i+3}} & 0 \\ 0 & 0 & J_{z_{i+3}} \end{bmatrix}; \; \mv\ssh_i = \begin{bmatrix} 0 \\ 0 \\ 0 \end{bmatrix}; \; \mg\ssh_i = \begin{bmatrix} 0 \\ 0 \\ 0 \end{bmatrix}; \; \mu_i = \begin{bmatrix} 0 \\ 0 \\ 0  \end{bmatrix}; \; \mp\ssh_i = \begin{bmatrix} \omega_{x_{i+3}} \\ \omega_{y_{i+3}} \\ \omega_{z_{i+3}} \end{bmatrix}, \; i = 1, 2
\end{equation}

Quasi-velocities constraints:

\begin{equation}
\begin{cases}
\nvct{\vomega_{\scriptscriptstyle 4}}_{\ttB_4} = \nvct{\mone}_{\ttB_4 \rl \ttB_2} \nvct{\vomega_{\scriptscriptstyle 2}}_{\ttB_2} + \begin{bmatrix} \dot{\theta}_3 \\ 0 \\0 \end{bmatrix} \\
\nvct{\vomega_{\scriptscriptstyle 5}}_{\ttB_5} = \nvct{\mone}_{\ttB_5 \rl \ttB_2} \nvct{\vomega_{\scriptscriptstyle 2}}_{\ttB_2} + \begin{bmatrix} \beta\dot{\theta}_3 \\ 0 \\0 \end{bmatrix} \\
\end{cases}
\Rightarrow
\begin{cases}
\omega_{x_4} = \dot{\theta}_2 + \dot{\theta}_3 \\
\omega_{y_4} = (\dot{\theta}_1 \ssin(\theta_2))\ccos(\theta_3) + (\dot{\theta}_1 \ccos(\theta_2))\ssin(\theta_3) \\
\omega_{z_4} = -(\dot{\theta}_1 \ssin(\theta_2))\ssin(\theta_3) + (\dot{\theta}_1 \ccos(\theta_2))\ccos(\theta_3) \\
\omega_{x_5} = \dot{\theta}_2 + \beta\dot{\theta}_3 \\
\omega_{y_5} = (\dot{\theta}_1 \ssin(\theta_2))\ccos(\beta\theta_3) + (\dot{\theta}_1 \ccos(\theta_2))\ssin(\beta\theta_3) \\
\omega_{z_5} = -(\dot{\theta}_1 \ssin(\theta_2))\ssin(\beta\theta_3) + (\dot{\theta}_1 \ccos(\theta_2))\ccos(\beta\theta_3) \\
\end{cases}
\end{equation}

\begin{equation}
\therefore
\overline{\mp} = 
\begin{bmatrix}
\omega_{x_4} - \dot{\theta}_2 - \dot{\theta}_3 \\
\omega_{y_4} - \dot{\theta}_1 \ssin(\theta_2+\theta_3) \\
\omega_{z_4} - \dot{\theta}_1 \ccos(\theta_2+\theta_3) \\
\omega_{x_5} - \dot{\theta}_2 - \beta\dot{\theta}_3 \\
\omega_{y_5} - \dot{\theta}_1 \ssin(\theta_2+\beta\theta_3) \\
\omega_{z_5} - \dot{\theta}_1 \ccos(\theta_2+\beta\theta_3) \\
\end{bmatrix}
=\mzr
\end{equation}

\begin{equation}
\mA\ssh = \frac{\partial \overline{\mp}}{\partial \mp\ssh} = -
\begin{bmatrix}
0 & 1 & 1 \\
\ssin(\theta_2+\theta_3) & 0 & 0 \\
\ccos(\theta_2+\theta_3) & 0 & 0 \\
0 & 1 & \beta \\
\ssin(\theta_2+\beta\theta_3) & 0 & 0 \\
\ccos(\theta_2+\beta\theta_3) & 0 & 0 \\
\end{bmatrix}; \;
\mA^\circ = \frac{\partial \overline{\mp}}{\partial \mp^\circ} = \mone; \;
\mC =
\begin{bmatrix}
\mone \\
-(\mA^\circ)^{-1} \mA\ssh
\end{bmatrix}  =
\begin{bmatrix}
1 & 0 & 0 \\
0 & 1 & 0 \\
0 & 0 & 1 \\
0 & 1 & 1 \\
\ssin(\theta_2+\theta_3) & 0 & 0 \\
\ccos(\theta_2+\theta_3) & 0 & 0 \\
0 & 1 & \beta \\
\ssin(\theta_2+\beta\theta_3) & 0 & 0 \\
\ccos(\theta_2+\beta\theta_3) & 0 & 0 \\
\end{bmatrix} 
\end{equation}

\begin{equation}
\begin{cases}
D'_{11} = D_{11} + J_{y_4}\ssin^2(\theta_2+\theta_2) + J_{y_5}\ssin^2(\beta\theta_2+\theta_2) + J_{z_4}\ccos^2(\theta_2+\theta_2) + J_{z_5}\ccos^2(\beta\theta_2+\theta_2) \\
D'_{22} = D_{22} + J_{x_4} + J_{x_5} \\
D'_{33} = D_{33} + J_{x_4} + J_{x_5} \beta^2\\
D'_{12} = D'_{13} = 0\\
D'_{23} = D_{23} + J_{x_4} + J_{x_5} \beta \\
D'_{211} = D_{211} \\
D'_{311} = D_{311} \\
D'_{111} = D'_{122} = D'_{133} = D'_{222} = D'_{233} = D'_{322} = D'_{333} = 0 \\
D'_{112} = D_{112} +  \frac{1}{4} \Big( \big(J_{y_4} - J_{z_4}\big)\ssin(2\theta_2+2\theta_3) + \big(J_{y_5} - J_{z_5}\big)\ssin(2\beta\theta_2+2\theta_3) \Big) \\
D'_{113} =  D_{113} + \frac{1}{4} \Big( (J_{y_4} - J_{z_4})\ssin(2\theta_2+2\theta_3) + (J_{y_5} - J_{z_5})\ssin(2\beta\theta_2+2\theta_3) \Big) \\
D'_{123} = D'_{212} = D'_{213} = D'_{223} = D'_{312} = D'_{313} = D'_{323} = 0 \\
\end{cases}
\end{equation}

Dynamic balancing:
\begin{equation}
\begin{cases}
D'_{23} = 0 \\
D'_{211} = 0 \\
D'_{311} = 0 \\
D'_{112} = 0 \\
D'_{113} = 0 \\
\end{cases}
\Rightarrow
\begin{cases}
\beta = -\frac{J_{x_3}+J_{x_4}}{J_{x_5}} \\
J_{y_2} = J_{z_2} \\
J_{y_3} = J_{z_3} + m_3 l_2^2 ( 1 + \frac{m_3}{m_2} )\\
J_{y_4} = J_{z_4} \\
J_{y_5} = J_{z_5} \\
\end{cases}
\end{equation}

Dynamic balanced mechanism:
\begin{equation}
\begin{cases}
\tau_1 = k_1 \ddot{\theta}_1 \\
\tau_2 = k_2 \ddot{\theta}_2 \\
f_3 = k_3 \ddot{d}_3 \\
\end{cases}
\end{equation}

Being:
\begin{equation}
\begin{cases}
k_1 = J_{z_1} + J_{z_2} + J_{z_3} + J_{z_4} + J_{z_5} + m_2 l_1^2 + m_3 (l_1^2 + l_2^2) + \frac{l_1^2 m_3^2}{m_2} \\
k_2 =  J_{x_2} + J_{x_3} + J_{x_4} + J_{x_5} + m_3 l_2^2 + \frac{l_1^2 m_3^2}{m_2}\\
k_3 = \frac{(J_{x_3}+J_{x_4})(J_{x_3}+J_{x_4}+J_{x_5})}{J_{x_5}} \\
\end{cases}
\end{equation}

\subsection{3-DOF RRP spatial serial mechanism (SCARA)} \label{S03-2}

\begin{equation}
\begin{cases}
D_1 = g[ (m_1 l_{g_1} + m_2 l_1 + m_3 l_1 ) \ccos(\theta_1) + m_2 l_{g_2} \ccos(\theta_1+\theta_2) ] \\
D_2 = g[ m_2 l_{g_2} \ccos(\theta_1+\theta_2)  ] \\
D_3 = 0 \\
\end{cases}
\end{equation} 

Static balancing:
\begin{equation}
\begin{cases}
D_1 = 0 \\
D_2 = 0 \\
\end{cases}
\Rightarrow
\begin{cases}
l_{g_1} = -\frac{l_1 (m_2+m_3)}{m_1} \\
l_{g_2} = 0 \\
\end{cases}
\end{equation}

Static balanced mechanism:
\begin{equation}
\begin{cases}
D_{11} = J_{x_1} + J_{x_2} + J_{x_3} + m_2 l_1^2 + m_3 l_1^2 + \frac{l_1^2 (m_2 + m_3)^2}{m_1} \\
D_{22} = J_{x_2} + J_{x_3} \\
D_{33} = m_3 \\
D_{12} = D_{22} \\
D_{13} = D_{23} = 0 \\
\mv = \mzr \\
\mg = \mzr
\end{cases}
\end{equation}

Coupling 2 discs: \\

Planar disc models: 
\begin{equation}
\mM\ssh_i = \begin{bmatrix} J_{x_{i+3}} \end{bmatrix}; \; \mv\ssh_i = \begin{bmatrix} 0 \end{bmatrix}; \; \mg\ssh_i = \begin{bmatrix} 0 \end{bmatrix}; \; \mu_i = \begin{bmatrix} 0 \end{bmatrix}; \; \mp\ssh_i = \begin{bmatrix} \omega_{x_{i+3}} \end{bmatrix}, \; i = 1, 2
\end{equation}

Quasi-velocities constraints:

\begin{equation}
\begin{cases}
\omega_{x_4} = \omega_{x_1} + \dot{\theta}_2 \\
\omega_{x_5} = \omega_{x_1} + \beta\dot{\theta}_2 \\
\end{cases}
\Rightarrow
\begin{cases}
\omega_{x_4} = \dot{\theta}_1 + \dot{\theta}_2 \\
\omega_{x_5} = \dot{\theta}_1 + \beta\dot{\theta}_2 \\
\end{cases}
\therefore
\overline{\mp} = 
\begin{bmatrix}
\omega_{x_4} - \dot{\theta}_1 - \dot{\theta}_2 \\
\omega_{x_5} - \dot{\theta}_1 - \beta\dot{\theta}_2 \\
\end{bmatrix}
=\mzr
\end{equation}

\begin{equation}
\mA\ssh = \frac{\partial \overline{\mp}}{\partial \mp\ssh} = -
\begin{bmatrix}
1 & 1     & 0 \\
1 & \beta & 0 \\
\end{bmatrix}; \;
\mA^\circ = \frac{\partial \overline{\mp}}{\partial \mp^\circ} = \mone; \;
\mC =
\begin{bmatrix}
\mone \\
-(\mA^\circ)^{-1} \mA\ssh
\end{bmatrix}  =
\begin{bmatrix}
1 & 0 & 0 \\
0 & 1 & 0 \\
0 & 0 & 1 \\
1 & 1     & 0 \\
1 & \beta & 0 \\
\end{bmatrix} 
\end{equation}

\begin{equation}
\begin{cases}
D'_{11} = D_{11} + J_{x_4} + J_{x_5} \\
D'_{22} = D_{22} + J_{x_4} + J_{x_5} \beta^2 \\
D'_{33} = D_{33} \\
D'_{12} = D_{12} + J_{x_4} + J_{x_5} \beta \\
D'_{13} = 0 \\
D'_{23} = 0\\
\mv' = \mzr \\
\end{cases}
\end{equation}

Dynamic balancing:
\begin{equation}
D'_{12} = 0 \Rightarrow \beta = -\frac{J_{x_2}+J_{x_3}+J_{x_4}}{J_{x_5}}
\end{equation}

Dynamic balanced mechanism:
\begin{equation}
\begin{cases}
\tau_1 = k_1 \ddot{\theta}_1 \\
\tau_2 = k_2 \ddot{\theta}_2 \\
f_3 = k_3 \ddot{d}_3 \\
\end{cases}
\end{equation}

Being:
\begin{equation}
\begin{cases}
k_1 = J_{x_1} + J_{x_2} + J_{x_3} + J_{x_4} + J_{x_5} + m_2 l_1^2 + m_3 l_1^2 + \frac{l_1^2 (m_2 + m_3)^2}{m_1} \\
k_2 = \frac{(J_{x_2}+J_{x_3}+J_{x_4})(J_{x_2}+J_{x_3}+J_{x_4}+J_{x_5})}{J_{x_5}} \\
k_3 = m_3 \\
\end{cases}
\end{equation}

%--------------------CONCLUSIONS--------------------%

%\newpage

\section{Conclusions}\label{S04}


%--------------------ACKNOWLEDGMENTS--------------------%

%\section*{Acknowledgments}


%--------------------BIBLIOGRAPHY--------------------%

% \newpage
\phantomsection 


\begin{thebibliography}{99}


\bibitem{1wijk} 
V. Van der Wijk,
\newblock Shaking moment balancing of mechanisms with principal vectors and moments
\newblock {\em Front. Mech. Eng.}, 8(1): 10--16, 2013.
 
\bibitem{2arakelian}
V. H. Arakelian V. , M. R. Smith,
\newblock Design of planar 3-dof 3-RRR reactionless parallel manipulators
\newblock {\em Mechatronics}, 18: 601--606, 2008. 
 
\bibitem{3seo}
J.-T. Seo, J. H. Woo, H. Lim, J. Chung, W. K. Kim, and B.-J. Yi,
\newblock Design of an Antagonistically Counter-Balancing Parallel Mechanism
\newblock {\em IEEE/RSJ International Conference on
Intelligent Robots and Systems (IROS)}, Tokyo, November 3-7: 2882--2887, 2013. 

\bibitem{4wu}
Y. Wu, C. M. Gosselin,
\newblock Design of reactionless 3-dof and 6-dof parallel manipulators using parallelepiped mechanisms
\newblock {\em IEEE Transactions on Robotics}, 21(5): 821--833, 2005.

\bibitem{5gosselin}
C. M. Gosselin, F. Vollmer, G. C�t�, Y. Wu,
\newblock Synthesis and design of reactionless three-degree-of-freedom parallel mechanisms
\newblock {\em IEEE Transactions on Robotics and Automation}, 20(2): 191--199, 2004.

\bibitem{6wang}
J. Wang, C. M. Gosselin,
\newblock Static balancing of spatial four-degree-of-freedom parallel mechanisms
\newblock {\em Mech. Mach. Theory}, 35: 563--592, 2000.

\bibitem{7wang}
J. Wang, C. M. Gosselin,
\newblock Static balancing of spatial three-degree-of-freedom parallel mechanisms
\newblock {\em Mech. Mach. Theory}, 34: 437--452, 1999.

\bibitem{8alici}
G. Alici, B. Shirinzadeh,
\newblock Optimum Force Balancing with Mass Distribution and a Single Elastic Element for a
Five-bar Parallel Manipulator
\newblock {\em Proceedings of the IEEE International Conference on Robotics and Automation},Taipei, September 14-19: 3666--3671, 2003.

\bibitem{9alici}
G. Alici, B. Shirinzadeh,
\newblock Optimum dynamic balancing of planar parallel
manipulators based on sensitivity analysis
\newblock {\em Mech. Mach. Theory}, 41: 1520--1532, 2006.

\bibitem{10dehkordi}
M. B. Dehkordi, A. Frisoli, E. Sotgiu, M. Bergamasco,
\newblock Modelling and Experimental Evaluation of a Static Balancing Technique for
a new Horizontally Mounted 3-UPU Parallel Mechanism
\newblock {\em International Journal of Advanced Robotic Systems}, 9: 193--205, 2012.

\bibitem{11wang}
K. Wang, M. Luo, T. Mei, J. Zhao, Y. Cao,
\newblock Dynamics Analysis of a Three-DOF Planar Serial-Parallel Mechanism for Active
Dynamic Balancing with Respect to a Given Trajectory
\newblock {\em International Journal of Advanced Robotic Systems}, 10: 23--33, 2013.

\bibitem{12russo}
A. Russo, R. Sinatra, F. Xi,
\newblock Static balancing of parallel robots
\newblock {\em Mech. Mach. Theory}, 40: 191--202, 2005.

\bibitem{13agrawal}
S. K. Agrawal, A. Fattah,
\newblock Gravity-balancing of spatial robotic manipulators
\newblock {\em Mech. Mach. Theory}, 39: 1331--1344, 2004.

\bibitem{14briot}
S. Briot, V. Arakelian, J.-P. Le Baron,
\newblock Shaking force minimization of high-speed robots via centre of mass
acceleration control
\newblock {\em Mech. Mach. Theory}, 57: 1--12, 2012.

\bibitem{15coelho}
T. A. H. Coelho, L. Yong, V. F. A. Alves,
\newblock Decoupling of dynamic equations by means of
adaptive balancing of 2-dof open-loop mechanisms
\newblock {\em Mech. Mach. Theory}, 39: 871--881, 2004.

\bibitem{16moradi}
M. Moradi, A. Nikoobin, S. Azadi,
\newblock Adaptive Decoupling for Open Chain Planar Robots
\newblock {\em Transaction B: Mechanical Engineering}, 17(5): 376--386, 2010.

\bibitem{17arakelian}
V. Arakelian, S. Sargsyan,
\newblock On the design of serial manipulators with decoupled dynamics
\newblock {\em Mechatronics}, 22(6): 904--909, 2012.

\bibitem{18tsai}
J. Chen, D.Z. Chen, L.W. Tsai,
\newblock A Systematic Methodology for the Dynamic Analysis of Articulated Gear-Mechanisms,
\newblock 1990.

\bibitem{19kane}
T. R. Kane, D. A. Levinson,
\newblock {\em {Dynamics, Theory and Applications}}.
\newblock McGraw-Hill series in mechanical engineering. McGraw Hill, 1985.

\bibitem{20altuzarra}
O. Altuzarra, P. M. Eggers, F. J. Campa, C. Roldan-Paraponiaris, C. Pinto, 
\newblock Dynamic Modelling of Lower-Mobility Parallel Manipulators Using the Boltzmann-Hamel Equations
\newblock {\em Mechanisms, Transmissions and Applications}, 31: 157--165, 2015.

\bibitem{21orsino}
R. M. M. Orsino, T. A. H. Coelho, C. P. Pesce, 
\newblock Analytical mechanics approaches in the dynamic modelling of Delta mechanism
\newblock {\em Robotica}, 33(4): 953--973, 2015.

\bibitem{22orsino}
R. M. M. Orsino, A. G. Coutinho, T. A. H. Coelho,
\newblock Dynamic modelling and control of balanced parallel mechanisms.
\newblock Book chapter of {\em Dynamic Balancing of Mechanisms and Synthesizing of Parallel Robots}, Springer, 2016 (in press).

\bibitem{23orsino}
R. M. M. Orsino, T. A. H. Coelho (2015).
\newblock A contribution on the modular modelling of multibody systems.
\newblock Manuscript submitted for publication

\end{thebibliography}


%\end{document}


% \end{multicols}


\end{document}

