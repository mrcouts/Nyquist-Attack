\subsection{Controle por modos deslizantes extendido}\label{S02-4}

Como foi visto na se\c{c}\~ao de modelagem, \'e muito conveniente utilizar coordenadas redundantes para realizar a modelagem de mecanismos paralelos. Sendo assim, propomos nesta se\c{c}\~ao uma lei de controle para sistemas descritos por coordenadas redundantes.

Let $\ssM$ be a multibody mechanical system whose mathematical model is given 
by equations~(\ref{eq:02-207A},~\ref{eq:02-207B}).
Suppose that each $\mf_{n}$ is an affine function of the control inputs $u_{n,k}$
in which the coefficients of the $u_{n,k}$ may depend on the instantaneous
configuration of the system.
Suppose additionally that that all the $\mA_{n}$ are independent
of the quasi-velocities $p_{n,j}$ and all the $\mM_{n}$, $\mg_{n}$, $\mA_{n}$ and $\mb_{n}$ 
are independent of the $u_{n,k}$.
Under these conditions, matrices $\mC_{n}$ will not depend on 
any quasi-velocity, $\md$ can be expressed as an affine function of the control inputs 
and $\mc$ is independent of them.
Considering that the number of control inputs in a mechanical system is
exactly equal to the number of degrees of freedom of $\ssM$, 
it may be possible to solve equations~(\ref{eq:02-207B}) in order to
express each $u_{n,k}$ as an explicit function of state variables and
physical parameters of $\ssM$.
For the sake of brevity, omitting the indexes $n$ of the subsystems of $\ssM$,
it can be stated that, in such cases, for $k \in \{ 1,\ldots,\nu\ssh(\ssM) \}$:
\begin{align}
&	u_{k} = \sum_{r} C_{kr} (t,\mq) \Big(  \sum_{i} M_{ri}(t,\mq) \, \dot p_i 
		+ w_{i}(t, \mq, \mp)
		+ z_{i}(t, \mq) \Big)
	\label{eq:03-101Especial} 
\end{align}
That is equivalent to:
\begin{align}
&	u_{k} = \sum_{r} M'_{kr}(t,\mq) \, \dot p_r 
		+ w'_{k}(t, \mq, \mp)
		+ z'_{k}(t, \mq)
	\label{eq:03-101A} 
\end{align}
It can be shown that each $w'_{k}(t, \mq, \mp)$ can be expressed as a sum of 
a bilinear and a linear function on the quasi-velocities, i.e., there are functions
$D'_{krs}(t,\mq)$ and $B'_{kr}(t,\mq)$ such that:
\begin{align}	
&	w'_{k}(t, \mq, \mp) = \sum_{r} \sum_{s} D'_{krs}(t,\mq) \, p_r \, p_s
		+ \sum_{r} B'_{kr}(t,\mq) \, p_r
	\label{eq:03-101B}	
\end{align}
Adaptive balancing types will be defined in the next section based on equation~(\ref{eq:03-101A}).

For keeping the sake of brevity, the indexes $n$ of the subsystems of $\ssM$ will be omitted from this point in this sub-section.

Para o ponto de vista do controle, \'e conveniente utilizar $\dot{\mq}$ no lugar de $\mp$, pois muitas vezes as velocidades generalizadas $\mp$ s\~ao n\~ao integr\'aveis, o que impossibilita a realimenta\c{c}\~ao de posi\c{c}\~ao na dire\c{c}\~ao destas coordenadas, ent\~ao assim faremos. \\

Baseado nas equa\c{c}\~oes \eqref{eq:03-101Especial}  e \eqref{eq:02-201A}, seja o modelo de um sistema mec\^anico multi-corpos descrito pelas seguintes equa\c{c}\~oes:
\begin{equation} \label{eq:MechanicalSystem}
\begin{cases}
\mC^\msT (\mq) \Big( \mM (\mq) \ddot{\mq} + \mw (\mq, \dot{\mq}) + \mz (\mq) \Big) = \mu \\
\mA (\mq) \ddot{\mq} + \mb (\mq, \dot{\mq}) = \mzr
\end{cases}
\end{equation}

De maneira matricial compacta:
\begin{equation} \label{eq:MechanicalSystemMatrix}
\begin{bmatrix}
\mC^\msT \mM \\
\mA
\end{bmatrix}
\ddot{\mq}
=
\begin{bmatrix}
\mu - \mC^\msT(\mw + \mz) \\
-\mb
\end{bmatrix}
\end{equation}

Gostaria que $ \ddot{\mq} = \mv $, sendo $\mv$ uma entrada de controle. Para que isso aconte\c{c}a, utilizamos a seguinte lei de controle:
\begin{equation} \label{eq:ControlLawV}
\mu = \mC^\msT ( \mM \mv + \mw + \mz )
\end{equation}

Como queremos que $ \ddot{\mq} = \mv $ e $\ddot{\mq}$ tem restri\c{c}\~oes, $\mv$ deve respeitar as mesmas restri\c{c}\~oes, ou seja:
\begin{equation} \label{eq:ControlLawVRestriction}
\mA \mv + \mb = \mzr
\end{equation}

Aplicando a lei de controle \eqref{eq:ControlLawV} e a restri\c{c}\~ao \eqref{eq:ControlLawVRestriction} em \eqref{eq:MechanicalSystemMatrix}, temos: \\

$ \begin{bmatrix}
\mC^\msT \mM \\
\mA
\end{bmatrix}
\ddot{\mq}
=
\begin{bmatrix}
\mC^\msT ( \mM \mv + \mw + \mz ) - \mC^\msT(\mw + \mz) \\
\mA \mv
\end{bmatrix}
=
\begin{bmatrix}
\mC^\msT  \mM \mv \\
\mA \mv
\end{bmatrix}
=
\begin{bmatrix}
\mC^\msT \mM \\
\mA
\end{bmatrix}
\mv $

\begin{equation} \label{eq:ClosedLoopV}
\therefore \ddot{\mq} = \mv
\end{equation}

Seja $\mv'$ dado pela lei de controle por modos deslizantes:
\begin{equation} \label{eq:SMControlowLasV1'}
\mv' = \ddot{\mq}_{n, d} + \lambda \dot{\me} + k \sign (\dot{\me} + \lambda \me)
\end{equation}
Sendo $ \me = \mq_{n,d} - \mq $ o erro de controle e $\mq_{n,d}$ o sinal de refer\^encia. Se n\~ao houvesse restri\c{c}\~oes, poderiamos fazer $ \mv = \mv' $ :
$$ \ddot{\mq} = \mv \Rightarrow  \ddot{\me} + \lambda \dot{\me} + k \sign (\dot{\me} + \lambda \me) = \mzr \Leftrightarrow \dot{\ms} = - k \sign(\ms)$$
Isso garantiria que $\me \rightarrow 0$ quando $t \rightarrow \infty$ para quaisquer condi\c{c}\~oes iniciais, como visto na se\c{c}\~ao anterior. \\

Como temos restri\c{c}\~oes em $\mv$, procuramos $\mv$ mais pr\'oximo poss\'ivel de $\mv'$ atraves da solu\c{c}\~ao do seguinte problema de otimiza\c{c}\~ao:
\begin{equation} \label{eq:Optimization}
\begin{aligned}
& \underset{\mv}{\text{Min}}
& & (\mv - \mv')^\msT \mM (\mv - \mv') \\
& \text{tal que}
& & \mA \mv + \mb = \mzr
\end{aligned}
\end{equation}

Como $\mM$ \'e n\~ao-negativa definida, temos que $(\mv - \mv')^\msT \mM (\mv - \mv') \geq 0 $ para qualquer valor de $\mv$.

Aplicando a t\'enica dos multiplicadores de Lagrange, pode-se dizer que o seguinte problema \'e equivalente:
\begin{equation}
\begin{aligned}
& \underset{\mv, \mlambda}{\text{Min}}
& & L = (\mv - \mv')^\msT \mM (\mv - \mv') + (\mA \mv + \mb)^\msT \mlambda \\
\end{aligned}
\end{equation}


Para solucionar o problema, imp\~oe-se a estacionariedade da fun\c{c}\~ao lagrangeana:

$$ \dl L = 0 \Rightarrow \dl \mv^\msT \mM (\mv - \mv') + (\mv - \mv')^\msT \mM \dl \mv + (\mA \dl \mv)^\msT \mlambda + (\mA \mv + \mb)^\msT \dl \mlambda = 0 $$
$$ \Rightarrow \dl \mv^\msT \Big( (\mM + \mM^\msT)(\mv - \mv') + \mA^\msT \mlambda \Big) + \dl \mlambda^\msT (\mA \mv + \mb) = 0 $$

Como $\mM$ \'e sim\'etrica e $\dl \mv$ e $\dl \mlambda$ s\~ao arbitr\'arios, temos:
\begin{equation} \label{eq:OptimizationSol}
\begin{cases}
2 \mM (\mv - \mv') + \mA^\msT \mlambda = \mzr \\
\mA \mv + \mb = \mzr
\end{cases}
\end{equation}

Como $\mC$ \'e o complemento ortogonal de $\mA$, multiplicando a primeira equa\c{c}\~ao de \eqref{eq:OptimizationSol} por $\mC^\msT$, temos: \\

$ 2 \mC^\msT \mM (\mv - \mv') + \mC^\msT \mA^\msT \mlambda = \mzr $
$ \Rightarrow  \mC^\msT \mM (\mv - \mv')  = \mzr $
\begin{equation} \label{eq:OptimizationSol2}
\therefore \mC^\msT \mM \mv  = \mC^\msT \mM  \mv'
\end{equation}

Sendo assim, temos que a lei de controle que torna o sistema em malha fechado o mais pr\'oximo poss\'ivel de $\ddot{\mq} = \mv'$, segundo o crit\'erio de otimiza\c{c}\~ao adotado, \'e:
\begin{equation} \label{eq:ControlLawFinal}
\mu = \mC^\msT ( \mM \mv' + \mw + \mz )
\end{equation}