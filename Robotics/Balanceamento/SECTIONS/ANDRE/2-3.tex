\subsection{Introduction to sliding modes control}\label{S02-3}

In this subsection, a brief introduction to the sliding modes control will be done. The theme will be explored only to perform second order systems control, without parametric uncertainties, to not escape the scope of the chapter. \\

Consider a dynamical system given by the following differential equation:
\begin{equation} \label{eq:SimpleODE}
\ddot{x} = u
\end{equation}

A curve in the error phase plan, called sliding surface, can be defined:
\begin{equation} \label{eq:SlidingSurface}
s(e, \dot{e}) = - (\dot{e} + \lambda e) = 0, \, \lambda > 0
\end{equation}

Being $e = x^\sdia - x$ the error signal and $x^\diamond$ reference signal. Note that if the system is on the sliding surface, we have:
\begin{equation} \label{eq:SlidingError}
\dot{e} + \lambda e = 0 \Rightarrow e(t) = c \, \mathsf{e}^{- \lambda t}
\end{equation}

Thus, the error drops exponentially to zero, with time constant $1/\lambda$.

To find a control law that brings the system to the sliding surface, we start from the definition of $s$: \\

$ s = -(\dot{e} + \lambda e) $ \\

Differentiating with respect to time:

\begin{equation} \label{eq:dotS}
\dot{s} =  -(\ddot{e} + \lambda \dot{e}) = \ddot{x} - \ddot{x}^\sdia - \lambda \dot{e} 
\end{equation}

Substituting \eqref{eq:SimpleODE} into \eqref{eq:dotS}:
\begin{equation} \label{dotS2}
\dot{s} = u - \ddot{x}^\sdia - \lambda \dot{e}
\end{equation}

Using the following control law:
\begin{equation} \label{SMControlLaw1D}
u = \ddot{x}^\sdia + \lambda \dot{e} - k \sign (s), \, k>0
\end{equation}

He have:
\begin{equation} \label{CloserLoop1D}
\dot{s} = -k \sign(s) 
\end{equation}

Suppose that the system starts at $s(0) = s_0 >0$. Solving the ODE for $s>0$:

$$ \dot{s} = -k \Rightarrow s = -k t + c $$
$$ s(0) = s_0 \Rightarrow c = s_0 $$
$$ \therefore s = s_0 - k t, \, s>0 $$

According to the solution found, when $t \rightarrow t_s = \frac{|s_0|}{k}$, $s \rightarrow 0 $. Solving the ODE for $s(t_s) = 0$:

$$ \dot{s} = 0 \Rightarrow s =  c $$
$$ s(t_s) = 0 \Rightarrow c = 0 $$

Therefore, the solution of the ODE for $s(0) = s_0 > 0$ is given by:
\begin{equation} \label{eq:SM-ODE-Sol1}
s(t) =
\begin{cases}
s_0 - k t, \, t < t_s \\
0, \,\,\,\,\,\,\,\,\,\,\,\,\,\,\,\, t \geq t_s \\
\end{cases}
\end{equation}

An analogous result is found solving the ODE for $s(0) = s_0 < 0$:

\begin{equation} \label{eq:SM-ODE-Sol2}
s(t) =
\begin{cases}
s_0 + k t, \, t < t_s \\
0, \,\,\,\,\,\,\,\,\,\,\,\,\,\,\,\, t \geq t_s \\
\end{cases}
\end{equation}

Thus, it can be concluded that the ODE \eqref{CloserLoop1D} converges to $s=0$, regardless of the initial condition.
Therefore, we have that the control law \eqref{SMControlLaw1D} makes the system represented by \eqref{eq:SimpleODE} follow the reference signal, because the error signal converges to zero.

