\subsection{Inverse Dynamics and Control}\label{S02-3}

\noindent {\bf Din\^amica Inversa}\\

O problema da din\^amica inversa constitui-se basicamente em calcular os esfor\c{c}os ativos aplicados pelos atuadores necess\'arios para um mecanismo realizar uma dada trajet\'oria.

\'E um procedimento importante para avaliar a efic\'acia do balanceamento realizado, pois torna poss\'ivel comparar os esfor\c{c}os dos atuadores em diferentes trajet\'orias para o mecanismo balanceado e desbalanceado, mostrando em quais casos o balanceamento \'e vantajoso. \\

Primeiramente, devemos fazer algumas defini\c{c}\~oes:

\begin{itemize}
\item $\mq_n\ssh$: vetor de coordenadas generalizadas independentes. \'E um sub-vetor de $\mq_n$, contendo $\nu\ssh(\ssS_n)$ elementos de $\mq_n$ independentes;
\item $\mu_n$: vetor de $\nu\ssh(\ssS_n)$ entradas de controle independentes. Cada elemento $u_{n,k}$ de $\mu_n$ atua na dire\c{c}\~ao da velocidade generalizada independente $\dot{q}_{n,k}\ssh$.
\item $\mh_n$: vetor dos v\'inculos cinem\'aticos de posi\c{c}\~ao. Como $h_{n, r}$ foi definido anteriormente, podemos definir:
\begin{align}
&	\mh_n = \nvct{h_{n,r}}
	\qquad \text{for} \qquad
	r \in \{1, 2, \ldots, \nu_q(\ssS_n) - \nu\ssh(\ssS_n)\}
	\label{eq:Teste2} %\\
\end{align}
\item $\mathbb{\psi}_n$: vetor dos v\'inculos cinem\'aticos de velocidades. Como $\psi_{n, r}$ foi definido anteriormente, podemos definir:

\begin{align}
&	\mathbb{\psi}_n = \nvct{\psi_{n,r}}
	\qquad \text{for} \qquad
	r \in \{1, 2, \ldots, \nu_p(\ssS_n) - \nu\ssh(\ssS_n)\}
	\label{eq:Teste3} %\\
\end{align}

\end{itemize}

Feitas as defini\c{c}\~oes acima, pode-se dizer o objetivo da din\^amica inversa \'e: para cada instante de tempo $t$, conhencendo-se $\mq_n\ssh$, $\dot{\mq}_n\ssh$ e $\ddot{\mq}_n\ssh$, determinar $\mu_n$.

Como foi visto na se\c{c}\~ao 2-1, a equa\c{c}\~ao \eqref{eq:02-112B} nos mostra que: \\

$ \mC_{n}^\msT \left( \mf_{n} + \mg_{n} - \mM_{n} \, \dot \mp_{n} \right) = \mzr $ \\

Usualmente \'e poss\'ivel decompor os esfor\c{c}os ativos $\mf_{n}$ em duas parcelas: $\mf'_n$, os esfor\c{c}os ativos provenientes dos atuadores, e $-\mz_n$, os esfor\c{c}os ativos provevientes da for\c{c}a peso.
\begin{equation} \label{eq:ActiveForces}
\mf_n (\mq_n) = \mf'_n - \mz_n(\mq_n)
\end{equation}

Supondo que a decomposi\c{c}\~ao acima seja poss\'ivel e n\~ao haja atua\c{c}\~ao redundante, temos que $\mf'_n$ cont\'em apenas $ \nu\ssh(\ssS_n)$ elementos n\~ao nulos. Supondo tamb\'em $\mC_n^\msT \mf'_n$ contenha apenas os elemementos n\~ao nulos de $\mf'_n$, o que \'e muito comum, temos que:
\begin{equation} \label{eq:ControlInputs}
\mu_n = \mC_n^\msT \mf'_n
\end{equation}

Al\'em disso, definimos:
\begin{equation} \label{eq:GiroscopycForces}
\mw_n (\mq_n, \mp_n) = -\mg_n (\mq_n, \mp_n)
\end{equation}

Substituindo \eqref{eq:ActiveForces}, \eqref{eq:ControlInputs}, e \eqref{eq:GiroscopycForces} em \eqref{eq:02-112B}, \'e poss\'ivel calcular dos esfor\c{c}os dos atuadores pela seguinte express\~ao:
\begin{equation} \label{eq:InverseDynamics}
\mu_n = \mC_n^\msT(\mq_n) \Big( \mM_n(\mq_n) \dot{\mp}_n + \mw_n(\mq_n, \mp_n) + \mz_n(\mq_n) \Big)
\end{equation}

Por\'em, nesta express\~ao, $\mu_n$ depende de $\mq_n$, $\mp_n$ e $\dot{\mp}_n$. Sendo assim, para o problema da din\^amica inversa ser resolvido, \'e necess\'ario determinar $\mq_n$, $\mp_n$ e $\dot{\mp}_n$ dados $\mq_n\ssh$, $\dot{\mq}_n\ssh$ e $\ddot{\mq}_n\ssh$. \\

Para determinar $\mq_n$ dado $\mq_n\ssh$, basta resolver as equa\c{c}\~oes vinculares de posi\c{c}\~ao: 
\begin{equation} \label{eq:PositionConstraints}
\mh_n(t, \mq_n) = \mzr
\end{equation}

%Como $\mh_n$ \'e um vetor de tamanho $\nu_q - \nu\ssh$ e $\nu\ssh$ elementos de $\mq_n$ s\~ao conhecidos ($\mq_n\ssh$), temos $\nu_q - \nu\ssh$ equa\c{c}\~oes linearmente independentes e $\nu_q - \nu\ssh$ incognitas, o que garante que o sistema tenha um n\'umero finito de solu\c{c}\~oes.

Como os v\'inculos de posi\c{c}\~ao normalmente s\~ao equa\c{c}\~oes n\~ao-lineares, para configura\c{c}\~oes poss\'iveis \'e comum encontrar duas ou mais solu\c{c}\~oes para dados $t$ e $\mq_n\ssh$. Para saber qual \'e a solu\c{c}\~ao representativa para o problema em quest\~ao, \'e necess\'ario verificar qual delas \'e condizente com as configura\c{c}\~oes de montagem do mecanismo. No caso de solu\c{c}\~oes anal\'iticas, basta identificar qual solu\c{c}\~ao representa a configura\c{c}\~ao de montagem e utiliza-la para todos os pontos da trajet\'oria. No caso de solu\c{c}\~oes num\'ericas, \'e necess\'ario uma estimativa da configura\c{c}\~ao desejada para determinar $\mq_n$ com uma dada precis\~ao. Normalmente estima-se uma solu\c{c}\~ao para $t=0$, e para os demais instantes de tempo utiliza-se como estimativa a configura\c{c}\~ao do \'ultimo instante de tempo calculado. \\

Com $\mq_n$ j\'a definido e dado $\dot{\mq}_n\ssh = \mp_n\ssh$, $\mp_n$ pode ser determinado atrav\'es dos v\'inculos de velocidade:
\begin{equation} \label{eq:VelocityConstraints}
\mathbb{\psi}_n (t, \mq, \mp) = \mzr
\end{equation}

%Como $\mathbb{\psi}_n$ \'e um vetor de tamanho $\nu_p - \nu\ssh$ e $\nu\ssh$ elementos de $\mp_n$ s\~ao conhecidos ($\mp_n\ssh$), temos $\nu_p - \nu\ssh$ equa\c{c}\~oes linearmente independentes e $\nu_p - \nu\ssh$ incognitas, o que garante que o sistema tenha um n\'umero finito de solu\c{c}\~oes.

Os v\'inculos de velocidades normalmente s\~ao equa\c{c}\~oes lineares em $\mp_n$, o que torna simples determinar $\mp_n$, dados $t$, $\mq_n$ e $\mp_n\ssh$. \\

Com $\mq_n$ e $\mp_n$ j\'a definidos e dado $\ddot{\mq}_n\ssh = \dot{\mp}_n\ssh$, $\dot{\mp}_n$ pode ser determinado atrav\'es dos v\'inculos de acelera\c{c}\~ao:
\begin{equation} \label{eq:AccelerationConstraints}
\mc_n (t, \mq, \mp, \dot{\mp}) = \mzr
\end{equation}

%Como $\mc_n$ \'e um vetor de tamanho $\nu_p - \nu\ssh$ e $\nu\ssh$ elementos de $\dot{\mp}_n$ s\~ao conhecidos ($\dot{\mp}_n\ssh$), temos $\nu_p - \nu\ssh$ equa\c{c}\~oes linearmente independentes e $\nu_p - \nu\ssh$ incognitas, o que garante que o sistema tenha um n\'umero finito de solu\c{c}\~oes.

Analogamente aos v\'inculos de velocidades, os v\'inculos de acelera\c{c}\~oes normalmente s\~ao equa\c{c}\~oes lineares em $\dot{\mp}_n$, o que torna simples determinar $\dot{\mp}_n$, dados $t$, $\mq_n$, $\mp_n$ e $\dot{\mp}\ssh$. \\

Finalmente, com $\mq_n$, $\mp_n$ e $\dot{\mp}_n$ determinados, a express\~ao \eqref{eq:InverseDynamics} pode ser utilizada para o c\'alculo dos esfor\c{c}os nos atuadores. \\


 
\noindent {\bf Controle por modos deslizantes}\\

Nesta se\c{c}\~ao ser\'a feita uma breve introdu\c{c}\~ao ao controle por modos deslizantes. O tema ser\'a explorado apenas para o controle de sistemas de segunda ordem, sem incertezas param\'etricas, para n\~ao fugir do escopo do cap\'itulo. \\

Seja um sistema din\^amico dado pela seguinte equa\c{c}\~ao diferencial:
\begin{equation} \label{eq:SimpleODE}
\ddot{x} = u
\end{equation}

Definimos a seguinte superf\'icie, chamada de superf\'icie de escorregamento:
\begin{equation} \label{eq:SlidingSurface}
s(e, \dot{e}) = - (\dot{e} + \lambda e) = 0, \, \lambda > 0
\end{equation}

Sendo $e = x_d - x$ o erro de controle e $x_d$ o sinal de refer\^encia. Repare que se o sistema estiver na superf\'icie de escorregamento, temos:
\begin{equation} \label{eq:SlidingError}
\dot{e} + \lambda e = 0 \Rightarrow e(t) = C e^{- \lambda t}
\end{equation}

Sendo assim, o erro cai exponencialmente para zero, com constante de tempo $1/\lambda$.

Para encontrar a lei de controle que leva o sistema \`a superf\'icie de escorregamento, parte-se da defini\c{c}\~ao de $s$:

$ s = -(\dot{e} + \lambda e) $ \\

Derivando no tempo:
\begin{equation} \label{eq:dotS}
\dot{s} =  -(\ddot{e} + \lambda \dot{e}) = \ddot{x} - \ddot{x}_d - \lambda \dot{e} 
\end{equation}

Substituindo \eqref{eq:SimpleODE} em \eqref{eq:dotS}:
\begin{equation} \label{dotS2}
\dot{s} = u - \ddot{x}_d - \lambda \dot{e}
\end{equation}

Utizando a seguinte lei de controle:
\begin{equation} \label{SMControlLaw1D}
u = \ddot{x}_d + \lambda \dot{e} - k \sign (s), \, k>0
\end{equation}

Temos:
\begin{equation} \label{CloserLoop1D}
\dot{s} = -k \sign(s) 
\end{equation}

Supondo que o sistema come\c{c}a em $s(0) = s_0 >0$. Resolvendo a EDO para $s>0$:

$$ \dot{s} = -k \Rightarrow s = -k t + C $$
$$ s(0) = s_0 \Rightarrow C = s_0 $$
$$ \therefore s = s_0 - k t, \, s>0 $$

Em $t = t_s = \frac{|s_0|}{k}$, $s$ chega em zero. Resolvendo a EDO para $s(t_s) = 0$:

$$ \dot{s} = 0 \Rightarrow s =  C $$
$$ s(t_s) = 0 \Rightarrow C = 0 $$

Portanto, para a solu\c{c}\~ao da EDO para $s(0) = s_0 > 0$ é
\begin{equation} \label{eq:SM-ODE-Sol1}
s(t) =
\begin{cases}
s_0 - k t, \, t < t_s \\
0, \,\,\,\,\,\,\,\,\,\,\,\,\,\,\,\, t \geq t_s \\
\end{cases}
\end{equation}

Resolvendo para $s(0) = s_0 < 0$, temos um resultado an\'alogo:

\begin{equation} \label{eq:SM-ODE-Sol2}
s(t) =
\begin{cases}
s_0 + k t, \, t < t_s \\
0, \,\,\,\,\,\,\,\,\,\,\,\,\,\,\,\, t \geq t_s \\
\end{cases}
\end{equation}

Assim, pode-se concluir que a EDO \eqref{CloserLoop1D} converge para $s=0$, independente da condi\c{c}\~ao inicial. Portanto, temos que a lei de controle \eqref{SMControlLaw1D} faz com que o sistema representado por \eqref{eq:SimpleODE} siga o sinal de refer\^encia, pois o erro de controle converge para zero. \\

\noindent {\bf Controle por modos deslizantes extendido}\\

Nesta se\c{c}\~ao, para tornar o texto mais leg\'ivel, iremos omitir o \'indice $n$ referente ao sub-sistema mec\^anico. \\

Como foi visto na se\c{c}\~ao de modelagem, \'e muito conveniente utilizar coordenadas redundantes para realizar a modelagem de mecanismos paralelos. Sendo assim, propomos nesta se\c{c}\~ao uma lei de controle para sistemas descritos por coordenadas redundantes. Por\'em, para o ponto de vista do controle, \'e conveniente utilizar $\dot{\mq}$ no lugar de $\mp$, pois muitas vezes as velocidades generalizadas $\mp$ s\~ao n\~ao integr\'aveis, o que impossibilita a realimenta\c{c}\~ao de posi\c{c}\~ao na dire\c{c}\~ao destas coordenadas, ent\~ao assim faremos. \\

Baseado nas equa\c{c}\~oes \eqref{eq:InverseDynamics} e \eqref{eq:02-201A}, seja o modelo de um sistema mec\^anico multi-corpos descrito pelas seguintes equa\c{c}\~oes:
\begin{equation} \label{eq:MechanicalSystem}
\begin{cases}
\mC^\msT (\mq) \Big( \mM (\mq) \ddot{\mq} + \mw (\mq, \dot{\mq}) + \mz (\mq) \Big) = \mu \\
\mA (\mq) \ddot{\mq} + \mb (\mq, \dot{\mq}) = \mzr
\end{cases}
\end{equation}

De maneira matricial compacta:
\begin{equation} \label{eq:MechanicalSystemMatrix}
\begin{bmatrix}
\mC^\msT \mM \\
\mA
\end{bmatrix}
\ddot{\mq}
=
\begin{bmatrix}
\mu - \mC^\msT(\mw + \mz) \\
-\mb
\end{bmatrix}
\end{equation}

Gostaria que $ \ddot{\mq} = \mv $, sendo $\mv$ uma entrada de controle. Para que isso aconte\c{c}a, utilizamos a seguinte lei de controle:
\begin{equation} \label{eq:ControlLawV}
\mu = \mC^\msT ( \mM \mv + \mw + \mz )
\end{equation}

Como queremos que $ \ddot{\mq} = \mv $ e $\ddot{\mq}$ tem restri\c{c}\~oes, $\mv$ deve respeitar as mesmas restri\c{c}\~oes, ou seja:
\begin{equation} \label{eq:ControlLawVRestriction}
\mA \mv + \mb = \mzr
\end{equation}

Aplicando a lei de controle \eqref{eq:ControlLawV} e a restri\c{c}\~ao \eqref{eq:ControlLawVRestriction} em \eqref{eq:MechanicalSystemMatrix}, temos: \\

$ \begin{bmatrix}
\mC^\msT \mM \\
\mA
\end{bmatrix}
\ddot{\mq}
=
\begin{bmatrix}
\mC^\msT ( \mM \mv + \mw + \mz ) - \mC^\msT(\mw + \mz) \\
\mA \mv
\end{bmatrix}
=
\begin{bmatrix}
\mC^\msT  \mM \mv \\
\mA \mv
\end{bmatrix}
=
\begin{bmatrix}
\mC^\msT \mM \\
\mA
\end{bmatrix}
\mv $

\begin{equation} \label{eq:ClosedLoopV}
\therefore \ddot{\mq} = \mv
\end{equation}

Seja $\mv'$ dado pela lei de controle por modos deslizantes:
\begin{equation} \label{eq:SMControlowLasV1'}
\mv' = \ddot{\mq}_{n, d} + \lambda \dot{\me} + k \sign (\dot{\me} + \lambda \me)
\end{equation}
Sendo $ \me = \mq_{n,d} - \mq $ o erro de controle e $\mq_{n,d}$ o sinal de refer\^encia. Se n\~ao houvesse restri\c{c}\~oes, poderiamos fazer $ \mv = \mv' $ :
$$ \ddot{\mq} = \mv \Rightarrow  \ddot{\me} + \lambda \dot{\me} + k \sign (\dot{\me} + \lambda \me) = \mzr \Leftrightarrow \dot{\ms} = - k \sign(\ms)$$
Isso garantiria que $\me \rightarrow 0$ quando $t \rightarrow \infty$ para quaisquer condi\c{c}\~oes iniciais, como visto na se\c{c}\~ao anterior. \\

Como temos restri\c{c}\~oes em $\mv$, procuramos $\mv$ mais pr\'oximo poss\'ivel de $\mv'$ atraves da solu\c{c}\~ao do seguinte problema de otimiza\c{c}\~ao:
\begin{equation} \label{eq:Optimization}
\begin{aligned}
& \underset{\mv}{\text{Min}}
& & (\mv - \mv')^\msT \mM (\mv - \mv') \\
& \text{tal que}
& & \mA \mv + \mb = \mzr
\end{aligned}
\end{equation}

Como $\mM$ \'e n\~ao-negativa definida, temos que $(\mv - \mv')^\msT \mM (\mv - \mv') \geq 0 $ para qualquer valor de $\mv$.

Aplicando a t\'enica dos multiplicadores de Lagrange, pode-se dizer que o seguinte problema \'e equivalente:
\begin{equation}
\begin{aligned}
& \underset{\mv, \mlambda}{\text{Min}}
& & L = (\mv - \mv')^\msT \mM (\mv - \mv') + (\mA \mv + \mb)^\msT \mlambda \\
\end{aligned}
\end{equation}


Para solucionar o problema, imp\~oe-se a estacionariedade da fun\c{c}\~ao lagrangeana:

$$ \dl L = 0 \Rightarrow \dl \mv^\msT \mM (\mv - \mv') + (\mv - \mv')^\msT \mM \dl \mv + (\mA \dl \mv)^\msT \mlambda + (\mA \mv + \mb)^\msT \dl \mlambda = 0 $$
$$ \Rightarrow \dl \mv^\msT \Big( (\mM + \mM^\msT)(\mv - \mv') + \mA^\msT \mlambda \Big) + \dl \mlambda^\msT (\mA \mv + \mb) = 0 $$

Como $\mM$ \'e sim\'etrica e $\dl \mv$ e $\dl \mlambda$ s\~ao arbitr\'arios, temos:
\begin{equation} \label{eq:OptimizationSol}
\begin{cases}
2 \mM (\mv - \mv') + \mA^\msT \mlambda = \mzr \\
\mA \mv + \mb = \mzr
\end{cases}
\end{equation}

Como $\mC$ \'e o complemento ortogonal de $\mA$, multiplicando a primeira equa\c{c}\~ao de \eqref{eq:OptimizationSol} por $\mC^\msT$, temos: \\

$ 2 \mC^\msT \mM (\mv - \mv') + \mC^\msT \mA^\msT \mlambda = \mzr $
$ \Rightarrow  \mC^\msT \mM (\mv - \mv')  = \mzr $
\begin{equation} \label{eq:OptimizationSol2}
\therefore \mC^\msT \mM \mv  = \mC^\msT \mM  \mv'
\end{equation}

Sendo assim, temos que a lei de controle que torna o sistema em malha fechado o mais pr\'oximo poss\'ivel de $\ddot{\mq} = \mv'$, segundo o crit\'erio de otimiza\c{c}\~ao adotado, \'e:
\begin{equation} \label{eq:ControlLawFinal}
\mu = \mC^\msT ( \mM \mv' + \mw + \mz )
\end{equation}

\newpage

\noindent {\bf Prova que o m\'etodo converge, n\~ao vai para o livro}\\

$$
\begin{bmatrix}
\mC^T \mM \\
\mA
\end{bmatrix}
\ddot{\mq}
=
\begin{bmatrix}
\mathbb{\zeta} \mu \\
-\mb
\end{bmatrix}
$$



$$ \mu = \mathbb{\zeta}^{-1} \mC^T \mM \mu' $$
$$ \mu' = \ddot{\mq}_d + \lambda \dot{\me} - k  \sign (\ms) $$

$$ \ms = - \dot{\me} - \lambda \me $$
$$ \dot{\ms} = - \ddot{\me} - \lambda \dot{\me} = \ddot{\mq} - \ddot{\mq}_d  - \lambda \dot{\me} $$
$$ \dot{\ms} =  \begin{bmatrix}
\mC^T \mM \\
\mA
\end{bmatrix}^{-1}
\begin{bmatrix}
\mathbb{\zeta} \mu \\
-\mb
\end{bmatrix}
 - \ddot{\mq}_d  - \lambda \dot{\me} $$
 
Aplicando a lei de controle:

$$ \dot{\ms} =  \begin{bmatrix}
\mC^T \mM \\
\mA
\end{bmatrix}^{-1}
\begin{bmatrix}
 \mC^T \mM (\ddot{\mq}_d + \lambda \dot{\me} - k  \sign (\ms)) \\
-\mb
\end{bmatrix}
 - \ddot{\mq}_d  - \lambda \dot{\me} $$
 
 $$ \dot{\ms} =  \begin{bmatrix}
\mC^T \mM \\
\mA
\end{bmatrix}^{-1}
\begin{bmatrix}
 \mC^T \mM (\ddot{\mq}_d + \lambda \dot{\me} - k  \sign (\ms)) -  \mC^T \mM(\ddot{\mq}_d  + \lambda \dot{\me}) \\
-\mb - \mA(\ddot{\mq}_d  + \lambda \dot{\me})
\end{bmatrix} $$

 $$ \dot{\ms} =  \begin{bmatrix}
\mC^T \mM \\
\mA
\end{bmatrix}^{-1}
\begin{bmatrix}
- \mC^T \mM  k  \sign (\ms) \\
-\mb - \mA(\ddot{\mq}_d  + \lambda \dot{\me})
\end{bmatrix} $$

Definindo:
$$\begin{bmatrix}
\mC^T \mM \\
\mA
\end{bmatrix}^{-1}
=
\begin{bmatrix}
(\mC^T \mM)^\dagger & \mA^\dagger
\end{bmatrix} $$

Temos:

$$\dot{\ms} = 
- (\mC^T \mM)^\dagger \mC^T \mM  k  \sign (\ms) - \mA^\dagger\mb - \mA^\dagger \mA(\ddot{\mq}_d  + \lambda \dot{\me}) $$

Sendo assim, se a seguinte inequação for respeitada para pelo menos $\nu\ssh$ componentes de $\dot{\ms}$, o erro vai a zero:

$$\dot{\ms} = 
- (\mC^T \mM)^\dagger \mC^T \mM  k  \sign (\ms) - \mA^\dagger\mb - \mA^\dagger \mA(\ddot{\mq}_d  + \lambda \dot{\me}) \leq \mzr $$