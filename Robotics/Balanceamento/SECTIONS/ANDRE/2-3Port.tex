\subsection{Controle por modos deslizantes}\label{S02-3}

Nesta se\c{c}\~ao ser\'a feita uma breve introdu\c{c}\~ao ao controle por modos deslizantes. O tema ser\'a explorado apenas para o controle de sistemas de segunda ordem, sem incertezas param\'etricas, para n\~ao fugir do escopo do cap\'itulo. \\

Seja um sistema din\^amico dado pela seguinte equa\c{c}\~ao diferencial:
\begin{equation} \label{eq:SimpleODE}
\ddot{x} = u
\end{equation}

Definimos a seguinte superf\'icie, chamada de superf\'icie de escorregamento:
\begin{equation} \label{eq:SlidingSurface}
s(e, \dot{e}) = - (\dot{e} + \lambda e) = 0, \, \lambda > 0
\end{equation}

Sendo $e = x_d - x$ o erro de controle e $x_d$ o sinal de refer\^encia. Repare que se o sistema estiver na superf\'icie de escorregamento, temos:
\begin{equation} \label{eq:SlidingError}
\dot{e} + \lambda e = 0 \Rightarrow e(t) = C e^{- \lambda t}
\end{equation}

Sendo assim, o erro cai exponencialmente para zero, com constante de tempo $1/\lambda$.

Para encontrar a lei de controle que leva o sistema \`a superf\'icie de escorregamento, parte-se da defini\c{c}\~ao de $s$:

$ s = -(\dot{e} + \lambda e) $ \\

Derivando no tempo:
\begin{equation} \label{eq:dotS}
\dot{s} =  -(\ddot{e} + \lambda \dot{e}) = \ddot{x} - \ddot{x}_d - \lambda \dot{e} 
\end{equation}

Substituindo \eqref{eq:SimpleODE} em \eqref{eq:dotS}:
\begin{equation} \label{dotS2}
\dot{s} = u - \ddot{x}_d - \lambda \dot{e}
\end{equation}

Utizando a seguinte lei de controle:
\begin{equation} \label{SMControlLaw1D}
u = \ddot{x}_d + \lambda \dot{e} - k \sign (s), \, k>0
\end{equation}

Temos:
\begin{equation} \label{CloserLoop1D}
\dot{s} = -k \sign(s) 
\end{equation}

Supondo que o sistema come\c{c}a em $s(0) = s_0 >0$. Resolvendo a EDO para $s>0$:

$$ \dot{s} = -k \Rightarrow s = -k t + c $$
$$ s(0) = s_0 \Rightarrow c = s_0 $$
$$ \therefore s = s_0 - k t, \, s>0 $$

Em $t = t_s = \frac{|s_0|}{k}$, $s$ chega em zero. Resolvendo a EDO para $s(t_s) = 0$:

$$ \dot{s} = 0 \Rightarrow s =  c $$
$$ s(t_s) = 0 \Rightarrow c = 0 $$

Portanto, para a solu\c{c}\~ao da EDO para $s(0) = s_0 > 0$ é
\begin{equation} \label{eq:SM-ODE-Sol1}
s(t) =
\begin{cases}
s_0 - k t, \, t < t_s \\
0, \,\,\,\,\,\,\,\,\,\,\,\,\,\,\,\, t \geq t_s \\
\end{cases}
\end{equation}

Resolvendo para $s(0) = s_0 < 0$, temos um resultado an\'alogo:

\begin{equation} \label{eq:SM-ODE-Sol2}
s(t) =
\begin{cases}
s_0 + k t, \, t < t_s \\
0, \,\,\,\,\,\,\,\,\,\,\,\,\,\,\,\, t \geq t_s \\
\end{cases}
\end{equation}

Assim, pode-se concluir que a EDO \eqref{CloserLoop1D} converge para $s=0$, independente da condi\c{c}\~ao inicial. Portanto, temos que a lei de controle \eqref{SMControlLaw1D} faz com que o sistema representado por \eqref{eq:SimpleODE} siga o sinal de refer\^encia, pois o erro de controle converge para zero.

