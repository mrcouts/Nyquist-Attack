\subsection{Controle por modos deslizantes extendido}\label{S02-4}

As seen in subsections 2.1 and 2.2, it's very convenient to use redundant coordinates to perform parallel mechanism dynamic modeling. Thus, we propose in this subsection a control law for systems described by redundant coordinates.

Let $\ssM$ be a multibody mechanical system whose mathematical model is given 
by equations~(\ref{eq:02-207A},~\ref{eq:02-207B}).
Suppose that each $\mf_{n}$ is an affine function of the control inputs $u_{n,k}$
in which the coefficients of the $u_{n,k}$ may depend on the instantaneous
configuration of the system.
Suppose additionally that that all the $\mA_{n}$ are independent
of the quasi-velocities $p_{n,j}$ and all the $\mM_{n}$, $\mg_{n}$, $\mA_{n}$ and $\mb_{n}$ 
are independent of the $u_{n,k}$.
Under these conditions, matrices $\mC_{n}$ will not depend on 
any quasi-velocity, $\md$ can be expressed as an affine function of the control inputs 
and $\mc$ is independent of them.
Considering that the number of control inputs in a mechanical system is
exactly equal to the number of degrees of freedom of $\ssM$, 
it may be possible to solve equations~(\ref{eq:02-207B}) in order to
express each $u_{n,k}$ as an explicit function of state variables and
physical parameters of $\ssM$.
For the sake of brevity, omitting the indexes $n$ of the subsystems of $\ssM$,
it can be stated that, in such cases, for $k \in \{ 1,\ldots,\nu\ssh(\ssM) \}$:
\begin{align}
&	u_{k} = \sum_{r} C_{kr} (t,\mq) \Big(  \sum_{j} M_{rj}(t,\mq) \, \dot p_j 
		+ w_{r}(t, \mq, \mp)
		+ z_{r}(t, \mq) \Big)
	\label{eq:03-101Especial} 
\end{align}
That is equivalent to:
\begin{align}
&	u_{k} = \sum_{r} M'_{kr}(t,\mq) \, \dot p_r 
		+ w'_{k}(t, \mq, \mp)
		+ z'_{k}(t, \mq)
	\label{eq:03-101A} 
\end{align}
It can be shown that each $w'_{k}(t, \mq, \mp)$ can be expressed as a sum of 
a bilinear and a linear function on the quasi-velocities, i.e., there are functions
$D'_{krs}(t,\mq)$ and $B'_{kr}(t,\mq)$ such that:
\begin{align}	
&	w'_{k}(t, \mq, \mp) = \sum_{r} \sum_{s} D'_{krs}(t,\mq) \, p_r \, p_s
		+ \sum_{r} B'_{kr}(t,\mq) \, p_r
	\label{eq:03-101B}	
\end{align}
Adaptive balancing types will be defined in the next section based on equation~(\ref{eq:03-101A}).

For keeping the sake of brevity, the indexes $n$ of the subsystems of $\ssM$ will be omitted from this point in this subsection.

From the control perspective, it's convenient to use $\dot{\mq}$ instead of $\mp$, because the quasi-velocities $\mp$ usually are not integrable, what prevents the position feedback in the direction of these coordinates, then we will do so.

Based on equations \eqref{eq:03-101Especial} and \eqref{eq:02-201A}, consider the dynamic model of a multi-body mechanical system described by the following equations:

\begin{equation} \label{eq:MechanicalSystem}
\begin{cases}
\mC^\msT (t, \mq) \Big( \mM (t, \mq) \ddot{\mq} + \mw (t, \mq, \dot{\mq}) + \mz (t, \mq) \Big) = \mu \\
\mA (t, \mq) \ddot{\mq} + \mb (t, \mq, \dot{\mq}) = \mzr
\end{cases}
\end{equation}

Rewriting in a compact matrix way:

\begin{equation} \label{eq:MechanicalSystemMatrix}
\begin{bmatrix}
\mC^\msT \mM \\
\mA
\end{bmatrix}
\ddot{\mq}
=
\begin{bmatrix}
\mu - \mC^\msT(\mw + \mz) \\
-\mb
\end{bmatrix}
\end{equation}

We want to find a control law such that, in closed loop, $ \ddot{\mq} = \mv $, being $\mv$ a control input matrix column. For this to happen, the following control law is used:
\begin{equation} \label{eq:ControlLawV}
\mu = \mC^\msT ( \mM \mv + \mw + \mz )
\end{equation}

As we want $ \ddot{\mq} = \mv $ and $\ddot{\mq}$ has restrictions, $\mv$ must respect the same restrictions, i.e.:
\begin{equation} \label{eq:ControlLawVRestriction}
\mA \mv + \mb = \mzr
\end{equation}

Applying the control law \eqref{eq:ControlLawV} and the rectritions \eqref{eq:ControlLawVRestriction} in \eqref{eq:MechanicalSystemMatrix}, we have: \\

$ \begin{bmatrix}
\mC^\msT \mM \\
\mA
\end{bmatrix}
\ddot{\mq}
=
\begin{bmatrix}
\mC^\msT ( \mM \mv + \mw + \mz ) - \mC^\msT(\mw + \mz) \\
\mA \mv
\end{bmatrix}
=
\begin{bmatrix}
\mC^\msT  \mM \mv \\
\mA \mv
\end{bmatrix}
=
\begin{bmatrix}
\mC^\msT \mM \\
\mA
\end{bmatrix}
\mv $

\begin{equation} \label{eq:ClosedLoopV}
\therefore \ddot{\mq} = \mv
\end{equation}

Let $\mv'$ be given by the sliding modes control law:
\begin{equation} \label{eq:SMControlowLasV1'}
\mv' = \ddot{\mq}_{n}^\sdia + \lambda \dot{\me} + k \sign (\dot{\me} + \lambda \me)
\end{equation}
Being $ \me = \mq_{n}^\sdia - \mq $ the error signal and $\mq_{n}^\sdia$ the reference signal. If there wasn't any retrictions, we could do $ \mv = \mv' $ :
$$ \ddot{\mq} = \mv \Rightarrow  \ddot{\me} + \lambda \dot{\me} + k \sign (\dot{\me} + \lambda \me) = \mzr \Leftrightarrow \dot{\ms} = - k \sign(\ms)$$
This would ensure that $\me \rightarrow 0$ when $t \rightarrow \infty$ for any initial condition, as seen in the last subsection. \\

As we have resctriction on $\mv$, we look for the closest possible $\mv$ of $\mv'$ 
by solving the following optimization problem:
\begin{equation} \label{eq:Optimization}
\begin{aligned}
& \underset{\mv}{\min}
& & (\mv - \mv')^\msT \mM (\mv - \mv') \\
& \text{s.t.}
& & \mA \mv + \mb = \mzr
\end{aligned}
\end{equation}

As $\mM$ is positive-semidefinite, we have $(\mv - \mv')^\msT \mM (\mv - \mv') \geq 0 $ for any $\mv$.

Applying the method of Lagrange multipliers, it can be said that the following problem is equivalent to:
\begin{equation}
\begin{aligned}
& \underset{\mv, \mlambda}{\min}
& & L = (\mv - \mv')^\msT \mM (\mv - \mv') + (\mA \mv + \mb)^\msT \mlambda \\
\end{aligned}
\end{equation}

To solve the problem, the Lagrangian function must be stationary:

$$ \dl L = 0 \Rightarrow \dl \mv^\msT \mM (\mv - \mv') + (\mv - \mv')^\msT \mM \dl \mv + (\mA \dl \mv)^\msT \mlambda + (\mA \mv + \mb)^\msT \dl \mlambda = 0 $$
$$ \Rightarrow \dl \mv^\msT \Big( (\mM + \mM^\msT)(\mv - \mv') + \mA^\msT \mlambda \Big) + \dl \mlambda^\msT (\mA \mv + \mb) = 0 $$

As $\mM$ is symmetric and $\dl \mv$ and $\dl \mlambda$ are arbitrary, we have:
\begin{equation} \label{eq:OptimizationSol}
\begin{cases}
2 \mM (\mv - \mv') + \mA^\msT \mlambda = \mzr \\
\mA \mv + \mb = \mzr
\end{cases}
\end{equation}

As $\mC$ is the orthogonal complement of $\mA$, pre-multiplying the first equation of \eqref{eq:OptimizationSol} by $\mC^\msT$, we have: \\

$ 2 \mC^\msT \mM (\mv - \mv') + \mC^\msT \mA^\msT \mlambda = \mzr $
$ \Rightarrow  \mC^\msT \mM (\mv - \mv')  = \mzr $
\begin{equation} \label{eq:OptimizationSol2}
\therefore \mC^\msT \mM \mv  = \mC^\msT \mM  \mv'
\end{equation}

Thus, we have that the control law that makes the closed loop system as close as possible of $\ddot{\mq} = \mv'$, 
according to the optimization criterion adopted, is:
\begin{equation} \label{eq:ControlLawFinal}
\mu = \mC^\msT ( \mM \mv' + \mw + \mz )
\end{equation}