\section{Conclusions}\label{S05}

%
% methodology
% dynamic modelling: use of redundant generalized coordinates,
%           compensation inertias (independent, modular definition)
%           successive coupling (a posteriore) without the need
%            of rewriting (a priori)
% control: sliding mode why?
% simulation results: why they are good?
% future works
%

This work dealt with the dynamic modelling and control of balanced parallel
mechanisms. The dynamic modelling process constitutes an important issue considering
the structural complexity of parallel mechanisms. 
Therefore, this book chapter described a dynamic formalism capable to deal with
redundant generalized coordinates in association with the successive coupling of additional balancing elements 
to the original system model. 
This represents {\em a posteriore} procedure because the analyst can 
successively include compensation inertias  
during the modelling
without the need of rewriting ({\em a priori} procedure) the dynamic equations.
Then, not only the compensation conditions can be derived 
but also the desired input torques for the motion control of the parallel mechanism.
This work discussed the advantages of the dynamic model,
developed in accordance with the methodology shown here, for the
sliding mode control. Finally, the simulation results have demonstrated
how effective is the presented methodology for the planar 5-bar mechanism with revolute joints. 