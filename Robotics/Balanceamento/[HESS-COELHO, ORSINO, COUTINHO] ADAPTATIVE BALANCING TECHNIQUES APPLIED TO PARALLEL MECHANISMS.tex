%***************************************************************************************%
%            ADAPTATIVE BALANCING TECHNIQUES APPLIED TO PARALLEL MECHANISMS             %
% 																					   	%
% July 11, 2014														 					%
% Authors: Tarcisio A. H. Coelho, Andre G. Coutinho, Renato M. M. Orsino				%
% bash adapta.sh							 											%
% 																						%
%***************************************************************************************%



%%%%%%%%%%%%%%%%%%%%%%
\documentclass[a4paper,11pt,brazil,fleqn]{article}
\synctex=1
%%%%%%%%%%%%%%%%%%%%%%

\usepackage{natbib}
\usepackage[english]{babel}
% \usepackage{amsmath,amssymb,amsthm,amsfonts,textcomp}
% \usepackage{eucal,eufrak,mathrsfs,bbm,stmaryrd}
\usepackage{color}
\usepackage{array,hhline,supertabular}
\usepackage[colorlinks,citecolor=black,urlcolor=black,linkcolor=black]{hyperref}
\usepackage[pdftex]{graphicx}
\usepackage{multicol}
\usepackage[symbol]{footmisc}
\usepackage{enumitem}
\usepackage{float}
\usepackage{titlesec}
\usepackage{nomencl}
\usepackage{amsthm}
\usepackage{EXTRAS/special-char}
\usepackage{EXTRAS/special-conf}
\usepackage{subfigure}

\graphicspath{{FIGURES/}{../FIGURES/}}
\makenomenclature


%%%%%%%%%%%%%%%%%%%%%%
\begin{document}
%%%%%%%%%%%%%%%%%%%%%%

\noindent
{\bf \huge Adaptative balancing techniques applied to parallel mechanisms}\\

\noindent
{\Large 	Tarcisio Antonio Hess Coelho$\,{}^\text{a}$, 
			Renato Maia Matarazzo Orsino$\,{}^\text{b}$, 
			Andr\'e Garnier Coutinho$\,{}^\text{a}$
}\\

\noindent
{${}^\text{a}$ \it Department of Mechatronics and Mechanical Systems Engineering, Escola Politecnica, 
University of Sao Paulo, Brazil. E-mail: tarchess@usp.br}

\noindent
{${}^\text{b}$ \it Department of Mechanical Engineering, Escola Politecnica, University of Sao Paulo, Brazil.}

\vspace{24pt}

% \begin{multicols}{1}

\section*{SUMMARY}

\noindent
KEYWORDS: {}

\printnomenclature[5em]

% basic mathematical alphabets

\nomenclature[A001]{$a,b, \ldots$}{Scalars, components of column-matrices or indexes}
\nomenclature[A002]{$A,B, \ldots$}{Scalars or components of matrices}

\nomenclature[A012]{$\ma, \mb, \ldots$}{Column-matrices}
\nomenclature[A013]{$\mA, \mB, \ldots$}{Matrices}

\nomenclature[A021]{$\va, \vb, \ldots$}{Vectors}
\nomenclature[A022]{$\vA, \vB, \ldots$}{Tensors}

\nomenclature[A031]{$\tta, \ttb, \ldots$}{Points}
\nomenclature[A032]{$\ttA, \ttB, \ldots$}{Coordinate systems}

\nomenclature[A041]{$\llA, \llB, \ldots$}{Rigid bodies or reference frames}

\nomenclature[A051]{$\ssA, \ssB, \ldots$}{Sets or multibody mechanical systems\footnote{
	A multibody mechanical system will be conceived as a set whose elements are 
	material bodies, joints, actuators, energy storage, dissipation and transformation elements
	and a mathematical model (which includes physical parameters, model variables and 
	constitutive, constraint and dynamic equations).
	}}


% special char

\nomenclature[CAA01]{$a_{n,l}$}{Adjustable physical parameter}
\nomenclature[CAA02]{$\nb a_{n,l}$}{Fixed physical parameter}
\nomenclature[CAA11]{$\mA_{n}$}{Jacobian matrix of kinematic invariants ($\mc_n$) with respect to 
	quasi-accelerations ($\dot\mp_n$)}
\nomenclature[CAA21]{$\va_{\ttp \rl \llE}$}{Acceleration of point $\ttp$ measured relatively to reference frame $\llE$}

\nomenclature[CAC11]{$\mc_{n}$}{Kinematic invariants (constraints) column-matrix}
\nomenclature[CAC51]{$\ssC^s$}{Differentiability class}
\nomenclature[CAC91]{$\ccos_{n,i}$}{Shorthand notation for $\cos(q_{n,i})$}

\nomenclature[CAD11]{$\md_{n}$}{Dynamic invariants column-matrix}
\nomenclature[CAD91]{$\dd$}{Differential operator}
\nomenclature[CGD91]{$\dl$}{Variation operator}

\nomenclature[CAF01]{$f_{n,j}$}{Generalized force}
\nomenclature[CAF11]{$\mf_{n}$}{Generalized forces column-matrix}
\nomenclature[CAF21]{$\vf_{s}$}{Arbitrary force}

\nomenclature[CAG01]{$g_{n,j}$}{Generalized gyroscopic inertia force}
\nomenclature[CAG11]{$\mg_{n}$}{Generalized gyroscopic inertia forces column-matrix}

\nomenclature[CAI01]{$I_s$}{Arbitrary moment of inertia}
\nomenclature[CAI21]{$\vI_{\llB \rl \ttp}$}{Inertia tensor of rigid body $\llB$ relative to point $\ttp$}
\nomenclature[CAI51]{$\ssI_{x}(\ssS_{n})$}{Set of indexes of variables $x_{n,r}$ defined in the model 
	of system $\ssS_{n}$, i.e., $\ssI_{x}(\ssS_{n}) = \{ r \,\vert\, x_{n,r} \in \ssS_{n} \}$ }

\nomenclature[CAM01]{$m_{s}$}{Arbitrary mass}
\nomenclature[CAM21]{$\vm_{s}$}{Arbitrary moment (torque)}
\nomenclature[CAM11]{$\mM_{n}$}{Generalized inertia matrix}

\nomenclature[CAN41]{$\llN$}{Inertial reference frame}
\nomenclature[CGN01]{$\nu_{x}(\ssS_{n})$}{Number of elements of the set $\ssI_{x}(\ssS_{n})$}
\nomenclature[CGN02]{$\nu\ssh(\ssS_{n})$}{Number of degrees of freedom of the mechanical system $\ssS_{n}$}

\nomenclature[CAP01]{$p_{n,j}$}{Quasi-velocity}
\nomenclature[CAP02]{$\dot p_{n,j}$}{Quasi-acceleration}
\nomenclature[CAP11]{$\mp_{n}$}{Quasi-velocities column-matrix}
\nomenclature[CAP12]{$\dot\mp_{n}$}{Quasi-accelerations column-matrix}

\nomenclature[CAQ01]{$q_{n,i}$}{Generalized coordinate}
\nomenclature[CAQ11]{$\mq_{n}$}{Generalized coordinates column-matrix}

\nomenclature[CAR21]{$\vr_{\ttp_2 \rl \ttp_1}$}{Position of point $\ttp_2$ relative to point $\ttp_1$}
\nomenclature[CAR51]{$\ssR^s$}{Set of $s$-tuples of real numbers}

\nomenclature[CAS91]{$\ssin_{n,i}$}{Shorthand notation for $\sin(q_{n,i})$}

\nomenclature[CAT01]{$t$}{Time}

\nomenclature[CAU01]{$u_{n,k}$}{Control input}

\nomenclature[CAV21]{$\vv_{\ttp \rl \llE}$}{Velocity of point $\ttp$ measured relatively to reference frame $\llE$}

\nomenclature[CGW21]{$\vomega_{\llB \rl \llE}$}{Angular velocity of rigid body or reference frame $\llB$ 
	measured relatively to reference frame $\llE$}

\nomenclature[CN011]{$\mzr$}{Null column-matrix or null matrix}
\nomenclature[CN021]{$\vzr$}{Null vector or null tensor}

\nomenclature[CN111]{$\mone$}{Identity matrix}
\nomenclature[CN121]{$\vone$}{Identity tensor}


% matrix/vector notations

\nomenclature[SM001]{$\nvct{x_{r}}$}{Column-matrix defined by the entries $x_{r}$}
\nomenclature[SM011]{$\nmat{X_{rs}}$}{Matrix defined by the entries $X_{rs}$}
\nomenclature[SM012]{$\ndmat{x_{r}}$}{Diagonal-matrix representation of $x_{r}$}

\nomenclature[SM021]{$\nvct{\vw}_{\ttE}$}{Coordinate column-matrix of vector $\vw$
 	in coordinate system $\ttE$}
\nomenclature[SM022]{$\nvct{\ttp}_{\ttE}$}{Coordinate column-matrix of point $\ttp$
 	in coordinate system $\ttE$}
\nomenclature[SM023]{$\nhvct{\ttp}_{\ttE}$}{Homogeneous coordinates column-matrix of point $\ttp$
 	in coordinate system $\ttE$} 

\nomenclature[SM031]{$\nmat{\vZ}_{\ttE' \rl \ttE}$}{Matrix representing tensor $\vZ$
 	in terms of coordinate systems $\ttE$ and $\ttE'$ (if $\vw'=\vZ \cdot \vw$, then 
 	$\nvct{\vw'}_{\ttE'}= \nmat{\vZ}_{\ttE' \rl \ttE} \, \nvct{\vw}_{\ttE}$)}
\nomenclature[SM032]{$\nsmat{\vw}_{\ttE \rl \ttE}$}{Skew-symmetric matrix representation
	of $\nvct{\vw}_{\ttE}$} 	

\nomenclature[SM101]{$\ntmat{\cdot}$}{Matrix transposition}
\nomenclature[SM102]{$\nimat{\cdot}$}{Matrix inversion}	




%--------------------INTRODUCTION--------------------%

\section{Introduction and literature review}\label{S01}

Balancing is an important issue related to the design of mechanical systems in general,
and also parallel mechanisms, in particular. In fact, the performance of parallel mechanisms
associated to specific applications depends on the choice of the balancing method, namely, 
either static or dynamic, either passive or active, whether it is valid for a given trajectory
or even for any motion.

% definition of Static balancing, purpose
In a statically balanced mechanism, the potential energy is invariant. Hence,
the actuator torques/forces are null at any configuration [6].
% definition of Dynamic balancing, purpose
On the other hand, in a dynamically balanced mechanism, the shaking forces and
shaking moments at its frame are reduced or even eliminated. As a consequence, 
the mechanism components are less susceptible to vibration, wear and fatigue [4],
improving its life.

% passive and active balancing
Passive balancing means that the original mechanism
architecture is modified by using some techniques.
The most common techniques are
the addition of counterweights
~\cite{1wijk,2arakelian,3seo,4wu,6wang,7wang,12russo,15coelho,16moradi}, 
the use of counter-rotating inertias~\cite{2arakelian,4wu,15coelho,16moradi} 
and the redistribution of masses~\cite{8alici,9alici}.
Alternatively, other works propose the addition of extra
links~\cite{5gosselin,12russo,13agrawal}.
However, for high speed manipulators, these techniques might cause the increase of the actuator
torques and the size of links and joints.
Hence, in order to avoid such undesirable effects, some 
authors~\cite{6wang,7wang,8alici,9alici,10dehkordi,13agrawal} recommend the use of
 elastic springs attached to the mechanism.

For the active balancing~\cite{2arakelian,3seo,11wang,14briot,15coelho,16moradi}, 
more complex modifications are needed to implement it. 
For instance, the counterweights position in the moving links might be altered
according to the end-effector load or the given trajectory. Hence, 
additional actuators/sensors and control are usually employed to reach satisfactory performance levels. Arakelian and Smith~\cite{2arakelian} employ a computer control of the motion of a inertia flywheel connected to the mechanism.

Moreover, Coelho et al.~\cite{15coelho} and Moradi et al.~\cite{16moradi} use the adaptive balancing to achieve the 
decoupling of dynamic equations for open-loop kinematic chain mechanisms. 
Consequently, this action simplifies the control of manipulators due to the fact that the 
actuators can be controlled independently. 

% esclarecer a contribuição
The main contribution of this work is to highlight the importance of the dynamic modelling 
process (see Figure~\ref{F01-101}) in order to achieve the compensation conditions associated to the chosen 
balancing technique. Due to the fact that parallel mechanisms have highly complex structures,
the use of dynamic formalisms that employ redundant generalized coordinates,
in association with the successive coupling of additional balancing elements 
to the original system model, can bring remarkable benefits.
Additionally, this book chapter also discusses the impact of
the dynamic model, developed in accordance with the methodology shown here, for the 
control strategy of parallel mechanisms.

% seções
This chapter is organized as follows. Section 2 treats of the theoretical background for the dynamic modelling and control, the adaptive balancing techniques are described in section 3, section 4 shows a case study for the planar 5-bar mechanism with revolute joints.  Finally, the conclusions are drawn in section 5.


% shaking force and shaking to keep base vibrations low [1]
% reduction of the variable dynamic loads on
%     the frame and, as a result, a reduction of vibrations [2]
%   However, similar to other robotic
%   devices, they exert forces and moments on their base while
%   moving, causing fatigue, vibration, noise, and disturbances in
%     the supporting structure of the mechanism [4]
% reactionless or dynamically balanced
% if the reaction forces (excluding gravity) and reaction moments—
% or shaking forces and shaking moments—at its base are
% identically equal to zero at all times and for any motion of the mechanism. [4]
% 
% static balancing is defined as the set of conditions under which the weight of the links of the mechanism does not
% produce any torque (or force) at the actuators under static conditions, for any con®guration of
%  the manipulator or mechanism. This condition is also referred to as gravity compensation. [6]
%  unbalanced forces on the base will lead to vibrations, wear and other undesirable side effects.[7]
% gravity compensation condition, the total potential
% energy of the system should remain constant for all of the
% manipulator’s configurations [10]
% constant total potential energy of the
% manipulator for all of its configurations. This follows that the
% manipulator is in equilibrium with zero actuator force, and
% less powerful and smaller actuators can be employed to move it. [8]
% the inertia-induced force (shaking force) and moment (shaking moment) transmitted to the mechanism
% frame. If their magnitudes and directions change throughout the operation of the mechanism, the mechanism
% will vibrate undesirably, and consequently, its dynamic performance will be unsatisfactory. [9]
% a technique that brings some modifications to the original kinematic chain
% of unbalanced mechanisms in such a way to obtain static balancing and complete decoupling of
% dynamic equations, thus eliminating Coriolis, centripetal, gravitational and cross inertia terms,
% even when payload changes. [14]

%
% and the actuator torques. 


% definition of passive and active balancing

% balancing techniques

 

% analytical approach numerical approach

% valid for any motion for given motion

% Simulation
% Experimental tests

\begin{figure}[H]
	\centering
	\includegraphics[scale=0.4]{model-1.png}
	\caption{Dynamic modelling process}
	\label{F01-101}
\end{figure}



%--------------------MODELING--------------------%

\input{SECTIONS/RENATO/2-1}

\subsection{Dynamic Models}\label{S02-2}

\begin{itemize}
\item[•] $\mathbb{\Re}$ Modelo do mecanismo \underline{R}\underline{R}

\begin{figure}[h!]
	\centering
	\includegraphics[scale=1.5]{RR.jpg}  
	\caption{Rob\^o \underline{R}\underline{R}}
	\label{fig:figura2}
\end{figure}

	\begin{itemize}
	\item[i)] Primeiro definimos $\nu_q$ coordenadas  $\mq$. Estas podem ser subdivididas em $\nu\ssh$ coordenadas independentes $\mq\ssh$ e $\nu_q^\circ$ 	coordenadas redudantes $\mq^\circ$.
	
	$$
	\mq = \begin{bmatrix}
	\mq\ssh \\
	\mq^\circ
	\end{bmatrix}
	$$

	No caso do mecanismo $\underline{R}\underline{R}$, temos:

	\begin{equation}
	\mq\ssh = \begin{bmatrix}
	\theta_1 & \theta_2
	\end{bmatrix}^T \\
	\end{equation}
	
	\begin{equation}
	\mq^\circ = \begin{bmatrix}
	x_1 & y_1 & x_2 & y_2
	\end{bmatrix}^T
	\end{equation}

	Com $\nu\ssh = 2$ e $\nu_q^\circ = 4$. Neste caso, as componentes de $\mq^\circ$ s\~ao as coordenadas dos centros de massa das barras, escritas no referencial 	inercial $O_{xy}$. \\
	
	\item[ii)] Depois definimos os vetores de velocidades absolutas:
	
	$$ \mw =
	\begin{bmatrix}
	\mw_\omega \\
	\mw_v
	\end{bmatrix}
	$$
	
	\begin{equation}
	\mw_\omega = \begin{bmatrix}
	\omega_{z1} & \omega_{z2}
	\end{bmatrix}^T
	\end{equation}
	
	\begin{equation}
	\mw_v = \begin{bmatrix}
	v_{x1} & v_{y1} & v_{x2} & v_{y2}
	\end{bmatrix}^T
	\end{equation}
	
	Sendo $\mw_v$ as componentes das velocidades absolutas dos centros de massa das barras, escritas nas bases presas às barras, e $\mw_\omega$ as componentes das velocidades angulares absolutas, escritas nas bases presas às barras. \\
	
	\item[iii)] Definimos $\nu_p$ coordenadas  $\mp$. Estas podem ser subdivididas em $\nu\ssh$ coordenadas independentes $\mp\ssh$ e $\nu_p^\circ$ coordenadas redudantes $\mp^\circ$. As coordenadas $\mp\ssh$ podem ser subdividas em $\nu_\omega\ssh$ velocidades angulares $\momega^{\#}$ e $\nu_v\ssh$ velocidades lineares $\mathbb{\nu}\ssh$. As coordenadas $\mp^\circ$ podem ser subdividas em $\nu_\omega^\circ$ velocidades angulares $\momega^\circ$ e $\nu_v^\circ$ velocidades lineares $\mathbb{\nu}^\circ$.
	
	
	\begin{multicols}{3}
	$ \mp = \begin{bmatrix}
	\mp\ssh \\
	\mp^\circ
	\end{bmatrix} $

	$ \mp\ssh = \begin{bmatrix}
	\momega\ssh \\
	\mathbb{\nu}\ssh
	\end{bmatrix} $

	$ \mp^\circ = \begin{bmatrix}
	\momega^\circ \\
	\mathbb{\nu}^\circ
	\end{bmatrix} $

	\end{multicols}
	
	Como \'e conveniente que as velocidades generalizadas $\mp$ sejam velocidades absolutas, escolhemos as componentes de $\mp$ como sendo as mesmas componentes de $\mw$, respeitando a ordena\c{c}\~ao indicada acima. \\ 

	No caso do mecanismo $\underline{R}\underline{R}$, temos:

	\begin{equation}
	\momega\ssh = \begin{bmatrix}
	\omega_{z1} \\
	\omega_{z2}
	\end{bmatrix}
	\end{equation}
	
	\begin{equation}
	\mathbb{\nu}\ssh = \emptyset
	\end{equation}		
	
	\begin{equation}
	\momega^\circ = \emptyset
	\end{equation}		
	
	\begin{equation}
	\mathbb{\nu}^\circ = \begin{bmatrix}
	v_{x1} & v_{y1} & v_{x2} & v_{y2}
	\end{bmatrix}^T
	\end{equation}\\

	Com $\nu_\omega\ssh = 2$, $\nu_v\ssh = 0$, $\nu_\omega^\circ = 0$, $\nu_v^\circ = 4$ e $\nu_p^\circ = \nu_\omega^\circ + \nu_v^\circ = 4$. \\
	
		\item[iv)] Realizamos a cinemática de posi\c{c}\~ao para os centros de massa das barras, de modo a relacionar as coordenadas $\mq^\circ$ com as coordenadas $\mq\ssh$. Para isso, utilizamos matrizes de transforma\c{c}\~ao homog\^enea.



	$$ \nmat{\vH}_{\ttB_0 \rl \ttB_1}  =
	\begin{bmatrix}
	 Rot(\theta_1, z_0) & \nmat{\overrightarrow{\ttO_0 \ttO_1}}_{\ttB_0} \\
	 \mzr_{2x1} & 1 \\
	\end{bmatrix}
	=
	\begin{bmatrix}
	 \ccos_1 & -\ssin_1 & 0 \\
	 \ssin_1 & \ccos_1 & 0 \\
	 0 & 0 & 1 \\
	\end{bmatrix} ;
	\begin{bmatrix}
	\nmat{\overrightarrow{\ttO_1 \ttG_1}}_{\ttB_1} \\
	1
	\end{bmatrix}
	=
	\begin{bmatrix}
	l_{1g} \\
	0 \\
	1 \\
	\end{bmatrix}
	$$

	$$ \nmat{\vH}_{\ttB_1 \rl \ttB_2} =
	\begin{bmatrix}
	 Rot(\theta_2, z_1) & \nmat{\overrightarrow{\ttO_1 \ttO_2}}_{\ttB_1} \\
	 \mzr_{2x1} & 1 \\
	\end{bmatrix}
	=
	\begin{bmatrix}
	 \ccos_2 & -\ssin_2 & l_1 \\
	 \ssin_2 & \ccos_2 & 0 \\
	 0 & 0 & 1 \\
	\end{bmatrix} ;
	\begin{bmatrix}
	 \nmat{\overrightarrow{\ttO_2 \ttG_2}}_{\ttB_2} \\
	1
	\end{bmatrix}
	=
	\begin{bmatrix}
	l_{2g} \\
	0 \\
	1 \\
	\end{bmatrix}
	$$

	$$
	\nmat{\vH}_{\ttB_0 \rl \ttB_2} = \nmat{\vH}_{\ttB_0 \rl \ttB_1} \nmat{\vH}_{\ttB_1 \rl \ttB_2}  =
	\begin{bmatrix}
	\ccos_{1+2} & -\ssin_{1+2} & l_1 \ccos_1 \\
	\ssin_{1+2} & \ccos_{1+2} & l_1 \ssin_1 \\
	0 & 0 & 1 \\
	\end{bmatrix}
	$$


	\begin{equation}
	\therefore \begin{bmatrix}
	x_1 \\
	y_1 \\
	1 \\
	\end{bmatrix}
	=
	\nmat{\vH}_{\ttB_0 \rl \ttB_1}
	\begin{bmatrix}
	\nmat{\overrightarrow{\ttO_1 \ttG_1}}_{\ttB_1} \\
	1
	\end{bmatrix}
	=
	\begin{bmatrix}
	 l_{1g} \ccos_1 \\
	 l_{1g} \ssin_1 \\
	 1 \\
	\end{bmatrix}
	\end{equation}

	\begin{equation}
	\begin{bmatrix}
	x_2 \\
	y_2 \\
	1 \\
	\end{bmatrix}
	=
	\nmat{\vH}_{\ttB_0 \rl \ttB_2}
	\begin{bmatrix}
	 \nmat{\overrightarrow{\ttO_2 \ttG_2}}_{\ttB_2} \\
	1
	\end{bmatrix}
	=
	\begin{bmatrix}
	 l_1 \ccos_1 + l_{2g} \ccos_{1+2} \\
	 l_1 \ssin_1 + l_{2g} \ssin_{1+2} \\
	 1 \\
	\end{bmatrix}
	\end{equation}

	Repare que a partir das matrizes de transforma\c{c}\~ao homogênea encontradas, encontramos também as seguintes matrizes de mudan\c{c}a de base:
	
	\begin{equation}
	\mR_1 = \nmat{\vone}_{\ttB_0 \rl \ttB_1} =
	\begin{bmatrix}
	 \ccos_1 & -\ssin_1  \\
	 \ssin_1 & \ccos_1  \\
	\end{bmatrix}
	\end{equation}
	
	\begin{equation}
	\mR_2 = \nmat{\vone}_{\ttB_0 \rl \ttB_2} =
	\begin{bmatrix}
	 \ccos_{1+2} & -\ssin_{1+2}  \\
	 \ssin_{1+2} & \ccos_{1+2}  \\
	\end{bmatrix}
	\end{equation}		

	Com a cinemática de posi\c{c}\~ao, conseguimos obter $\nu_q^\circ = 4$ equa\c{c}\~oes vinculares de posi\c{c}\~ao. Sendo assim, o vetor dos v\'inculos de posi\c{c}\~ao é dado por:
	
	
	\begin{equation}
	\mphi(\mq)
	=
	\begin{bmatrix}
	x_1 - l_{1g} \ccos_1 \\
	y_1 - l_{1g} \ssin_1 \\
	x_2 - l_1 \ccos_1 - l_{2g} \ccos_{1+2} \\
	y_2 - l_1 \ssin_1 - l_{2g} \ssin_{1+2} \\
	\end{bmatrix}
	\end{equation}
	
	\item[v)] Utilizamos as matrizes de rota\c{c}\~ao para calcular as velocidades angulares em fun\c{c}\~ao de $\mq\ssh$ e $\dot{\mq}\ssh$:
	
	\begin{equation}
	\nsmat{\vomega_1}_{\ttB_1 \rl \ttB_1}  = \mR_1^T \dot{\mR}_1 =
	\begin{bmatrix}
	0 & -\dot{\theta}_1  \\
	\dot{\theta}_1 & 0  \\
	\end{bmatrix}
	\Rightarrow
	\nmat{\vomega_1}_{\ttB_1} = \dot{\theta}_1 \hat{k}
	\end{equation}
	
	\begin{equation}
	\nsmat{\vomega_2}_{\ttB_2 \rl \ttB_2} = \mR_2^T \dot{\mR}_2 =
	\begin{bmatrix}
	0 & -\dot{\theta}_1 -\dot{\theta}_2 \\
	\dot{\theta}_1 + \dot{\theta}_2 & 0  \\
	\end{bmatrix}
	\Rightarrow
	\nmat{\vomega_2}_{\ttB_2} = (\dot{\theta}_1 + \dot{\theta}_2) \hat{k}
	\end{equation}
	
	\item[vi)] Derivamos as equa\c{c}\~oes de posi\c{c}\~ao ((29) e (30)) para encontrar as velocidades dos centros de massa:
	
	\begin{equation}
	\nmat{\vv_1}_{\ttB_0} =
	\begin{bmatrix}
	\dot{x}_1 \\
	\dot{y}_1 \\
	\end{bmatrix}
	=
	\begin{bmatrix}
	- l_{1g} \ssin_1 \dot{\theta}_1 \\
	l_{1g} \ccos_1 \dot{\theta}_1 \\
	\end{bmatrix}
	\end{equation}
	
	\begin{equation}
	\nmat{\vv_2}_{\ttB_0} =
	\begin{bmatrix}
	\dot{x}_2 \\
	\dot{y}_2 \\
	\end{bmatrix}
	=
	\begin{bmatrix}
	- l_1 \ssin_1 \dot{\theta}_1 - l_{2g} \ssin_{1+2} ( \dot{\theta}_1 + \dot{\theta}_2)  \\
	l_1 \ccos_1 \dot{\theta}_1 + l_{2g} \ccos_{1+2} ( \dot{\theta}_1 + \dot{\theta}_2)  \\
	\end{bmatrix}
	\end{equation}
	
	\item[vii)] Passamos as velocidades dos centros de massa para as bases presas nas barras:

	$$ \nmat{\vv_1}_{\ttB_1} = \nmat{\vone}_{\ttB_1 \rl \ttB_0} \nmat{\vv_1}_{\ttB_0} = \mR_1^T \nmat{\vv_1}_{\ttB_0} $$
	$$ \nmat{\vv_2}_{\ttB_2} = \nmat{\vone}_{\ttB_2 \rl \ttB_0} \nmat{\vv_2}_{\ttB_0} = \mR_2^T \nmat{\vv_2}_{\ttB_0} $$
	
	Definindo:
	
	
	\begin{equation}
	\mR_\clubsuit =
	\begin{bmatrix}
	\mR_1 & \mzr_{2x2} \\
	\mzr_{2x2} & \mR_2
	\end{bmatrix}
	\end{equation}
	
	Temos:
	
	\begin{equation}
	\mw_v = \mR_\clubsuit^T \dot{\mq}^\circ
	\end{equation}
	
	
	\begin{equation}
	\therefore
	\begin{bmatrix}
	v_{x1} \\
	v_{y1} \\
	v_{x2} \\
	v_{y2} \\
	\end{bmatrix}
	=
	\begin{bmatrix}
	\ccos_1 & -\ssin_1 & 0 & 0 \\
	\ssin_1 & \ccos_1 & 0 & 0 \\
	0 & 0 & \ccos_{1+2} & -\ssin_{1+2} \\
	0 & 0 & \ssin_{1+2} & \ccos_{1+2} \\
	\end{bmatrix}^T
	\begin{bmatrix}
	\dot{x}_1 \\
	\dot{y}_1 \\	
	\dot{x}_2 \\
	\dot{y}_2 \\	
	\end{bmatrix}
	=
	\begin{bmatrix}
	0 \\
	l_{1g} \dot{\theta}_1 \\
	l_1 \ssin_2 \dot{\theta}_1\\
	(l_1 \ccos_2 + l_{2g} )\dot{\theta}_1 + l_{2g} \dot{\theta}_2 \\
	\end{bmatrix}
	\end{equation}
	
	\item[viii)] Montamos os vetores $\mp\ssh$ e $\mp^\circ$ em fun\c{c}\~ao de $\mq\ssh$ e $\dot{\mq}\ssh$:

	\begin{equation}
	\mp\ssh = 
	\begin{bmatrix}
	\omega_{z1} \\
	\omega_{z2} \\
	\end{bmatrix}
	= \mp\ssh_\star (\mq\ssh, \dot{\mq}\ssh ) =
	\begin{bmatrix}
	\dot{\theta}_1 \\
	\dot{\theta}_1 +\dot{\theta}_2 \\
	\end{bmatrix}
	\end{equation}
	
	\begin{equation}
	\mp^\circ = 
	\begin{bmatrix}
	v_{x1} \\
	v_{y1} \\
	v_{x2} \\
	v_{y2} \\
	\end{bmatrix}
	= \mp^\circ_ \star (\mq\ssh, \dot{\mq}\ssh ) =
	\begin{bmatrix}
	0 \\
	l_{1g} \dot{\theta}_1 \\
	l_1 \ssin_2 \dot{\theta}_1\\
	(l_1 \ccos_2 + l_{2g} )\dot{\theta}_1 + l_{2g} \dot{\theta}_2 \\
	\end{bmatrix}
	\end{equation}
	
	\item[ix)] Utilizando o fato de que $\mp\ssh_\star (\mq\ssh, \dot{\mq}\ssh )$ e $\mp^\circ_ \star (\mq\ssh, \dot{\mq}\ssh )$ s\~ao lineares em $\dot{\mq}\ssh$, encontramos as transforma\c{c}\~oes lineares $\mathbb{\Psi}(\mq)$ e $\mathbb{\Upsilon}(\mq)$ e o vetor dos v\'inculos de velocidades $\mathbb{\Lambda}(\mq,\mp)$:
	
	\begin{equation}
	\mp\ssh = \mp\ssh_\star (\mq\ssh, \dot{\mq}\ssh ) = \frac{\partial \mp\ssh_\star}{\partial \dot{\mq}\ssh} \dot{\mq}\ssh = \mathbb{\Psi} \dot{\mq}\ssh
	\end{equation}
	
	\begin{equation}
	\mp^\circ = \mp^\circ_\star (\mq\ssh, \dot{\mq}\ssh ) = \frac{\partial \mp^\circ_\star}{\partial \dot{\mq}\ssh} \dot{\mq}\ssh = \mathbb{\Upsilon} \dot{\mq}\ssh
	\end{equation}

	No caso do mecanismo $\underline{R}\underline{R}$, temos:
	
	\begin{equation}
	\mathbb{\Psi} = \frac{\partial \mp\ssh_\star}{\partial \dot{\mq}\ssh} =
	\begin{bmatrix}
	1 & 0  \\
	1 & 1  \\
	\end{bmatrix}
	\end{equation}
	
	\begin{equation}
	\mathbb{\Upsilon} = \frac{\partial \mp^\circ_\star}{\partial \dot{\mq}\ssh} =
	\begin{bmatrix}
	0 & 0 \\
	l_{1g} & 0 \\
	l_1 \ssin_2 & 0 \\
	l_1 \ccos_2 + l_{2g} & l_{2g} 
	\end{bmatrix}
	\end{equation}\\

	Como $\mp\ssh$ e $\dot{\mq}\ssh$ s\~ao independentes e tem o mesmo tamanho:

	$$ \dot{\mq}\ssh = \mathbb{\Psi}^{-1} \mp\ssh $$
	$$ \Rightarrow  \mp^\circ = \mathbb{\Upsilon} \mathbb{\Psi}^{-1} \mp\ssh $$
	\begin{equation}
	\therefore \mathbb{\Lambda}(\mq,\mp) = \mathbb{\Upsilon} \mathbb{\Psi}^{-1} \mp\ssh - \mp^\circ  = 0
	\end{equation}
	
	\item[x)] A partir dos v\'inculos de velocidades, encontramos a matriz $\mathbb{C}$ dos v\'inculos cinem\'aticos:
	
	$$ \mp^\circ = \mathbb{\Upsilon} \mathbb{\Psi}^{-1} \mp\ssh \Rightarrow \mp =
	\begin{bmatrix}
	\mone\\
	\mathbb{\Upsilon} \mathbb{\Psi}^{-1}
	\end{bmatrix}
	\mp\ssh
	$$
	
	\begin{equation}
	\therefore \mathbb{C} =
	\begin{bmatrix}
	\mone\\
	\mathbb{\Upsilon} \mathbb{\Psi}^{-1}
	\end{bmatrix}
	=
	\begin{bmatrix}
	1 & 0 \\
	0 & 1 \\
	0 & 0 \\
	l_{1g} & 0\\
	l_1 s_{2} & 0 \\
	l_1 c_{2} & l_{2g} \\
	\end{bmatrix}
	\end{equation}
	
	\item[xi)] Como (39) e (43) são transforma\c{c}\~oes invers\'iveis, encontramos a transforma\c{c}\~ao linear $\Gamma(q)$:
	
	\begin{equation}
	\dot{q} =
	\begin{bmatrix}
	\dot{q}^{\#} \\
	\dot{q}^o \\
	\end{bmatrix}
	= \dot{Q}(q,p) =
	\begin{bmatrix}
	\Psi^{-1} (q) p^{\#} \\
	\mathbf{R} (q) \nu_v (p)
	\end{bmatrix}
	=
	\begin{bmatrix}
	\omega_{z1} \\
	\omega_{z2}-\omega_{z1}\\
	v_{x1} c_1 - v_{y1} s_1 \\
	v_{x1} s_1 + v_{y1} c_1 \\
	v_{x2} c_{1+2} - v_{y2} s_{1+2} \\
	v_{x2} s_{1+2} + v_{y2} c_{1+2} \\
	\end{bmatrix}
	\end{equation}
	
	\begin{equation}
	\Gamma(q) = \frac{\partial \dot{Q}}{\partial p} =
	\begin{bmatrix}
	1 & 0 & 0 & 0 & 0 & 0 \\
	-1 & 1 & 0 & 0 & 0 & 0 \\
	0 & 0 & c_1 & -s_1 & 0 & 0 \\
	0 & 0 & s_1 & c_1 & 0 & 0 \\
	0 & 0 & 0 & 0 & c_{1+2} & -s_{1+2} \\
	0 & 0 & 0 & 0 & s_{1+2} & c_{1+2} \\
	\end{bmatrix}
	\end{equation}

	\item[xii)] Aplicamos o métodos de Gibbs-Appel extendido: \\
	
	O método de Gibbs-Appell apresenta certa simularidade com o método de Lagrange, pois utiliza derivadas de uma fun\c{c}\~ao energia para encontrar a equa\c{c}\~oes de movimento do sistema. Por\'em, a fun\c{c}\~ao energia utilizada não é a energia cinética, mas sim a energia de acelera\c{c}\~oes. A energia de acelera\c{c}\~oes para um corpo r\'igido \'e dada pela seguinte express\~ao:

	$$ \mathcal{S} = \frac{1}{2} m ( a_G \cdot a_G ) + \frac{1}{2}( \dot{\omega} \cdot \mathbb{I} \dot{\omega} + 2 \dot{\omega} (\omega \wedge 	\mathbb{I} \omega )  )  $$
 
	Sendo $m$ a massa do corpo r\'igido, $\mathbb{I}$ seu tensor de in\'ercia, $a_G$ o vetor acelera\c{c}\~ao absoluta de seu centro de massa e $\omega$ o vetor velocidade angular absoluta.  \\

	O modelo dinâmico utilizando o método de Gibbs-Appel extendido, é dado pela seguinte expressão:
	
	\begin{equation}
	\mathbb{C}(q)^T ( \mathbb{M}(q) \dot{p} + \mathbb{V}(q,p) + \mathbb{G}(q)) = (\Psi^T)^{-1} f_{\dot{q}^{\#}} 
	\end{equation}
	
	Com:
	
	\begin{equation}
	\mathbb{M}(q) = \frac{\partial^2 \mathcal{S}}{\partial \dot{p}^2}
	\end{equation}
	
	\begin{equation}
	\mathbb{V}(q,p) =  \frac{\partial \mathcal{S}}{\partial \dot{p}} - \frac{\partial^2 \mathcal{S}}{\partial \dot{p}^2} \dot{p}
	\end{equation}
	
	\begin{equation}
	\mathbb{G}(q) =  \Gamma^T \frac{\partial E_p}{\partial q}
	\end{equation}
	
	Sendo $E_p$ a energia potencial do sistema e $f_{\dot{q}^{\#}}$ os esfor\c{c}os nas dire\c{c}\~oes de $\dot{q}^{\#}$.
	
	No caso do mecanismo $\underline{R}\underline{R}$, temos:
	
	\begin{equation}
	\mathcal{S} = \frac{1}{2} \Big( m_1 ( \dot{v}_{x1}^2 + \dot{v}_{y1}^2 ) + m_2 ( \dot{v}_{x2}^2 + \dot{v}_{y2}^2  ) + J_{z1} \dot{\omega}_{z1}^2 + + J_{z2} \dot{\omega}_{z2}^2 \Big)
	\end{equation}
	
	\begin{equation}
	E_p = m_1 g y_1 + m_2 g y_2
	\end{equation}
	
	Calculando as derivadas:
	
	\begin{equation}
	\mathbb{M}(q) =
	\begin{bmatrix}
	J_{z1} & 0 & 0 & 0 & 0 & 0 \\
	0 & J_{z2} & 0 & 0 & 0 & 0 \\
	0 & 0 & m_1 & 0 & 0 & 0 \\
	0 & 0 & 0 & m_1 & 0 & 0 \\
	0 & 0 & 0 & 0 & m_2 & 0 \\
	0 & 0 & 0 & 0 & 0 & m_2 \\
	\end{bmatrix}
	\end{equation}
	
	\begin{equation}
	\mathbb{V}(q,p) = 0
	\end{equation}
	
	\begin{equation}
	\mathbb{G}(q) =
	\begin{bmatrix}
	0 \\
	0 \\
	m_1 g s_1 \\
	m_1 g c_1 \\
	m_1 g s_{1+2} \\
	m_1 g c_{1+2}\\
	\end{bmatrix}
	\end{equation}
	
	Sendo assim, o modelo dinâmico para o mecanismo \underline{R}\underline{R} é dado por:
	
	\begin{equation}
	\begin{bmatrix}
	1 & 0 \\
	0 & 1 \\
	0 & 0 \\
	l_{1g} & 0\\
	l_1 s_{2} & 0 \\
	l_1 c_{2} & l_{2g} \\
	\end{bmatrix}^T
	\begin{Bmatrix}
		\begin{bmatrix}
		J_{z1} \dot{\omega}_{z11} \\
		J_{z2} \dot{\omega}_{z12} \\
		m_1 \dot{v}_{x11} \\ 
		m_1 \dot{v}_{y11} \\
		m_2 \dot{v}_{x12} \\
		m_2 \dot{v}_{y12} \\
		\end{bmatrix}
		+
		\begin{bmatrix}
		0 \\
		0 \\
		m_1 g s_{1} \\
		m_1 g c_{1} \\
		m_2 g s_{1+2} \\
		m_2 g c_{1+2} \\
		\end{bmatrix}
	\end{Bmatrix}
	=
	\begin{bmatrix}
	1 & 1 \\
	0 & 1 \\
	\end{bmatrix}^{-1}
	\begin{bmatrix}
	\tau_{11} \\
	\tau_{12} \\
	\end{bmatrix}
	\end{equation}
	
	Repare que o modelo não depende das coordenadas $q^o$. Elas foram uteis para a dedu\c{c}\~ao do modelo, mas com o modelo deduzido elas não tem mais utilidade.
	
	
	
	
	\end{itemize}
\end{itemize}

\begin{itemize}

\item Massa pontual:

\begin{equation}
	\mq_n = \begin{bmatrix}
	x_n \\
	y_n \\
	\end{bmatrix}
	=
	\begin{bmatrix}
	q_{n,1} \\
	q_{n,2} \\
	\end{bmatrix}
\end{equation}

\begin{equation}
	\mp_n = \begin{bmatrix}
	p_{n,1} \\
	p_{n,2} \\
	\end{bmatrix}
	=
	\begin{bmatrix}
	1 & 0 \\
	0 & 1 \\
	\end{bmatrix}
	\begin{bmatrix}
	\dot{q}_{n,1} \\
	\dot{q}_{n,2} \\
	\end{bmatrix}
\end{equation}

\begin{equation}
	\begin{cases}

	\begin{bmatrix}
	\dot{q}_{n,1} \\
	\dot{q}_{n,2} \\
	\end{bmatrix}
	=
	\begin{bmatrix}
	1 & 0 \\
	0 & 1 \\
	\end{bmatrix}
	\begin{bmatrix}
	p_{n,1} \\
	p_{n,2} \\
	\end{bmatrix} \\

	\begin{bmatrix}
	M_n \dot{p}_{n,1} \\
	M_n \dot{p}_{n,2} \\
	\end{bmatrix}
	+
	g \begin{bmatrix}
	0 \\
	M_n \\
	\end{bmatrix}
	=
	\begin{bmatrix}
	f_{n, 1} \\
	f_{n, 2} \\
	\end{bmatrix}
	\end{cases}
\end{equation}

Que pode ser reescrito como:

$$
\begin{bmatrix}
M_n & 0 \\
0 & M_n \\
\end{bmatrix}
\begin{bmatrix}
\ddot{q}_{n,1} \\
\ddot{q}_{n,2} \\
\end{bmatrix}
+
g \begin{bmatrix}
0 \\
M_n \\
\end{bmatrix}
=
\begin{bmatrix}
f_{n,1} \\
f_{n,2} \\
\end{bmatrix}
$$

\item RR:

\begin{equation}
	\mq_n = \begin{bmatrix}
	\theta_{1 \, n} \\
	\theta_{2 \, n} \\
	\end{bmatrix}
	=
	\begin{bmatrix}
	q_{n,1} \\
	q_{n,2} \\
	\end{bmatrix}
\end{equation}

\begin{equation}
	\mp_n = \begin{bmatrix}
	p_{n,1} \\
	p_{n,2} \\
	p_{n,3} \\
	p_{n,4} \\
	p_{n,5} \\
	\end{bmatrix}
	=
	\begin{bmatrix}
	1 & 0 \\
	1 & 1 \\
	l_{g1} & 0\\
	l_1 \ssin_{n,2} & 0 \\
	l_{g2} + l_1 \ccos_{n,2} & l_{g2}
	\end{bmatrix}
	\begin{bmatrix}
	\dot{q}_{n,1} \\
	\dot{q}_{n,2} \\
	\end{bmatrix}
\end{equation}

\begin{equation}
	\begin{cases}

	\begin{bmatrix}
	\dot{q}_{n,1} \\
	\dot{q}_{n,2} \\
	\end{bmatrix}
	=
	\begin{bmatrix}
	1 & 0 \\
	-1 & 1\\
	\end{bmatrix}
	\begin{bmatrix}
	p_{n,1} \\
	p_{n,2} \\
	\end{bmatrix} \\


	\begin{bmatrix}
	1 & 0 \\
	1 & 1 \\
	\end{bmatrix}^T
	\begin{bmatrix}
	 1 & 0 \\
	 0 & 1 \\
	 l_{g1} & 0 \\
	 l_1 \ssin_{n,2} & 0 \\
	 l_1 \ccos_{n,2} & l_{g2} \\
	\end{bmatrix}^T
	\begin{Bmatrix}
		\begin{bmatrix}
		J_{z1} \dot{p}_{n,1} \\
		J_{z2} \dot{p}_{n,2} \\
		m_1 \dot{p}_{n,3} \\
		m_2 \dot{p}_{n,4} \\
		m_2 \dot{p}_{n,5} \\
		\end{bmatrix}
		+
		\begin{bmatrix}
		0 \\
		0 \\
		0 \\
		- m_1 p_{n,2} p_{n,5} \\
		  m_1 p_{n,2} p_{n,4} \\
		\end{bmatrix}
		+
		g \begin{bmatrix}
		0 \\
		0 \\
		m_1 \ccos_{n,1} \\
		m_2 \ssin_{n,1+2} \\
		m_2 \ccos_{n,1+2} \\
		\end{bmatrix}
	\end{Bmatrix}
	=
	\begin{bmatrix}
	u_{n,1} \\
	u_{n,2}
	\end{bmatrix} \\

	\begin{bmatrix}
	l_{g1} & 0 & -1 & 0 & 0 \\
	l_1 \ssin_{n,2} & 0 & 0 & -1 & 0 \\
	l_1 \ccos_{n,2} & l_{g2}  & 0 & 0 & -1 \\
	\end{bmatrix}
	\begin{bmatrix}
	\dot{p}_{n,1} \\
	\dot{p}_{n,2} \\
	\dot{p}_{n,3} \\
	\dot{p}_{n,4} \\
	\dot{p}_{n,5} \\
	\end{bmatrix}
	=
	-
	\begin{bmatrix}
	0 \\
	l_1 \ccos_{n,2} p_{n,1} (-p_{n,1}+p_{n,2}) \\
	l_1 \ssin_{n,2} p_{n,1} (p_{n,1}-p_{n,2}) \\
	\end{bmatrix}

	\end{cases}
\end{equation}

Que pode ser reescrito como:

$$
\begin{bmatrix}
J_{z1} + J_{z2} + m_1 l_{g1}^2 + m_2 (l_1^2 + 2 l_1 l_{g2} \ccos_{n,2} + l_{g2}^2) & J_{z2} + m_2 l_{g2} (l_1 \ccos_{n,2} + l_{g2}) \\
J_{z2} + m_2 l_{g2} (l_1 \ccos_{n,2} + l_{g2}) & J_{z2} + m_2 l_{g2}^2
\end{bmatrix}
\begin{bmatrix}
\ddot{q}_{n,1} \\
\ddot{q}_{n,2} \\
\end{bmatrix}
$$
$$
+
\begin{bmatrix}
- m_2 l_1 l_{g2} \ssin_{n,2} \dot{q}_{n,2}^2 -2 m_2 l_1 l_{g2} \ssin_{n,2} \dot{q}_{n,1}  \dot{q} _{n,2} \\
m_2 l_1 l_{g2} \ssin_{n,2} \dot{q}_{n,1}^2 \\
\end{bmatrix}
+
g \begin{bmatrix}
m_1 l_{g1} \ccos_{n,1} + m_2 (l_{g2} \ccos_{n,1+2} + l_1 \ccos_{n,1}) \\
 m_2 l_{g2} \ccos_{n,1+2} \\
\end{bmatrix}
=
\begin{bmatrix}
u_{n,1} \\
u_{n,2} \\
\end{bmatrix}
$$

\item RR (0) com 2 massas acopladas (1 e 2):

\begin{equation}
	\begin{cases}

	\begin{bmatrix}
	\dot{q}_{0,1} \\
	\dot{q}_{0,2} \\
	\end{bmatrix}
	=
	\begin{bmatrix}
	1 & 0 \\
	-1 & 1\\
	\end{bmatrix}
	\begin{bmatrix}
	p_{0,1} \\
	p_{0,2} \\
	\end{bmatrix} \\

	\begin{bmatrix}
	1 & 0 \\
	1 & 1 \\
	\end{bmatrix}^T
	\begin{bmatrix}
	 1 & 0 \\
	 0 & 1 \\
	 l_{g1} & 0 \\
	 l_1 \ssin_{0,2} & 0 \\
	 l_1 \ccos_{0,2} & l_{g2} \\
	 -L_1 \ssin_{0,1} & 0 \\
	 L_1 \ccos_{0,1} & 0 \\
	 -l_1 \ssin_{0,1} & -L_2 \ssin_{0,1} \\
	 l_1 \ccos_{0,1} & L_2 \ccos_{0,1} \\
	\end{bmatrix}^T
	\begin{Bmatrix}
		\begin{bmatrix}
		J_{z1} \dot{p}_{0,1} \\
		J_{z2} \dot{p}_{0,2} \\
		m_1 \dot{p}_{0,3} \\
		m_2 \dot{p}_{0,4} \\
		m_2 \dot{p}_{0,5} \\
		M_1 \dot{p}_{1,1} \\
		M_1 \dot{p}_{1,2} \\
		M_2 \dot{p}_{2,1} \\
		M_2 \dot{p}_{2,1} \\
		\end{bmatrix}
		+
		\begin{bmatrix}
		0 \\
		0 \\
		0 \\
		- m_1 p_{0,2} p_{0,5} \\
		  m_1 p_{0,2} p_{0,4} \\
		0 \\
		0 \\
		0 \\
		0 \\
		\end{bmatrix}
		+
		g \begin{bmatrix}
		0 \\
		0 \\
		m_1 \ccos_{0,1} \\
		m_2 \ssin_{0,1+2} \\
		m_2 \ccos_{0,1+2} \\
		0 \\
		M_1 \\
		0 \\
		M_2 \\
		\end{bmatrix}
	\end{Bmatrix}
	=
	\begin{bmatrix}
	u_{0,1} \\
	u_{0,2}
	\end{bmatrix} \\

	\begin{bmatrix}
	l_{g1} & 0 & -1 & 0 & 0 & 0 & 0 & 0 & 0 \\
	l_1 \ssin_{i,2} & 0 & 0 & -1 & 0 & 0 & 0 & 0 & 0 \\
	l_1 \ccos_{i,2} & l_{g2}  & 0 & 0 & -1 & 0 & 0 & 0 & 0 \\
	L_1 \ssin_{0,1} & 0 & 0 & 0 & 0 & 1 & 0 & 0 & 0 \\
	-L_1 \ccos_{0,1} & 0 & 0 & 0 & 0 & 0 & 1 & 0 & 0 \\
	l_1 \ssin_{0,1} & L_2 \ssin_{0,1+2}  & 0 & 0 & 0 & 0 & 0 & 1 & 0 \\
	-l_1 \ccos_{0,1} & -L_2 \ccos_{0,1+2} & 0 & 0 & 0 & 0 & 0 & 0 & 1 \\
	\end{bmatrix}
	\begin{bmatrix}
	\dot{p}_{0,1} \\
	\dot{p}_{0,2} \\
	\dot{p}_{0,3} \\
	\dot{p}_{0,4} \\
	\dot{p}_{0,5} \\
	\dot{p}_{1,1} \\
	\dot{p}_{1,2} \\
	\dot{p}_{2,1} \\
	\dot{p}_{2,1} \\
	\end{bmatrix}
	=
	-
	\begin{bmatrix}
	0 \\
	l_1 \ccos_{0,2} p_{0,1} (-p_{0,1}+p_{0,2}) \\
	l_1 \ssin_{0,2} p_{0,1} (p_{0,1}-p_{0,2}) \\
	L_1 \ccos_{0,1} p_{0,1}^2 \\
	L_1 \ssin_{0,1} p_{0,1}^2 \\
	l_1 \ccos_{0,1} p_{0,1}^2 + L_1 \ccos_{0,1+2} p_{0,2}^2  \\
	l_1 \ssin_{0,1} p_{0,1}^2 + L_1 \ssin_{0,1+2} p_{0,2}^2 \\
	\end{bmatrix}

	\end{cases}
\end{equation}

Que pode ser reescrito como:

\begin{equation}
	\mM\ssh \ddot{\mq}_0 + \mv\ssh + \mg\ssh = \mu_0
\end{equation}

Sendo:

\begin{equation}
	\mM\ssh_{1,1} = J_{z1} + J_{z2} + M_1 L_1^2 + M_2 L_2^2 + m_1 l_{g1}^2 + m_2 l_{g2}^2 + (M_2 + m_2)l_1^2 + 2 l_1 \ccos_{0,2} (L_2 M_2 + m_2 l_{g2})
\end{equation}
\begin{equation}
	\mM\ssh_{1,2} = \mM\ssh_{2,1} = J_{z2} + M_2 L_2 (l_1 \ccos_{0,2} + L_2) + m_2 l_{g2} (l_1 \ccos_{0,2} + l_{g2})
\end{equation}
\begin{equation}
	\mM\ssh_{2,2} = J_{z2} + M_2 L_2^2 + m_2 l_{g2}^2
\end{equation}

\begin{equation}
	\mv\ssh_1 = - (M_2 L_2 + m_2 l_{g2}) l_1 \ssin_{0,2} \dot{q}_{0,2} (2 \dot{q}_{0,1} + \dot{q}_{0,2})  
\end{equation}
\begin{equation}
	\mv\ssh_2 = (M_2 L_2 + m_2 l_{g2}) l_1 \ssin_{0,2} \dot{q}_{0,1}^2  
\end{equation}

\begin{equation}
	\mg\ssh_1 = g( M_1 L_1 \ccos_{0,1} + m_1 l_{g1} \ccos_{0,1} + (M_2 + m_2) l_1 \ccos_{0,1} + (M_2 L_2 + m_2 l_{g2}) \ccos_{0,1+2} ) 
\end{equation}

\begin{equation}
	\mg\ssh_2 = g( M_2 L_2 + m_2 l_{g2} ) \ccos_{0,1+2} 
\end{equation}

\item RR balanceado: \\

Escolhendo $L_1$ e $L_2$ de modo que $\mg\ssh = \mzr$:

\begin{equation}
	\begin{cases}
	L_1 = -\frac{ M_2 l_1 + m_1 l_{g1} + m_2 l_2 }{M_1} \\
	L_2 = -\frac{ m_2 l_{g2} }{ M_2 }
	\end{cases}
\end{equation}

Obtemos o seguinte sistema:

\begin{equation}
	\mM\ssh_{1,1} = J_{z1} + J_{z2} + m_1 l_{g2}^2 + l_1^2 (M_2 + m_2) + \frac{m_2^2 l_{g2}^2}{M_2} + \frac{(m_1 l_{g1} + (M_2 + m_2) l_1 )^2 }{M_1}
\end{equation}
\begin{equation}
	\mM\ssh_{1,2} = \mM\ssh_{2,1} = \mM\ssh_{2,2} = J_{z2} + \frac{ m_2 (M_2 + m_2) l_{g2}^2}{M_2}
\end{equation}

\begin{equation}
	\mv\ssh_1 = 0  
\end{equation}
\begin{equation}
	\mv\ssh_2 = 0  
\end{equation}

\begin{equation}
	\mg\ssh_1 = 0
\end{equation}
\begin{equation}
	\mg\ssh_2 = 0
\end{equation}

Ou seja:

\begin{equation}
	\begin{bmatrix}
	J_{z1} + J_{z2} + m_1 l_{g2}^2 + l_1^2 (M_2 + m_2) + \frac{m_2^2 l_{g2}^2}{M_2} + \frac{(m_1 l_{g1} + (M_2 + m_2) l_1 )^2 }{M_1} & J_{z2} + \frac{ m_2 (M_2 + m_2) l_{g2}^2}{M_2} \\
	J_{z2} + \frac{ m_2 (M_2 + m_2) l_{g2}^2}{M_2} & J_{z2} + \frac{ m_2 (M_2 + m_2) l_{g2}^2}{M_2}
	\end{bmatrix}
	\begin{bmatrix}
	\ddot{q}_{n,1} \\
	\ddot{q}_{n,2} \\
	\end{bmatrix}
	=
	\begin{bmatrix}
	u_{n,1} \\
	u_{n,2} \\
	\end{bmatrix}
\end{equation}

\item 5R balanceado: \\

\begin{equation}
	\begin{cases}
	\begin{bmatrix}
	1 & 0 \\
	0 & 1 \\
	\frac{\ccos_{1,1+2}}{l_1 \ssin_{1,2}} &  \frac{\ssin_{1,1+2}}{l_1 \ssin_{1,2}} \\
	-\frac{l_1 \ccos_{1,1} + l_2 \ccos_{1,1+2}}{l_1 l_2 \ssin_{1,2}} &  -\frac{l_1 \ssin_{1,1} + l_2 \ssin_{1,1+2}}{l_1 l_2 \ssin_{1,2}} \\
	\frac{\ccos_{2,1+2}}{l_1 \ssin_{2,2}} &  \frac{\ssin_{2,1+2}}{l_1 \ssin_{2,2}} \\
	-\frac{l_1 \ccos_{2,1} + l_2 \ccos_{2,1+2}}{l_1 l_2 \ssin_{2,2}} &  -\frac{l_1 \ssin_{2,1} + l_2 \ssin_{2,1+2}}{l_1 l_2 \ssin_{2,2}} \\
	\end{bmatrix}^T
	\begin{bmatrix}
	0 & 0 & 0 & 0 & 0 & 0 \\
	0 & 0 & 0 & 0 & 0 & 0 \\
	0 & 0 & \mM\ssh_{1,1} & \mM\ssh_{1,2} & 0 & 0  \\
	0 & 0 & \mM\ssh_{1,2} & \mM\ssh_{1,2} & 0 & 0 \\
	0 & 0 & 0 & 0 & \mM\ssh_{1,1} & \mM\ssh_{1,2}   \\
	0 & 0 & 0 & 0 & \mM\ssh_{1,2} & \mM\ssh_{1,2}  \\
	\end{bmatrix}
	\begin{bmatrix}
	\ddot{q}_{0,1} \\
	\ddot{q}_{0,2} \\
	\ddot{q}_{1,1} \\
	\ddot{q}_{1,2} \\
	\ddot{q}_{2,1} \\
	\ddot{q}_{2,2} \\
	\end{bmatrix}
	=
	\begin{bmatrix}
	\frac{\ccos_{1,1+2}}{l_1 \ssin_{1,2}} &  \frac{\ssin_{1,1+2}}{l_1 \ssin_{1,2}} \\
	\frac{\ccos_{2,1+2}}{l_1 \ssin_{2,2}} &  \frac{\ssin_{2,1+2}}{l_1 \ssin_{2,2}} \\
	\end{bmatrix}
	\begin{bmatrix}
	u_{1,1} \\
	u_{2,1} \\
	\end{bmatrix} \\
	\begin{bmatrix}
	1 & 0 & l_1 \ssin_{1,1} + l_2 \ssin_{1,1+2} & l_2 \ssin_{1,1+2} & 0 & 0 \\
	0 & 1 & -l_1\ccos_{1,1} - l_2 \ccos_{1,1+2} & - l_2 \ccos_{1,1+2} & 0 & 0 \\
	1 & 0 & 0 & 0 & l_1 \ssin_{2,1} + l_2 \ssin_{2,1+2} & l_2 \ssin_{2,1+2}  \\
	0 & 1 & 0 & 0 & -l_1 \ccos_{2,1} - l_2 \ccos_{2,1+2} & - l_2 \ccos_{2,1+2} \\
	\end{bmatrix}
	\begin{bmatrix}
	\ddot{q}_{0,1} \\
	\ddot{q}_{0,2} \\
	\ddot{q}_{1,1} \\
	\ddot{q}_{1,2} \\
	\ddot{q}_{2,1} \\
	\ddot{q}_{2,2} \\
	\end{bmatrix}
	= \\
	- \begin{bmatrix}
	l_1 \ccos_{1,1} \dot{q}_{1,1}^2 + l_2 \ccos_{1,1+2} (\dot{q}_{1,1}+\dot{q}_{1,2})^2 \\
	l_1 \ssin_{1,1} \dot{q}_{1,1}^2 + l_2 \ssin_{1,1+2} (\dot{q}_{1,1}+\dot{q}_{1,2})^2 \\
	l_1 \ccos_{2,1} \dot{q}_{2,1}^2 + l_2 \ccos_{2,1+2} (\dot{q}_{2,1}+\dot{q}_{2,2})^2 \\
	l_1 \ssin_{2,1} \dot{q}_{2,1}^2 + l_2 \ssin_{2,1+2} (\dot{q}_{2,1}+\dot{q}_{2,2})^2 \\
	\end{bmatrix}
	\end{cases}
\end{equation}

\end{itemize}

\subsection{Introduction to sliding modes control}\label{S02-3}

In this subsection, a brief introduction to the sliding modes control will be done. The theme will be explored only to perform second order systems control, without parametric uncertainties, to not escape the scope of the chapter. \\

Consider a dynamical system given by the following differential equation:
\begin{equation} \label{eq:SimpleODE}
\ddot{x} = u
\end{equation}

A curve in the error phase plan, called sliding surface, can be defined:
\begin{equation} \label{eq:SlidingSurface}
s(e, \dot{e}) = - (\dot{e} + \lambda e) = 0, \, \lambda > 0
\end{equation}

Being $e = x^\sdia - x$ the error signal and $x^\sdia$ reference signal. Note that if the system is on the sliding surface, we have:
\begin{equation} \label{eq:SlidingError}
\dot{e} + \lambda e = 0 \Rightarrow e(t) = c \, \mathsf{e}^{- \lambda t}
\end{equation}

Thus, the error drops exponentially to zero, with time constant $1/\lambda$.

To find a control law that brings the system to the sliding surface, we start from the definition of $s$: \\

$ s = -(\dot{e} + \lambda e) $ \\

Differentiating with respect to time:

\begin{equation} \label{eq:dotS}
\dot{s} =  -(\ddot{e} + \lambda \dot{e}) = \ddot{x} - \ddot{x}^\sdia - \lambda \dot{e} 
\end{equation}

Substituting \eqref{eq:SimpleODE} into \eqref{eq:dotS}:
\begin{equation} \label{dotS2}
\dot{s} = u - \ddot{x}^\sdia - \lambda \dot{e}
\end{equation}

Using the following control law:
\begin{equation} \label{SMControlLaw1D}
u = \ddot{x}^\sdia + \lambda \dot{e} - k \sign (s), \, k>0
\end{equation}

He have:
\begin{equation} \label{CloserLoop1D}
\dot{s} = -k \sign(s) 
\end{equation}

Suppose that the system starts at $s(0) = s_0 >0$. Solving the ODE for $s>0$:

$$ \dot{s} = -k \Rightarrow s = -k t + c $$
$$ s(0) = s_0 \Rightarrow c = s_0 $$
$$ \therefore s = s_0 - k t, \, s>0 $$

According to the solution found, when $t \rightarrow t_s = \frac{|s_0|}{k}$, $s \rightarrow 0 $. Solving the ODE for $s(t_s) = 0$:

$$ \dot{s} = 0 \Rightarrow s =  c $$
$$ s(t_s) = 0 \Rightarrow c = 0 $$

Therefore, the solution of the ODE for $s(0) = s_0 > 0$ is given by:
\begin{equation} \label{eq:SM-ODE-Sol1}
s(t) =
\begin{cases}
s_0 - k t, \, t < t_s \\
0, \,\,\,\,\,\,\,\,\,\,\,\,\,\,\,\, t \geq t_s \\
\end{cases}
\end{equation}

An analogous result is found solving the ODE for $s(0) = s_0 < 0$:

\begin{equation} \label{eq:SM-ODE-Sol2}
s(t) =
\begin{cases}
s_0 + k t, \, t < t_s \\
0, \,\,\,\,\,\,\,\,\,\,\,\,\,\,\,\, t \geq t_s \\
\end{cases}
\end{equation}

Thus, it can be concluded that the ODE \eqref{CloserLoop1D} converges to $s=0$, regardless of the initial condition.
Therefore, we have that the control law \eqref{SMControlLaw1D} makes the system represented by \eqref{eq:SimpleODE} follow the reference signal, because the error signal converges to zero.



\subsection{Extended sliding modes control techniques}\label{S02-4}

As seen in subsections 2.1 and 2.2, it's very convenient to use redundant coordinates to perform parallel mechanism dynamic modeling. 
The aim of this subsection it to propose a control law for systems described by redundant coordinates.

Let $\ssM$ be a multibody mechanical system whose mathematical model is given 
by equations~(\ref{eq:02-105B},~\ref{eq:02-112B}).
For the sake of brevity, no system index $n$ will be used in this subsection.
Suppose that each $\mf$ is an affine function of the control inputs $u_{k}$
in which the coefficients of the $u_{k}$ may depend on the instantaneous
configuration of the system.
Suppose additionally that that all the $\mA$ are independent
of the quasi-velocities $p_{j}$ and all the $\mM$, $\mg$, $\mA$ and $\mb$ 
are independent of the $u_{k}$.
Under these conditions, matrices $\mC$ will not depend on 
any quasi-velocity, $\md$ can be expressed as an affine function of the control inputs 
and $\mc$ is independent of them.
Considering that the number of control inputs in $\ssM$ is
exactly equal to the number of degrees of freedom of $\ssM$, 
there exists a particular matrix $\mC(t,\mq)$ such that:
\begin{equation} \label{eq:InverseDynamics}
\mu = \mC^\msT(t,\mq) \Big( \mM (t, \mq) \, \dot{\mp} + \mw(t,\mq,\mp) + \mz(t,\mq) \Big)
\end{equation}
In equation \eqref{eq:InverseDynamics}, $\mw$ is a column-vector representing terms of generalized force 
or generalized gyroscopic inertia force which are linear or bilinear with respect to quasi-velocities 
and $\mz$ stems from terms that are independent of these variables.

From the control perspective, it is convenient to work with mathematical models in which $\mp = \dot \mq$, 
in order to have a position feedback control.
Thus, based on equations \eqref{eq:InverseDynamics} and \eqref{eq:02-201A}, consider that 
the mathematical model of $\ssM$ is given by the following equations:
\begin{equation} \label{eq:MechanicalSystem}
\begin{cases}
\mC^\msT (t, \mq) \Big( \mM (t, \mq) \, \ddot{\mq} + \mw (t, \mq, \dot{\mq}) + \mz (t, \mq) \Big) = \mu \\
\mA (t, \mq) \, \ddot{\mq} + \mb (t, \mq, \dot{\mq}) = \mzr
\end{cases}
\end{equation}
Rewriting in a compact matrix form:
\begin{equation} \label{eq:MechanicalSystemMatrix}
\begin{bmatrix}
\mC^\msT \mM \\
\mA
\end{bmatrix}
\ddot{\mq}
=
\begin{bmatrix}
\mu - \mC^\msT(\mw + \mz) \\
-\mb
\end{bmatrix}
\end{equation}
The desired control law should satisfy, in closed loop, the condition $ \ddot{\mq} = \mv $, with $\mv$ being 
a control input column-matrix. 
Thus, the following control law should be used:
\begin{equation} \label{eq:ControlLawV}
\mu = \mC^\msT ( \mM \, \mv + \mw + \mz )
\end{equation}
Once $ \ddot{\mq} = \mv $, and $\ddot{\mq}$ has to satisfy constraint equations, $\mv$ must respect the same restrictions, i.e.:
\begin{equation} \label{eq:ControlLawVRestriction}
\mA \, \mv + \mb = \mzr
\end{equation}
Applying the control law \eqref{eq:ControlLawV} and the 
rectritions \eqref{eq:ControlLawVRestriction} in \eqref{eq:MechanicalSystemMatrix}: 
\begin{align*}
&	\begin{bmatrix}
	\mC^\msT \, \mM \\
	\mA
	\end{bmatrix}
	\ddot{\mq}
	=
	\begin{bmatrix}
	\mC^\msT ( \mM \, \mv + \mw + \mz ) - \mC^\msT(\mw + \mz) \\
	\mA \, \mv
	\end{bmatrix}
	=
	\begin{bmatrix}
	\mC^\msT  \, \mM \, \mv \\
	\mA \, \mv
	\end{bmatrix}
	=
	\begin{bmatrix}
	\mC^\msT \mM \\
	\mA
	\end{bmatrix}
	\mv 
\end{align*}
Once the matrix $\begin{bmatrix} \mC^\msT \, \mM \\ \mA \end{bmatrix}$ is non-singular:
\begin{equation} \label{eq:ClosedLoopV}
\ddot{\mq} = \mv
\end{equation}

Let $\mv'$ be given by the sliding modes control law:
\begin{equation} \label{eq:SMControlowLasV1'}
\mv' = \ddot{\mq}^\sdia + \lambda \dot{\me} + k \sign (\dot{\me} + \lambda \me)
\end{equation}
Being $ \me = \mq^\sdia - \mq $ the error signal and $\mq^\sdia$ the reference signal. If there were no retrictions, it could be stated that $ \mv = \mv' $, and:
\begin{align*}
	\ddot{\mq} = \mv \qquad \Rightarrow \qquad \ddot{\me} + \lambda \dot{\me} + k \sign (\dot{\me} + \lambda \me) = \mzr 
	\qquad \Leftrightarrow \qquad \dot{\ms} = - k \sign(\ms)
\end{align*}
This would ensure that $\me \rightarrow 0$ when $t \rightarrow \infty$ for any initial condition, as seen in the last subsection. 

Once $\mv$ can not be freely set as $\mv'$, the following optimization problem is proposed:
\begin{align} \label{eq:Optimization}
& 	\min_{\mv} \quad (\mv - \mv')^\msT \, \mM \, (\mv - \mv') 
	\qquad \text{s.t.} \qquad \mA \, \mv + \mb = \mzr
\end{align}
As $\mM$ is positive-semidefinite, then $(\mv - \mv')^\msT \, \mM \, (\mv - \mv') \geq 0 $ for any $\mv$.

Applying the method of Lagrange undetermined multipliers, it can be stated that this
optimization problem is equivalent to minimize with respect to $\mv$ and $\mlambda$
the following functional:
\begin{equation}
L = (\mv - \mv')^\msT \, \mM \, (\mv - \mv') + (\mA \mv + \mb)^\msT \, \mlambda 
\end{equation}
To solve the problem, the Lagrangian function must be stationary. Thus:
\begin{align}
 	\dl L = 0 \qquad &\Rightarrow \qquad  
 	\dl \mv^\msT \mM (\mv - \mv') + (\mv - \mv')^\msT \mM \dl \mv + (\mA \dl \mv)^\msT \mlambda 
		+ (\mA \mv + \mb)^\msT \dl \mlambda = 0 
	\nonumber \\	
& 	\Rightarrow \qquad
	\dl \mv^\msT \left( (\mM + \mM^\msT)(\mv - \mv') + \mA^\msT \mlambda \right) + \dl \mlambda^\msT (\mA \mv + \mb) = 0 
\end{align}
Once $\mM$ is symmetric and $\dl \mv$ and $\dl \mlambda$ are arbitrary:
\begin{equation} \label{eq:OptimizationSol}
\begin{cases}
2 \, \mM \, (\mv - \mv') + \mA^\msT \mlambda = \mzr \\
\mA \, \mv + \mb = \mzr
\end{cases}
\end{equation}
Considering that $\mC$ is an orthogonal complement of $\mA$, pre-multiplying the first equation of 
\eqref{eq:OptimizationSol} by $\mC^\msT$ leads to:
\begin{align}
& 	2 \, \mC^\msT \, \mM (\mv - \mv') + \mC^\msT \, \mA^\msT \mlambda = \mzr 
	\qquad 	\Rightarrow \qquad \mC^\msT \, \mM (\mv - \mv')  = \mzr 
	\qquad \therefore \qquad \mC^\msT \, \mM \, \mv  = \mC^\msT \, \mM \, \mv'
	\label{eq:OptimizationSol2}
\end{align}
Thus, the control law that makes the closed loop system as close as possible of $\ddot{\mq} = \mv'$, 
according to the optimization criterion adopted, is:
\begin{equation} \label{eq:ControlLawFinal}
\mu = \mC^\msT ( \mM \, \mv' + \mw + \mz )
\end{equation}

\begin{figure}[H]
	\centering
	\includegraphics[scale=0.75]{ControlDiagram.png}
	\caption{Control block diagram}
	\label{ControlDiagram}
\end{figure}




%--------------------BALANCING--------------------%

\input{SECTIONS/RENATO/3-1}

 \subsection{Control}\label{S03-2}
 
\noindent {\bf Controle por modos deslizantes}\\

Seja um sistema dinâmico dado pela seguinte equação diferencial:

$$ \ddot{x} = u $$

Definimos a seguinte superfície, chamada de superfície de escorregamento:

$$ s(e, \dot{e}) = - (\dot{e} + \lambda e) = 0, \, \lambda > 0 $$

Sendo $e = x_d - x$ o erro de controle e $x_d$ o sinal de referência. Repare que se o sistema estiver na superfície de escorregamento, temos:

$$ \dot{e} + \lambda e = 0 \Rightarrow e(t) = C e^{- \lambda t} $$

Sendo assim, o erro cai exponencialmente para zero, com constante de tempo $\lambda$.

Para encontrar a lei de controle que leva o sistema à superficie de escorregamento, parte-se da definição de $s$:

$$ s = -(\dot{e} + \lambda e) $$

Derivando no tempo:

$$ \dot{s} =  -(\ddot{e} + \lambda \dot{e}) = \ddot{x} - \ddot{x}_d - \lambda \dot{e} $$

Substituindo a equação dinâmica:

$$ \dot{s} = u - \ddot{x}_d - \lambda \dot{e} $$

Utizando a seguinte lei de controle:

$$ u = \ddot{x}_d + \lambda \dot{e} - k \sign (s), \, k>0$$

Temos:

$$ \dot{s} = -k \sign(s) $$


 
Modelo do mecanismo:

$$
\begin{cases}
\mC_n^\top (\mq_n) \Big( \mM_n (\mq_n) \ddot{\mq}_n + \mw_n (\mq_n, \dot{\mq}_n) + \mz_n (\mq_n) \Big) = \mu_n \\
\mA_n \ddot{\mq}_n = - \mb_n (\mq_n, \dot{\mq}_n)
\end{cases}
$$

De maneira matricial compacta:

$$
\begin{bmatrix}
\mC_n^\top \mM_n \\
\mA_n
\end{bmatrix}
\ddot{\mq}_n
=
\begin{bmatrix}
\mu_n - \mC_n^\top(\mw_n + \mz_n) \\
-\mb_n
\end{bmatrix}
$$

Gostaria que $ \ddot{\mq}_n = \mv_n $, sendo $\mv'_n$ uma entrada de controle. Para que isso aconte\c{c}a, utilizamos a seguinte lei de controle:

$$ \mu_n = \mC_n^\top ( \mM_n \mv_n + \mw_n + \mz_n ) $$

Como queremos que $ \ddot{\mq}_n = \mv_n $ e $\ddot{\mq}_n$ tem restri\c{c}\~oes, $\mv_n$ deve respeitar as mesmas resti\c{c}\~oes, ou seja:

$$ \mA_n \mv_n = -\mb_n $$

Aplicando a lei de controle, temos:

$$
\begin{bmatrix}
\mC_n^\top \mM_n \\
\mA_n
\end{bmatrix}
\ddot{\mq}_n
=
\begin{bmatrix}
\mC_n^\top ( \mM_n \mv_n + \mw_n + \mz_n ) - \mC_n^\top(\mw_n + \mz_n) \\
\mA_n \mv_n
\end{bmatrix}
=
\begin{bmatrix}
\mC_n^\top  \mM_n \mv_n \\
\mA_n \mv_n
\end{bmatrix}
=
\begin{bmatrix}
\mC_n^\top \mM_n \\
\mA_n
\end{bmatrix}
\mv_n
$$
$$
\therefore \ddot{\mq}_n = \mv_n
$$
Seja $\mv'_n$ a lei de controle por modos deslizantes:
$$ \mv'_n = \ddot{\mq}_{n, d} + \lambda \dot{\me}_n + k \sign (\dot{\me}_n + \lambda \me_n) $$
Sendo $ \me_n = \mq_{n,d} - \mq_n $. Se n\~ao houvesse restri\c{c}\~oes, poderiamos fazer $ \mv_n = \mv'_n $ :
$$ \ddot{\mq}_n = \mv_n \Rightarrow  \ddot{\me}_n + \lambda \dot{\me}_n + k \sign (\dot{\me}_n + \lambda \me_n) = \mzr$$
Isso garantiria que $\me_n \rightarrow 0$ quando $t \rightarrow \infty$ para quaisquer condi\c{c}\~oes iniciais. \\

Como temos restri\c{c}\~oes em $\mv_n$, procuramos $\mv_n$ mais pr\'oximo poss\'ivel de $\mv'_n$ atraves da solu\c{c}\~ao do seguinte problema de otimiza\c{c}\~ao:

\begin{center}
$\begin{aligned}
& \underset{\mv_n}{\text{Min}}
& & (\mv_n - \mv'_n)^\top \mM_n (\mv_n - \mv'_n) \\
& \text{tal que}
& & \mA_n \mv_n + \mb_n = \mzr
\end{aligned}$
\end{center}

Como $\mM_n$ \'e n\~ao-negativa definida, temos que $(\mv_n - \mv'_n)^\top \mM_n (\mv_n - \mv'_n) \geq 0 $ para qualquer valor de $\mv_n$.

Aplicando a ténica dos multiplicadores de Lagrange, pode-se dizer que o seguinte problema é equivalente:

\begin{center}
$\begin{aligned}
& \underset{\mv_n, \mlambda}{\text{Min}}
& & \llL = (\mv_n - \mv'_n)^\top \mM_n (\mv_n - \mv'_n) + (\mA_n \mv_n + \mb_n)^\top \mlambda \\
\end{aligned}$
\end{center}

Para solucionar o problema, impõe-se a estacionariedade da função lagrangeana:

$$ \dl \llL = 0 \Rightarrow \dl \mv_n^\top \mM_n (\mv_n - \mv'_n) + (\mv_n - \mv'_n)^\top \mM_n \dl \mv_n + (\mA_n \dl \mv_n)^\top \mlambda + (\mA_n \mv_n + \mb_n)^\top \dl \mlambda = 0 $$
$$ \Rightarrow \dl \mv_n^\top \Big( (\mM_n + \mM_n^\top)(\mv_n - \mv'_n) + \mA_n^\top \mlambda \Big) + \dl \mlambda^\top (\mA_n \mv_n + \mb_n) = 0 $$

Como $\mM_n$ é simétrica e $\dl \mv_n$ e $\dl \mlambda$ são arbitr\'arios, temos:

$$
\begin{cases}
2 \mM_n (\mv_n - \mv'_n) + \mA_n^\top \mlambda = \mzr \\
\mA_n \mv_n + \mb_n = \mzr
\end{cases}
$$

Como $\mC_n$ é o complemento ortogonal de $\mA_n$, multiplicando a primeira equação por $\mC_n^\top$, temos:

$$ 2 \mC_n^\top \mM_n (\mv_n - \mv'_n) + \mC_n^\top \mA_n^\top \mlambda = \mzr $$
$$ \Rightarrow  \mC_n^\top \mM_n (\mv_n - \mv'_n)  = \mzr $$
$$ \therefore \mC_n^\top \mM_n \mv_n  = \mC_n^\top \mM_n  \mv'_n $$

Sendo assim, temos que a lei de controle que torna o sistema em malha fechado o mais pr\'oximo poss\'ivel, segundo o crit\'erio de otimiza\c{c}\~ao adotado, de $\ddot{\mq}_n = \mv'$ \'e:

$$ \mu_n = \mC_n^\top ( \mM_n \mv'_n + \mw_n + \mz_n ) $$

\newpage



$$
\begin{bmatrix}
\mC^T \mM \\
\mA
\end{bmatrix}
\ddot{\mq}
=
\begin{bmatrix}
\mathbb{\zeta} \mu \\
-\mb
\end{bmatrix}
$$



$$ \mu = \mathbb{\zeta}^{-1} \mC^T \mM \mu' $$
$$ \mu' = \ddot{\mq}_d + \lambda \dot{\me} - k  \sign (\ms) $$

$$ \ms = - \dot{\me} - \lambda \me $$
$$ \dot{\ms} = - \ddot{\me} - \lambda \dot{\me} = \ddot{\mq} - \ddot{\mq}_d  - \lambda \dot{\me} $$
$$ \dot{\ms} =  \begin{bmatrix}
\mC^T \mM \\
\mA
\end{bmatrix}^{-1}
\begin{bmatrix}
\mathbb{\zeta} \mu \\
-\mb
\end{bmatrix}
 - \ddot{\mq}_d  - \lambda \dot{\me} $$
 
Aplicando a lei de controle:

$$ \dot{\ms} =  \begin{bmatrix}
\mC^T \mM \\
\mA
\end{bmatrix}^{-1}
\begin{bmatrix}
 \mC^T \mM (\ddot{\mq}_d + \lambda \dot{\me} - k  \sign (\ms)) \\
-\mb
\end{bmatrix}
 - \ddot{\mq}_d  - \lambda \dot{\me} $$
 
 $$ \dot{\ms} =  \begin{bmatrix}
\mC^T \mM \\
\mA
\end{bmatrix}^{-1}
\begin{bmatrix}
 \mC^T \mM (\ddot{\mq}_d + \lambda \dot{\me} - k  \sign (\ms)) -  \mC^T \mM(\ddot{\mq}_d  + \lambda \dot{\me}) \\
-\mb - \mA(\ddot{\mq}_d  + \lambda \dot{\me})
\end{bmatrix} $$

 $$ \dot{\ms} =  \begin{bmatrix}
\mC^T \mM \\
\mA
\end{bmatrix}^{-1}
\begin{bmatrix}
- \mC^T \mM  k  \sign (\ms) \\
-\mb - \mA(\ddot{\mq}_d  + \lambda \dot{\me})
\end{bmatrix} $$

Definindo:
$$\begin{bmatrix}
\mC^T \mM \\
\mA
\end{bmatrix}^{-1}
=
\begin{bmatrix}
(\mC^T \mM)^\dagger & \mA^\dagger
\end{bmatrix} $$

Temos:

$$\dot{\ms} = 
- (\mC^T \mM)^\dagger \mC^T \mM  k  \sign (\ms) - \mA^\dagger\mb - \mA^\dagger \mA(\ddot{\mq}_d  + \lambda \dot{\me}) $$

Sendo assim, se a seguinte inequação for respeitada para pelo menos $\nu\ssh$ componentes de $\dot{\ms}$, o erro vai a zero:

$$\dot{\ms} = 
- (\mC^T \mM)^\dagger \mC^T \mM  k  \sign (\ms) - \mA^\dagger\mb - \mA^\dagger \mA(\ddot{\mq}_d  + \lambda \dot{\me}) \leq \mzr $$


%--------------------EXAMPLES--------------------%

% \section{Ex}\label{S04}

% \subsection{5R}\label{S04-1}

% \subsection{Delta}\label{S04-2}

\input{SECTIONS/RENATO/4-3}

\subsection{Inverse Dynamics and Control Simulations}\label{S04-4}

Dada a lei de controle apresentada na sub-se\c{c}\~ao 2.4 e o modelo din\^amico apresentado na sub-se\c{c}\~ao anterior, nesta sub-se\c{c}\~ao realizaremos simula\c{c}\~oes da din\^amica inversa e da aplica\c{c}\~ao da lei de controle no mecanismo 5R balanceado e desbalancedo. Para tanto, \'e necess\'ario definir valores para os par\^ametros do mecanismos, as condi\c{c}\~oes iniciais do sistema, trajetórias de referência e os par\^ametros do controlador.

Assim, definimos:

\begin{itemize}
\item Par\^ametros físicos fixos dos mecanismos RR utilizados para formar o 5R:

\begin{itemize}
\begin{multicols}{2}
\item $a_{1,1} = 0.1 \, m$
\item $a_{1,2} = 0.2 \, m$
\item $m_{1, \llB_1} = 0.2 \, kg$
\item $m_{1, \llB_2} = 0.4 \, kg$
\item $m_{2, \llB_1} = 1.5 \, kg$
\item $I_{1, \llB_1} = 67 \cdot 10^{-5} \, kg \, m^2$
\item $I_{1, \llB_2} = 134 \cdot 10^{-5} \, kg \, m^2$
\item $\gamma_{1, \llB_1} = \gamma_{1, \llB_2} = 0.5$
\end{multicols}
\end{itemize}

\item Par\^ametros físicos ajustáveis dos RR balanceados:

\begin{itemize}
\begin{multicols}{2}
\item $\gamma_{3, \llB_1} = 3.0$
\item $\gamma_{4, \llB_1} = 1.7$
\item $m_{3, \llB_1} = m_{4, \llB_1} = 1.0 \, kg$ 
\end{multicols}
\end{itemize}

\item Par\^ametros físicos ajustáveis dos RR desbalanceados:

\begin{itemize}
\item $m_{3, \llB_1} = m_{4, \llB_1} = 0$
\end{itemize}


\item Posicionamento das bases dos mecanismos RR para formar o 5R:

\begin{itemize}
\begin{multicols}{2}
\item $q_{1,\ttp_1,1} = -0.1 \, m$
\item $q_{1,\ttp_1,2} = 0 $
\item $q_{2,\ttp_1,1} = 0.1 \, m$
\item $q_{2,\ttp_1,2} = 0 $
\end{multicols}
\end{itemize}

\item Par\^ametros do controlador utilizado:

\begin{itemize}
\begin{multicols}{2}
\item $\lambda = 40$
\item $k = 10$
\end{multicols}
\end{itemize}

\item Condi\c{c}\~oes iniciais de posi\c{c}\~ao para o mecanismo 5R:

$$\begin{cases}
q_{1,\ttp_3,1}(0) = q_{2,\ttp_3,1}(0) = 0 \\
q_{1,\ttp_3,2}(0) = q_{2,\ttp_3,2}(0) = 0.02 \,m \\
q_{3, \llR_3}(0) = 173.282^\circ \\
q_{1,\llR_1}(0) = 175.249^\circ \\
q_{1,\llR_2}(0) = 188.11^\circ \\
q_{2,\llR_1}(0) = 4.75078^\circ \\
q_{2,\llR_2}(0) = 171.89^\circ \\
\end{cases}$$

\item Trajet\'oria de refer\^encia 1:

$$ \begin{cases}
q_{1,\ttp_3,1}^\sdia(t) = q_{2,\ttp_3,1}^\sdia(t) = 0 \\
q_{1,\ttp_3,2}^\sdia(t) = q_{2,\ttp_3,2}^\sdia(t) = 0.02 + 0.22 \Big( \frac{t}{5} - \frac{1}{2\pi} \sin \big( \frac{2\pi t}{5} \big) \Big) \\
\end{cases}$$

\item Trajet\'oria de refer\^encia 2:

$$ \begin{cases}
q_{1,\ttp_3,1}^\sdia(t) = q_{2,\ttp_3,1}^\sdia(t) = 0.005 \sin(7t) \\
q_{1,\ttp_3,2}^\sdia(t) = q_{2,\ttp_3,2}^\sdia(t) = 0.14 - 0.12 \cos(7t) \\
\end{cases}$$

\end{itemize}

Foram explicitadas apenas algumas coordenadas das trajet\'orias de refer\^encia, pois, definindo estas, as outras podem ser encontradas num\'erica ou analiticamente utilizando os v\'inculos de posi\c{c}\~ao e a condi\c{c}\~ao de montagem do mecanismo.

Para as simula\c{c}\~oes da lei de controle, a fun\c{c}\~ao $y(x) = \sign(x)$ foi substituida pela fun\c{c}\~ao $y_{sat}(x) = \frac{2}{\pi}\arctan(159.1 x)$, a qual apresenta as seguintes propriedades: $y_{sat}(0.2) = -y_{sat}(0.2) = 0.98$ e $y_{sat}(\infty) = - y_{sat}(-\infty) = 1$. Sua utiliza\c{c}\~ao torna muito mais eficiente as simula\c{c}\~oes num\'ericas, evita o chattering nos atuadores e ainda garante um erro em regime permanente desprez\'ivel para esta aplica\c{c}\~ao.

\begin{figure}[H]
	\centering
	\includegraphics[scale=0.5]{Sat.pdf}
	\caption{Fun\c{c}\~ao de satura\c{c}\~ao}
	\label{Sat}
\end{figure}

Na simula\c{c}\~oes da din\^amica inversa, calculam-se os esfor\c{c}os dos atuadores necess\'arios para que o mecanismo siga exatamente a traje\'oria de refer\^encia, ignorando as condi\c{c}\~oes iniciais de velocidade que ser\~ao definidas.

\begin{itemize}
\item[A)] Simula\c{c}\~ao da trajet\'oria 1: \\

Na simula\c{c}\~ao lei de controle sup\~oe-se as seguintes condi\c{c}\~oes iniciais de velocidades:

$$\begin{cases}
\dot{q}_{1,\ttp_3,1}(0) = \dot{q}_{2,\ttp_3,1}(0) = 0 \\
\dot{q}_{1,\ttp_3,2}(0) = \dot{q}_{2,\ttp_3,2}(0) = 1 \,m/s \\
\dot{q}_{3, \llR_3}(0) = -5.871 rad/s \\
\dot{q}_{1,\llR_1}(0) = -4.153 rad/s \\
\dot{q}_{1,\llR_2}(0) = 7.089 rad/s \\
\dot{q}_{2,\llR_1}(0) = 4.153 rad/s \\
\dot{q}_{2,\llR_2}(0) = -7.089 rad/s \\
\end{cases}$$

Isso faz com que haja um erro de velocidade n\~ao nulo para $t=0$ e torna mais interessante a anal\'ise da din\^amica dos erros de posi\c{c}\~ao e velocidade.

Aqui seguem os gr\'aficos da trajet\'oria de refer\^encia 1 para algumas coordenadas:

\begin{figure}[ht]
\centering
\begin{minipage}[b]{0.45\linewidth}
\includegraphics[scale=0.5]{AnglesA.pdf}
\includegraphics[scale=0.5]{AnglesAsub.pdf}
\label{fig:AnglesA}
\end{minipage}
\quad
\begin{minipage}[b]{0.45\linewidth}
\includegraphics[scale=0.5]{XA.pdf}
\includegraphics[scale=0.5]{XAsub.pdf}
\label{fig:XA}
\end{minipage}
\caption{Trajet\'oria de refer\^encia 1}
\end{figure}

\begin{itemize}
\item[A.1)] Mecanismo balanceado \\

Simula\c{c}\~ao dos esfor\c{c}os aplicados pelos atuadores:

\begin{figure}[H]
\centering
\begin{minipage}[b]{0.45\linewidth}
\includegraphics[scale=0.5]{InverseDynamicsA1.pdf}
\includegraphics[scale=0.5]{InverseDynamicsA1sub.pdf}
\label{fig:InverseDynamicsA1}
\caption{Inverse dynamics simulation}
\end{minipage}
\quad
\begin{minipage}[b]{0.45\linewidth}
\includegraphics[scale=0.5]{ControlA1.pdf}
\includegraphics[scale=0.5]{ControlA1sub.pdf}
\label{fig:ControlA1}
\caption{Control inputs}
\end{minipage}
\end{figure}

Din\^amica do erro de controle:

\begin{figure}[H]
\centering
\subfigure[Position error norm]{%
\includegraphics[scale=0.5]{PositionErrorA1.pdf}
\label{fig:PositionErrorA1}}
\quad
\subfigure[Velocity error norm]{%
\includegraphics[scale=0.5]{VelocityErrorA1.pdf}
\label{fig:VelocityErrorA1}}
\subfigure[$||\ms|| = || \dot{\me} + \lambda \me ||$]{%
\includegraphics[scale=0.5]{sA1.pdf}
\label{fig:sA1}}
%
\caption{Error dynamics}
\label{fig:figure}
\end{figure}

\item[A.2)] Mecanismo desbalanceado \\

Simula\c{c}\~ao dos esfor\c{c}os aplicados pelos atuadores:

\begin{figure}[H]
\centering
\begin{minipage}[b]{0.45\linewidth}
\includegraphics[scale=0.5]{InverseDynamicsA2.pdf}
\includegraphics[scale=0.5]{InverseDynamicsA1sub.pdf}
\label{fig:InverseDynamicsA2}
\caption{Inverse dynamics simulation}
\end{minipage}
\quad
\begin{minipage}[b]{0.45\linewidth}
\includegraphics[scale=0.5]{ControlA2.pdf}
\includegraphics[scale=0.5]{ControlA1sub.pdf}
\label{fig:ControlA2}
\caption{Control inputs}
\end{minipage}
\end{figure}

Din\^amica do erro de controle:

\begin{figure}[H]
\centering
\subfigure[Position error norm]{%
\includegraphics[scale=0.5]{PositionErrorA2.pdf}
\label{fig:PositionErrorA2}}
\quad
\subfigure[Velocity error norm]{%
\includegraphics[scale=0.5]{VelocityErrorA2.pdf}
\label{fig:VelocityErrorA2}}
\subfigure[$||\ms|| = || \dot{\me} + \lambda \me ||$]{%
\includegraphics[scale=0.5]{sA2.pdf}
\label{fig:sA2}}
%
\caption{Error dynamics}
\label{fig:figure}
\end{figure}


\end{itemize}


\item[B)] Simula\c{c}\~ao da trajet\'oria 1: \\

Na simula\c{c}\~ao lei de controle sup\~oe-se condi\c{c}\~oes iniciais de velocidades nulas.

Aqui seguem os gr\'aficos da trajet\'oria de refer\^encia 2 para algumas coordenadas:

\begin{figure}[ht]
\centering
\begin{minipage}[b]{0.45\linewidth}
\includegraphics[scale=0.5]{AnglesA.pdf}
\includegraphics[scale=0.5]{AnglesAsub.pdf}
\label{fig:AnglesB}
\end{minipage}
\quad
\begin{minipage}[b]{0.45\linewidth}
\includegraphics[scale=0.5]{XA.pdf}
\includegraphics[scale=0.5]{XAsub.pdf}
\label{fig:XB}
\end{minipage}
\caption{Trajet\'oria de refer\^encia 1}
\end{figure}

\begin{itemize}
\item[B.1)] Mecanismo balanceado \\

Simula\c{c}\~ao dos esfor\c{c}os aplicados pelos atuadores:

\begin{figure}[H]
\centering
\begin{minipage}[b]{0.45\linewidth}
\includegraphics[scale=0.5]{InverseDynamicsB1.pdf}
\includegraphics[scale=0.5]{InverseDynamicsA1sub.pdf}
\label{fig:InverseDynamicsB1}
\caption{Inverse dynamics simulation}
\end{minipage}
\quad
\begin{minipage}[b]{0.45\linewidth}
\includegraphics[scale=0.5]{ControlB1.pdf}
\includegraphics[scale=0.5]{ControlA1sub.pdf}
\label{fig:ControlB1}
\caption{Control inputs}
\end{minipage}
\end{figure}

Din\^amica do erro de controle:

\begin{figure}[H]
\centering
\subfigure[Position error norm]{%
\includegraphics[scale=0.5]{PositionErrorB1.pdf}
\label{fig:PositionErrorB1}}
\quad
\subfigure[Velocity error norm]{%
\includegraphics[scale=0.5]{VelocityErrorB1.pdf}
\label{fig:VelocityErrorB1}}
\subfigure[$||\ms|| = || \dot{\me} + \lambda \me ||$]{%
\includegraphics[scale=0.5]{sB1.pdf}
\label{fig:sA1}}
%
\caption{Error dynamics}
\label{fig:figure}
\end{figure}

\item[B.2)] Mecanismo desbalanceado \\

Simula\c{c}\~ao dos esfor\c{c}os aplicados pelos atuadores:

\begin{figure}[H]
\centering
\begin{minipage}[b]{0.45\linewidth}
\includegraphics[scale=0.5]{InverseDynamicsB2.pdf}
\includegraphics[scale=0.5]{InverseDynamicsA1sub.pdf}
\label{fig:InverseDynamicsB2}
\caption{Inverse dynamics simulation}
\end{minipage}
\quad
\begin{minipage}[b]{0.45\linewidth}
\includegraphics[scale=0.5]{ControlB2.pdf}
\includegraphics[scale=0.5]{ControlA1sub.pdf}
\label{fig:ControlB2}
\caption{Control inputs}
\end{minipage}
\end{figure}

Din\^amica do erro de controle:

\begin{figure}[H]
\centering
\subfigure[Position error norm]{%
\includegraphics[scale=0.37]{PositionErrorB2.pdf}
\label{fig:PositionErrorB2}}
\quad
\subfigure[Velocity error norm]{%
\includegraphics[scale=0.37]{VelocityErrorB2.pdf}
\label{fig:VelocityErrorB2}}
\subfigure[$||\ms|| = || \dot{\me} + \lambda \me ||$]{%
\includegraphics[scale=0.37]{sB2.pdf}
\label{fig:sB2}}
%
\caption{Error dynamics}
\label{fig:figure}
\end{figure}


\end{itemize}


\end{itemize}







%--------------------CONCLUSIONS--------------------%

%\newpage

% \section{Conclusions}\label{S05}


%--------------------ACKNOWLEDGMENTS--------------------%

\section*{Acknowledgments}


%--------------------BIBLIOGRAPHY--------------------%

% \newpage
\phantomsection 
\bibliographystyle{abbrv}
\bibliography{EXTRAS/bibpapers}


% \end{multicols}


%\newpage

Notation tests:
\begin{align*}
&	\vm_{\llB \rl \ttb^\star} = \vI_{\llB \rl \ttb^\star} \cdot \dot \vomega_{\llB \rl \llN} 
		+ \vomega_{\llB \rl \llN}  \times (\vI_{\llB \rl \ttb^\star} \cdot \vomega_{\llB \rl \llN}) 
	\\
& 	\vf_{\llB} = m_{\llB} \, \va_{\ttb^\star \rl \llN}
	\\
& 	\dot p_j = C_{jk} \, \pi_k 
	\\
& 	\dot \mp = \mC \, \mpi
	\\
& 	\nvct{\vv_{\ttp_1 \rl \llE}}_{\ttE'}
		= \nvct{\dot \vr_{\ttp_1 \rl \ttp_0}}_{\ttE'} + \nvct{\vv_{\ttp_0 \rl \llE}}_{\ttE'}
	\\
&	\nvct{\vv_{\ttp_1 \rl \llE_0}}_{\ttE_2} = \nmat{\vone}_{\ttE_2 \rl \ttE_1}\nvct{\vv_{\ttp_1 \rl \llE_0}}_{\ttE_1} 
	\\
& 	\mR = \nmat{\vone}_{\ttE_2 \rl \ttE_1} 
	\\
& 	\nsmat{\vomega_{\llB \rl \llN}}_{\ttB \rl \ttB} =\nmat{\vone}_{\ttB \rl \ttN}  
		\left( \frac{\dd}{\dd t}\nmat{\vone}_{\ttN \rl \ttB} \right) \\
& 	\nmat{\vone}_{\ttE_3 \rl \ttN} 
		= \mat{
		\begin{array}{ccc}
			\cos \theta & -\sin \theta & 0 \\
			\sin \theta & \cos \theta & 0 \\
			0 & 0 & 1
		\end{array}}
		= \mat{
		\begin{array}{ccc}
			\ccos_\theta & -\ssin_\theta & 0 \\
			\ssin_\theta & \ccos_\theta & 0 \\
			0 & 0 & 1
		\end{array}}
		= \mat{
		\begin{array}{ccc}
			\Rea(\eul^{\img \theta}) & -\Img(\eul^{\img \theta}) & 0 \\
			\Img(\eul^{\img \theta}) & \Rea(\eul^{\img \theta}) & 0 \\
			0 & 0 & 1
		\end{array}}
	\\
& 	j \in \ssI_{p}(\ssW)
	\\
& 	\md_n = \mC_n^\msT (\mg_n + \mv_n -\mM_n \, \dot \mp_n) = \mzr	
\end{align*}

$\mathrs{ABCDEFGHIJKLMNOPQRSTUVWXYZ}$

$\mathcal{ABCDEFGHIJKLMNOPQRSTUVWXYZ}$

$\mathru{ABCDEFGHIJKLMNOPQRSTUVWXYZ}$

$\mathbu{ABCDEFGHIJKLMNOPQRSTUVWXYZ}$

$\mathbb{ABCDEFGHIJKLMNOPQRSTUVWXYZienvw\bbzeta\bbgamma}$

$\mathbf{ABCDEFGHIJKLMNOPQRSTUVWXYZ01}$

$\mathsfbfit{ABCDEFGHIJKLMNOPQRSTUVWXYZ01}$

$\mathsf{ABCDEFGHIJKLMNOPQRSTUVWXYZ}$

$\mathtt{ABCDEFGHIJKLMNOPQRSTUVWXYZ}$

${ABCDEFGHIJKLMNOPQRSTUVWXYZ}$


\end{document}

