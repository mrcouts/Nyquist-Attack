\documentclass[12pt,a4paper]{article}
\usepackage[utf8x]{inputenc}
\usepackage{ucs}
\usepackage{amsmath}
\usepackage{amsfonts}
\usepackage{amssymb}
\usepackage{graphicx}
\usepackage{grffile}
\usepackage{float}
\usepackage{multicol}
\usepackage[portuguese]{babel}
\title{MAT2456 - P2-2014}
\author{André Garnier Coutinho}
\setlength{\textwidth}{17cm}
\setlength{\textheight}{24cm}
\addtolength{\topmargin}{-2cm}
\addtolength{\oddsidemargin}{-2cm}

\newcommand{\re}{\mathbb{R}}

\newcommand{\sen}{\mbox{\,sen}}

\begin{document}
%%%%%%%%%%%%%%%%%%%%%%%%%%%%%%%%%%%%%%%%%%TURMA A%%%%%%%%%%%%%%%%%%%%%%%%%%%%%%%%%%%%%%%%%%%%%%%
\begin{center}
\textbf{Instituto de Matemática e Estatística da USP\\
MAT2455 - Cálculo Diferencial e Integral IV para Engenharia\\}
\textbf{3a. Prova - 2o. Semestre 2014 - 24/11/2014}
\end{center}

\noindent{\bf 1ª Questão:} Resolva:
\begin{itemize}
	\item[a)] (1,5 ponto) $(xy^4 + 4xy^5)\,dx + (2x^2 y^4 -1)\,dy = 0$
	
	\item[b)] (1,5 ponto) $y' = \displaystyle\frac{4yx + 3y^2}{x^2}, \; x>0$
\end{itemize}

\noindent{\bf \\Solução:}
\begin{itemize}
    \item[a)] Sejam $P(x,y) = xy^4 + 4xy^5$ e $Q(x,y) = 2x^2 y^4 -1$. \\
    
    Como $\displaystyle\frac{\partial Q}{\partial x} = 4xy^4 \neq \frac{\partial P}{\partial y} = 4xy^3 + 20xy^4$, a equação não é exata. \\
    
    Como $\displaystyle\frac{\frac{\partial Q}{\partial x} - \frac{\partial P}{\partial y}}{P} = - \frac{4xy^3 + 16xy^4}{xy^4 + 4xy^5} = -\frac{4}{y}$ depende apenas de $y$, a equação admite um fator integrante que depende apenas de $y$, dado por:
    
    $$ \mu(y) = e^{\int (-\frac{4}{y}) \, dy} = e^{-4 \ln|y|} =  e^{\ln|\frac{1}{y^4}|} = \frac{1}{y^4} $$
    
    Multiplicando a EDO pelo fator integrante, temos:
    
    $$ (x+4xy)\,dx + \Big(2x^2 - \frac{1}{y^4}\Big)\,dy = 0 $$
    
   Sendo assim, agora temos: $P(x,y) = x+4xy$ e $Q(x,y) = 2x^2 - \displaystyle\frac{1}{y^4}$. \\
   
   Como $\displaystyle\frac{\partial Q}{\partial x} = 4x = \frac{\partial P}{\partial y} $, a equação é exata. \\
   
   A solução de uma EDO exata é dada por $\phi(x,y) = C, \, C \in \re$, sendo que $\nabla\phi = (P,Q)$. Assim, temos:
   
   $$ 
   \begin{cases}
   \displaystyle\frac{\partial \phi}{\partial x} = x+4xy \\
   \\
   \displaystyle\frac{\partial \phi}{\partial y} = 2x^2 - \displaystyle\frac{1}{y^4}
   \end{cases} 
   $$
   
Integrando em $x$ a primeira equação:

$$ \phi (x,y) = \frac{x^2}{2} + 2 x^2 y + c(y) $$

Derivando em $y$:
$$ \displaystyle \frac{ \partial \phi}{\partial y} = 2x^2 + c'(y) $$

Comparando com a segunda equação:

$$ c'(y) = -\frac{1}{y^4} \Rightarrow c(y) = \frac{1}{3 y^3}$$

Sendo assim, temos:

$$ \phi (x,y) = \frac{x^2}{2} + 2 x^2 y + \frac{1}{3 y^3} $$

Portanto, a solução geral da equação diferencial é dada por:

$$ \frac{x^2}{2} + 2 x^2 y + \frac{1}{3 y^3} = C, \, C \in \re $$ \\

    \item[b)]

$$ y' = \displaystyle\frac{4yx + 3y^2}{x^2} = f(x,y) , \; x > 0 $$

Como $ f(x,y) = f(tx,ty), \forall t \neq 0 $, temos que a equação é homogênea. Sendo assim, utilizamos a seguinte mudança de variável: $ u = \frac{y}{x} \Rightarrow y' = u'x + u $.

$$ y' = \frac{4yx + 3y^2}{x^2} =  \frac{y}{x} \Big ( 4 + 3 \frac{y}{x} \Big) \Rightarrow u'x + u = 4u + 3u^2 $$

$$ u'x = 3u + 3u^2 \Rightarrow \int \frac{du}{3u(u+1)} = \int \frac{dx}{x}, \; u \neq 0 \text{ e } u \neq -1  $$

Repare que $u=0 \Rightarrow y = 0$ e $u=-1 \Rightarrow y = -x$ são soluções da EDO.

Expandindo a função $ \frac{1}{u(u+1)} $ em frações parciais:

$$ \frac{1}{u(u+1)} = \frac{A}{u} + \frac{B}{u+1} $$
$$ A(u+1) + Bu = 1 \Rightarrow A = 1, B = -1 $$
$$ \therefore \int \frac{du}{u(u+1)} = \ln |u| - \ln |u+1| = \ln \Big| \frac{u}{u+1} \Big| $$

Sendo assim, temos:

$$ \frac{1}{3} \ln \Big| \frac{u}{u+1} \Big| = \ln x + C, C \in \re $$
$$ \ln \Big| \frac{u}{u+1} \Big| = 3\ln x + C, C \in \re $$
$$  \frac{u}{u+1} = C x^3, C \in \re $$

Portanto as soluções da equação diferencial para $ x > 0 $ são dadas por:

$$
\begin{cases}
\displaystyle\frac{y}{y+x} = C x^3, C \in \re \\
y = 0 \\
y = -x
\end{cases}
$$


\end{itemize}
%---------------------------------------QUESTAO 2-----------------------------------------
\newpage

\noindent{\bf 2ª Questão:} 

\begin{itemize}
\item[a)] (2,0 pontos) Encontre todas as soluções de

$$ y''' - 3y'' + 4y = e^{2x} + x^2 $$

\item[b)] (1,5 ponto) Encontre todas as soluções de

$$2x^2 y'' + y' (2x - 4x^2) + y (3x^2 -2x -2) = 0, \text{ para } x > 0,$$

admitindo que $y = x e^x$ é uma das soluções.

\end{itemize}

\noindent{\bf Solução:} \\

\begin{itemize}
\item[a)] Deseja-se resolver a seguinte EDO:

\begin{equation}\label{eq:EDO_0_B}
y''' - 3y'' + 4y = e^{2x} + x^2
\end{equation}

Para isso, primeiro vamos encontrar a solução da equação homogênea:

\begin{equation}\label{eq:EDO_H_B}
y''' - 3y'' + 4y = 0
\end{equation}

Como a equação possui coeficientes constantes, temos que as soluções da equação homogênea são da forma

\begin{equation} \label{eq:YH_B}
y(x) = e^{sx}
\end{equation}
 
Deverivando \eqref{eq:YH_B} e substituindo em \eqref{eq:EDO_H_B}, obtemos o seguinte polinômio característico:

$$ s^3 - 3s^2 + 4 = 0 \Rightarrow (s+1)(s-2)^2 = 0 $$

Sendo assim, a solução da equação homogênea é dada por:

$$ y_h(x) = C_1 e^{-x} + C_2 e^{2x} + C_3 x e^{2x}, \; C_1, C_2, C_3 \in \re $$

Para encontrar a solução geral da equação \eqref{eq:EDO_0_B}, iremos utilizar o Método dos Coeficientes a Determinar e o princípio da superposição. Assim, primeiramente encontramos uma solução particular para a seguinte EDO:

\begin{equation} \label{eq:EDO_1_B}
y''' - 3y'' + 4y = e^{2x}
\end{equation}

Como $2$ é raiz dupla do polinômio característico, propomos a seguinte solução particular:

\begin{equation} \label{eq:YP_1_B}
y_{p_1}(x) = x^2 A e^{2x}
\end{equation}

Derivando \eqref{eq:YP_1_B}:

$$ y_{p_1}'(x) = A (2x e^{2x} + 2x e^{2x}) = 2A (x+x^2) e^{2x} $$
$$ y_{p_1}''(x) = 2A [ (2x+1)e^{2x} + 2(x+x^2)e^{2x} ]  = 2A(1 + 4x + 2x^2) e^{2x}  $$
$$ y_{p_1}'''(x) = 2A[ (4x+4)e^{2x} + 2(1+4x+2x^2)] = 4A (2x^2 + 6x + 3) e^{2x}  $$

Substituindo em \eqref{eq:EDO_1_B}:

$$ 4A (2x^2 + 6x + 3) e^{2x} - 3 \cdot 2A(1 + 4x + 2x^2) e^{2x} + 4 \cdot x^2 A e^{2x}   = e^{2x}$$
$$ \Rightarrow A e^{2x} (8x^2 + 24x + 12 -12x^2 - 24x - 6 + 4x^2) = e^{2x} $$
$$ \Rightarrow A e^{2x} \cdot 6 = e^{2x} \Rightarrow A = \frac{1}{6} $$
$$ \therefore y_{p_1}(x) = \frac{x^2 e^{2x}}{6} $$

Agora encontramos uma solução particular para a seguinte EDO:

\begin{equation} \label{eq:EDO_2_B}
y''' - 3y'' + 4y = x^2
\end{equation}

Propomos a seguinte solução particular:

\begin{equation} \label{eq:YP_2_B}
y_{p_2}(x) = A x^2 + B x + C
\end{equation}

Derivando \eqref{eq:YP_2_B}:

$$ y_{p_2}'(x) = 2A x + B  $$
$$ y_{p_2}''(x) = 2A $$
$$ y_{p_2}'''(x) = 0  $$

Substituindo em \eqref{eq:EDO_2_B}:

$$ 1 \cdot 0 - 3 (2A) + 4 (Ax^2 + Bx + C)   = x^2$$
$$
\begin{cases}
4A = 1 \\
4B = 0 \\
-6A + 4C = 0
\end{cases}
\Rightarrow
\begin{cases}
A = \frac{1}{4} \\
B = 0 \\
C = \frac{3}{8}
\end{cases}$$
$$\therefore y_{p_2}(x) = \frac{x^2}{4} + \frac{3}{8}$$

Pelo pricípio da superposição, temos que $y_p(x) = y_{p_1}(x) + y_{p_2}(x)$ é uma solução particular de \eqref{eq:EDO_0_B}.

A solução geral da equação é a soma da solução homogênea com a solução particular. Assim:
$$ y(x) = C_1 e^{-x} + C_2 e^{2x} + C_3 x e^{2x} +  \frac{x^2 e^{2x}}{6} + \frac{x^2}{4} + \frac{3}{8}  , \; C_1, C_2, C_3 \in \re $$

\item[b)]

A EDO a ser resolvida apresenta o seguinte formato:

\begin{equation}\label{eq:EDO_W}
a_1(x) y'' + a_2(x) y' + a_3(x) y = 0
\end{equation}

Sendo:
$$
\begin{cases}
	a_1(x) = 2x^2 \\
	a_2(x) = 2x - 4x^2 \\
	a_3(x) = 3x^2 - 2x + 2
\end{cases}
$$

Sejam $y_1$ e $y_2$ 2 soluções LI da EDO. O Wronskiano de \eqref{eq:EDO_W} é dado por:

$$ W(x) = \begin{vmatrix}
y_1 & y_2 \\
y_1' & y_2' \\
\end{vmatrix} $$

Para EDOs no formato de \eqref{eq:EDO_W}, o Wronskiano respeita a seguinte relação:

$$ W'(x) = -\frac{a_2(x)}{a_1(x)} W(x) \Rightarrow W(x) = C e^{-\int \frac{a_2(x)}{a_1(x)} \,dx}, \; C \neq 0 $$

Sendo assim, dado $y_1(x)$, é possível determinar $y_2(x)$ resolvendo uma EDO de primeira ordem.

Dado que $y_1(x) = x e^x$ é solução da EDO e escolhendo $C=1$, temos:

$$
\begin{vmatrix}
x e^x & y_2 \\
(x+1)e^x & y_2'
\end{vmatrix}
=  e^{-\int \frac{2x - 4x^2}{2x^2} \,dx} = e^{-\int ( \frac{1}{x} - 2 ) \,dx}
$$
$$
\Rightarrow x e^x y_2' - (x+1)e^x y_2 = \frac{e^{2x}}{x}
$$
\begin{equation}\label{eq:EDO_FI}
\Rightarrow y_2' - \frac{x+1}{x} y_2 = \frac{e^{x}}{x^2}
\end{equation}

\eqref{eq:EDO_FI} é uma EDO linear de primeira ordem. Assim, tem-se que um fator integrante da EDO acima é dado por:

$$ \mu(x) = e^{\int( -\frac{x+1}{x}) \,dx} = e^{\int( -1 - \frac{1}{x}) \,dx} = \frac{e^{-x}}{x} $$

Multiplicando \eqref{eq:EDO_FI} pelo fator integrante $\mu(x)$, temos:

$$ y_2' \frac{e^{-x}}{x} - y_2 e^{-x} \frac{x+1}{x^2} = \frac{1}{x^3} $$
$$ \Rightarrow \Big(y_2 \frac{e^{-x}}{x}\Big)' = \frac{1}{x^3} $$
$$ \Rightarrow y_2 \frac{e^{-x}}{x} = - \frac{1}{2x^2} $$
$$ \therefore y_2(x) = - \frac{e^x}{2x} $$

Portanto, a solução geral de \eqref{eq:EDO_W} é dado por: $y(x) = C_1 x e^x + C_2 \frac{e^x}{x}, \; C_1, C_2 \in \re $





\end{itemize}

%---------------------------------------QUESTAO 3-----------------------------------------

\newpage
\noindent{\bf 3ª Questão: }
\begin{itemize}
\item[a)] (2,5 pontos) Sejam $f$ e $g$ contínuas no intervalo $\Big] - \displaystyle\sqrt{\frac{\pi}{2}} , \displaystyle\sqrt{\frac{\pi}{2}} \Big[$. \\

Sabendo que $y_1 = \cos(x^2) \,$ e $\, y_2 = x \cos(x^2)$ são soluções de $y'' + f(x)y' + g(x)y = 0$, encontre todas as soluções de

$$y'' + f(x)y' + g(x)y = \frac{\cos(x^2)}{\sqrt{1+x^2}}.$$

\item[b)] (1,0 ponto) Encontre todas as soluções da equação diferencial linear homogênea de coeficiêntes constantes, cujo polinômio característico é $(\lambda - 4)^3 (\lambda^2 - 8\lambda + 25)^2$.

\end{itemize}


\noindent{\bf Solução:}
\\

\begin{itemize}
\item[a)] A solução homogênea da equação é a combinação linear das duas soluções homogêneas, $y_1$ e $y_2$:

$$ y_h (x) = C_1 \cos(x^2) + C_2 x \cos(x^2), C_1, C_2, \in \re $$

Utilizamos o método da Variação dos Parâmetros para encontrar uma solução particular da seguinte forma:

$$ y_p (x) = u_1(x) \cos(x^2) + u_2(x) x \cos(x^2) $$

Aplicando o método:

$$
\begin{bmatrix}
y_1 & y_2 \\
y_1' & y_2'
\end{bmatrix}
\begin{bmatrix}
u_1' \\
u_2'
\end{bmatrix}
= 
\begin{bmatrix}
0 \\
q(x)
\end{bmatrix}
\Rightarrow
\begin{bmatrix}
\cos(x^2) & x\cos(x^2) \\
-2x\sen(x^2) & -2x\sen(x^2) + \cos(x^2)
\end{bmatrix}
\begin{bmatrix}
u_1' \\
u_2'
\end{bmatrix}
= 
\begin{bmatrix}
0 \\
\frac{\cos(x^2)}{\sqrt{1+x^2}}
\end{bmatrix}
$$

Resolvendo o sistema pela regra de Cramer:

$$ u_1' = \frac{ \begin{vmatrix}
0 & x\cos(x^2) \\
\frac{\cos(x^2)}{\sqrt{1+x^2}} & -2x\sen(x^2) + \cos(x^2)
\end{vmatrix}  }{ \begin{vmatrix}
\cos(x^2) & x\cos(x^2) \\
-2x\sen(x^2) & -2x\sen(x^2) + \cos(x^2)
\end{vmatrix} }
= \frac{ - \frac{ x \cos^2 (x^2)}{\sqrt{1+x^2}} }{ \cos^2(x^2) } = -\frac{x}{\sqrt{1+x^2}}
 $$
 
 $$ u_2' = \frac{ \begin{vmatrix}
\cos(x^2) & 0 \\
-2x\sen(x^2) & \frac{\cos(x^2)}{\sqrt{1+x^2}}
\end{vmatrix}  }{ \begin{vmatrix}
\cos(x^2) & x\cos(x^2) \\
-2x\sen(x^2) & -2x\sen(x^2) + \cos(x^2)
\end{vmatrix} }
= \frac{  \frac{ \cos^2 (x^2)}{\sqrt{1+x^2}} }{ \cos^2(x^2) } = \frac{1}{\sqrt{1+x^2}}
 $$

Integrando:

$$ u_1 = -\int \frac{x}{\sqrt{1+x^2}} \, dx \underset{du = 2x\,dx}{\underset{u = 1+x^2}{=}} - \frac{1}{2} \int \frac{du}{\sqrt{u}} = - \sqrt{u} = -\sqrt{1+x^2}  $$
$$ u_2 = \int \frac{1}{\sqrt{1+x^2}} \, dx \underset{dx = \sec^2 \theta \, d\theta}{\underset{x = \tan\theta}{=}} \int \sec\theta \, d\theta = \ln|\sec\theta + \tan\theta| = \ln|\sqrt{1+x^2} + x| $$
$$ \therefore y_p(x) = ( -\sqrt{1+x^2} )\cos(x^2) + (\ln|\sqrt{1+x^2} + x|)x\cos(x^2) $$

A solução geral é a soma da solução homogênea com a solução particular:
$$ y(x) = ( C_1   -\sqrt{1+x^2} ) \cos(x^2) + (C_2  + \ln|\sqrt{1+x^2} + x|)x\cos(x^2)  , C_1, C_2, \in \re $$

\item[b)] Pelo seu polinômio característico, trata-se de uma EDO linear de ordem 7. O polinômio apresenta raiz $\lambda = 4$ com multiplicidade 3. A esta raiz, associam-se as soluções:
$$\{ e^{4x}, x e^{4x}, x^2 e^{4x} \}$$

As outras duas raízes são $\lambda = \frac{8 \pm \sqrt{64-100}}{2} = 4 \pm 3i$, com multiplicidade 2. As soluções associadas a estas raízes são:
$$\{e^{4x}\cos(3x),e^{4x}\sen(3x),x e^{4x}\cos(3x),x e^{4x}\sen(3x) \} $$

Portanto, as solução geral da EDO é dada por:
$$ y(x) = e^{4x} (C_1 + C_2 x + C_3 x^2 + (C_4 x + C_5)\cos(3x) + (C_6 x + C_7)\sin(3x) ) , \, C_1, C_2, C_3, C_4, C_5, C_6, C_7 \in \re $$

\end{itemize}

\end{document} 
