\documentclass[12pt,a4paper]{article}
\usepackage[utf8x]{inputenc}
\usepackage{ucs}
\usepackage{amsmath}
\usepackage{amsfonts}
\usepackage{amssymb}
\usepackage{graphicx}
\usepackage{grffile}
\usepackage{float}
\usepackage{multicol}
\usepackage[portuguese]{babel}
\title{MAT2456 - P2-2014}
\author{André Garnier Coutinho}
\setlength{\textwidth}{17cm}
\setlength{\textheight}{24cm}
\addtolength{\topmargin}{-2cm}
\addtolength{\oddsidemargin}{-2cm}

\newcommand{\re}{\mathbb{R}}

\newcommand{\sen}{\mbox{\,sen}}

\begin{document}
%%%%%%%%%%%%%%%%%%%%%%%%%%%%%%%%%%%%%%%%%%TURMA A%%%%%%%%%%%%%%%%%%%%%%%%%%%%%%%%%%%%%%%%%%%%%%%
\begin{center}
\textbf{Instituto de Matemática e Estatística da USP\\
MAT2455 - Cálculo Diferencial e Integral IV para Engenharia\\}
\textbf{1a. Prova - 2o. Semestre 2015 - 01/09/2015}
\end{center}

\noindent {\bf Turma A}

\noindent{\bf 1ª Questão:}
\begin{itemize}
	\item[a)] (1,0) Mostre que a seguência
	
	$$0,29 \;\;\;\; 0,298 \;\;\;\; 0,2989 \;\;\;\; 0,29898 \;\;\;\; 0,298989$$
	
	é convergente e calcule o valor de seu limite como razão entre dois inteiros.
	
	\item[b)] Considere a série
	
	$$\displaystyle\sum_{n=1}^\infty (-1)^{n-1} \, \frac{1}{(2n-1)!} = 1 - \frac{1}{3!} + \frac{1}{5!} - ...$$
	
    Prove que a série é convergente e calcule explicitamente uma aproximação do valor da soma da série com erro menor que $2 \cdot 10^{-4}$.
\end{itemize}

\noindent{\bf \\Solução:}
\begin{itemize}
    \item[a)] A sequencia dada pode ser descrita como:
    
    $$
    \begin{cases}
    a_1 = 0,29 \\
    a_{n+1} =
        \begin{cases}
        a_n + 8 \cdot 10^{-(n+2)}, \text{$n$ ímpar} \\
        a_n + 9 \cdot 10^{-(n+2)}, \text{$n$ par} \\
        \end{cases}  
    \end{cases}
    $$
    
    Assim, pode-se concluir que a sequência é crescente, pois $a_{n+1} > a_n, \forall n \geq 1$. 
    
    Como a sequência é crescente e claramente limitada superiormente por $0,3$, temos que a sequência é convergente.
    
    O convergência da sequência garante que todas as sub-sequências devem convergir para o mesmo valor. Sendo assim, definimos a seguinte sub-sequência:
    
    $$
    \begin{cases}
    a_1 = 0,29 \\
    a_{n+2} =  a_n + 89 \cdot 10^{-(n+3)}, \text{$n$ ímpar}
    \end{cases}
    $$
    
    A partir da expressão recursiva, é possível determinar uma expressão não recursiva para o termo geral da sub-sequência:
    
    \begin{align*} 
    n &= 1 \rightarrow a_3 = 0,29 + 89 \cdot 10^{-4} \\ 
    n &= 3 \rightarrow a_5 = a_3 + 89 \cdot 10^{-6} = 0,29 + 89 \cdot 10^{-4} + 89 \cdot 10^{-6} \\
    n &= 5 \rightarrow a_7 = a_5 + 89 \cdot 10^{-8} = 0,29 + 89 \cdot 10^{-4} + 89 \cdot 10^{-6} + 89 \cdot 10^{-8}\\
    \vdots \\
    n &= k-2 \rightarrow a_k = 0,29 + 89 \cdot 10^{-4}(1 + 10^{-2} + 10^{-4} + ... + 10^{-k+3}), \; \text{$k$ ímpar}
    \end{align*}
    
    Assim, podemos calcular o limite desta sub-sequência da seguinte maneira:
    
    $$ \lim_{k \rightarrow \infty} a_k = 0,29 + 89 \cdot 10^{-4} \sum_{m=0}^{\infty} (10^{-2})^m $$
    
    Como $\displaystyle\sum_{m=0}^{\infty} (10^{-2})^m$ é a soma de uma PG de razão positiva menor que $1$, temos:
    
    $$ \lim_{k \rightarrow \infty} a_k = 0,29 + 89 \cdot 10^{-4} \cdot \frac{1}{1 - 10^{-2}} = \Big(29 + \frac{89}{99} \Big) \frac{1}{100} = \frac{148}{495} $$ \\



    \item[b)] Para analisar a convergência da série, utilizamos o critério das séries alternadas:
    
    \begin{itemize}
    \item[$\bullet$] $\displaystyle\lim_{n \rightarrow \infty} |a_n| = \lim_{n \rightarrow \infty} \displaystyle\frac{1}{(2n-1)!} = 0 $
    \item[$\bullet$] $|a_{n+1}| = \displaystyle\frac{1}{(2n+1)!} = \displaystyle\frac{1}{(2n+1)(2n)(2n-1)!} = \frac{|a_n|}{(2n+1)(2n)} \\ \Rightarrow |a_{n+1}| < |a_n|, \forall n \geq 1  $
    \end{itemize}
    
    Como o módulo do termo geral da série é descrescente e vai a zero, pode-se afirmar que, pelo critério das séries alternadas, a série é convergente.
    
    Deseja-se aproximar $\displaystyle\sum_{n=1}^\infty (-1)^{n-1} \, \frac{1}{(2n-1)!}$ por $\displaystyle\sum_{n=1}^k (-1)^{n-1} \, \frac{1}{(2n-1)!}$  com $erro < \varepsilon = 2 \cdot 10^{-4}$.
Para séries alternadas, o erro da aproximação respeita a seguinte relação:

$$ erro = \sum_{n=k+1}^\infty a_n < |a_{k+1}| $$

Repare que se encontrarmos um valor de $k$ que respeite $|a_{k+1}| < \epsilon$, temos que $erro < \epsilon$. Sendo assim, temos:

$$|a_{k+1}| = \frac{1}{(2k+1)!} < 2 \cdot 10^{-4} \Rightarrow (2k+1)! > 0,5 \cdot 10^{4} $$

\begin{align*} 
    k &= 1 \rightarrow (2k+1)! = 3! = 6 < 5000 \\ 
    k &= 2 \rightarrow (2k+1)! = 5! = 120 < 5000 \\ 
    k &= 3 \rightarrow (2k+1)! = 7! = 5040 > 5000
\end{align*}
Portanto: 
$$ \sum_{n=1}^\infty (-1)^{n-1} \, \frac{1}{(2n-1)!}  \simeq \sum_{n=1}^3 (-1)^{n-1} \, \frac{1}{(2n-1)!} = 1 - \frac{1}{3!} + \frac{1}{5!}  $$

Com $erro < 10^{-3}$.
\end{itemize}
%---------------------------------------QUESTAO 2-----------------------------------------
\newpage

\noindent{\bf 2ª Questão:} Decidir se a série dada converge ou não. Em caso afirmativo, dizer se a convergência é absoluta ou condicional.

\begin{itemize}
\item[a)] (1,0) $\displaystyle\sum_{n=1}^\infty (-1)^{n-1} \, \Big( \frac{n}{3n-1} \Big)^{2n-1}$
\item[b)] (1,0) $\displaystyle\sum_{n=1}^\infty (-1)^{n-1} \, \frac{e^{2n} (n!)^2}{n^{2n}}$
\item[c)] (1,0) $\displaystyle\sum_{n=1}^\infty (-1)^{n-1} \, \frac{\ln \frac{n}{3}}{n}$
\end{itemize}

\noindent{\bf Solução:} \\

\begin{itemize}
\item[a)] Aplicando o critério da razão:

$$\lim_{n\rightarrow\infty} \frac{|a_{n+1}|}{|a_n|} = \lim_{n\rightarrow\infty} \Big( \frac{n+1}{3n+2} \Big)^{2n+1} \Big( \frac{3n-1}{n} \Big)^{2n-1} = \lim_{n\rightarrow\infty}  \Big( \frac{n+1}{3n+2} \Big)^2 \Big( \frac{n+1}{n} \Big)^{2n-1} \Big( \frac{3n-1}{3n+2} \Big)^{2n-1} $$

\begin{itemize}
\item[$\bullet$] $\displaystyle\lim_{n\rightarrow\infty}  \Big( \displaystyle\frac{n+1}{3n+2} \Big)^2 = \displaystyle\lim_{n\rightarrow\infty}  \Big( \displaystyle\frac{1+\frac{1}{n}}{3+\frac{2}{n}} \Big)^2 = \displaystyle\frac{1}{3^2} $
\item[$\bullet$] $\displaystyle\lim_{n\rightarrow\infty} \Big( \displaystyle\frac{n+1}{n} \Big)^{2n-1} = \displaystyle\lim_{n\rightarrow\infty} e^{(2n-1) \ln(\frac{n+1}{n})} = e^2$, pois:
\begin{itemize}
\item[--] $\displaystyle\lim_{x\rightarrow\infty} (2x-1) \ln\Big(\frac{x+1}{x}\Big) = \displaystyle\lim_{x\rightarrow\infty} \frac{ \ln\big(\frac{x+1}{x}\big) }{\frac{1}{2x-1}} \underset{u = \frac{1}{x}}{=} \displaystyle\lim_{u\rightarrow 0} \frac{ \ln(u+1) }{\frac{u}{2-u}}$

$\underset{\frac{0}{0}}{\overset{L'H}{=}} \displaystyle\lim_{u\rightarrow 0} \displaystyle \frac{\frac{1}{u+1}}{\frac{1\cdot(2-u)-u\cdot(-1)}{(2-u)^2}}=2$
\end{itemize}
\item[$\bullet$] $\displaystyle\lim_{n\rightarrow\infty} \Big( \displaystyle\frac{3n-1}{3n+2} \Big)^{2n-1} = \displaystyle\lim_{n\rightarrow\infty} e^{(2n-1) \ln(\frac{3n-1}{3n+2})} = e^{-2}$, pois:
\begin{itemize}
\item[--] $\displaystyle\lim_{x\rightarrow\infty} (2x-1) \ln\Big(\frac{3x-1}{3x+2}\Big) = \displaystyle\lim_{x\rightarrow\infty} \frac{ \ln\big(\frac{3x-1}{3x+2}\big) }{\frac{1}{2x-1}} \underset{u = \frac{1}{x}}{=} \displaystyle\lim_{u\rightarrow 0} \frac{ \ln\big(\frac{3-u}{3+2u}\big) }{\frac{u}{2-u}}$

$\underset{\frac{0}{0}}{\overset{L'H}{=}} \displaystyle\lim_{u\rightarrow 0} \displaystyle \frac{\frac{3+2u}{3-u} \cdot \frac{(-1)\cdot(3+2u)-(3-u)\cdot 2}{(3+2u)^2}}{\frac{1\cdot(2-u)-u\cdot(-1)}{(2-u)^2}}= -2$
\end{itemize}
\end{itemize}

$$\therefore \lim_{n\rightarrow\infty} \frac{|a_{n+1}|}{|a_n|} = \frac{1}{3^2} \cdot e^2 \cdot e^{-2} = \frac{1}{9} < 1 $$

Como $\displaystyle\lim_{n\rightarrow\infty} \frac{|a_{n+1}|}{|a_n|} < 1$, a série converge absolutamente.

\item[b)] Aplicando o critério da razão:

$$ \frac{|a_{n+1}|}{|a_n|} =  \frac{e^{2n+2}((n+1)!)^2}{(n+1)^{2n+2}} \cdot  \frac{n^{2n}}{e^{2n} (n!)^2} =  \frac{e^{2}(n+1)^2 (n!)^2}{(n+1)^{2} (n+1)^{2n}} \cdot  \frac{n^{2n}}{ (n!)^2} = \Bigg( \frac{e}{(\frac{n+1}{n})^n} \Bigg)^2 $$
$$\lim_{n\rightarrow\infty} \frac{|a_{n+1}|}{|a_n|} = \lim_{n\rightarrow\infty} \Bigg( \frac{e}{(\frac{n+1}{n})^n} \Bigg)^2 = 1 $$

O critério da razão não leva a nenhuma conclusão, pois $\displaystyle\lim_{n\rightarrow\infty} \frac{|a_{n+1}|}{|a_n|} = 1 $. Porém sabe-se que:

$$ \frac{|a_{n+1}|}{|a_n|} = \Bigg( \frac{e}{(\frac{n+1}{n})^n} \Bigg)^2 > 1, \forall n > 0 $$

porque sabemos que a sequência $(\frac{n+1}{n})^n$ é crescente e tende a $e$. Isso implica que o termo geral não vai a zero e, portanto, a série diverge.

\item[c)] Para analisar a convergência da série, utilizamos o critério da integral, com $f(x) = \frac{\ln \frac{x}{3}}{x}$. Para isso, precisamos verificar se o termo geral da série vai a zero e se $f$ é decrescente.
    
    \begin{itemize}
    \item[$\bullet$] $\displaystyle\lim_{n \rightarrow \infty} |a_n| = \lim_{x \rightarrow \infty} f(x) = \lim_{x \rightarrow \infty} \frac{\ln \frac{x}{3}}{x} \underset{\frac{\infty}{\infty}}{ \overset{L'H}{=}} \lim_{x \rightarrow \infty} \frac{\frac{3}{x} \cdot \frac{1}{3}}{1} = 0$
    \item[$\bullet$] $f'(x) = \frac{\frac{1}{x} \cdot x - \ln(\frac{x}{3}) \cdot 1}{x^2} = \frac{1 - \ln(\frac{x}{3})}{x^2} < 0, \forall x > 3e  \Rightarrow f$ é decrescente $\forall x > 3e$.
    \end{itemize}
    
    $$ \int_{10}^\infty f(x) \, dx = \int_{10}^\infty \frac{\ln \frac{x}{3}}{x}  \, dx \underset{u = \ln \frac{x}{3} \Rightarrow du = \frac{dx}{x} }{=} \int_{\ln 10}^\infty u  \, du = \lim_{u\rightarrow\infty} \frac{u^2}{2} - \frac{(\ln 10)^2}{2} = \infty $$
    
    Como a integral imprópria diverge, $\displaystyle\sum_{n=1}^\infty |a_n|$ diverge. Porém, a série alternada converge pelo critério das séries alternadas, pois, como foi visto anteriormente, o módulo do termo geral da série é descrescente e vai a zero. Sendo assim, a série converge condicionalmente.

\end{itemize}

%---------------------------------------QUESTAO 3-----------------------------------------

\newpage
\noindent{\bf 3ª Questão: }
\begin{itemize}
\item[a)] (2,5) Determinar os valores de $x \in \mathbb{R}$ para os quais a série de potências

$$ \sum_{n=1}^\infty \frac{3n-2}{(n+1)^2 \, 2^{n+1}} (x-3)^n $$

é convergente.

\item[b)] (1,5) Estudar a convergência em termos de $\alpha > 0$:

$$ \sum_{n=3}^\infty \frac{1}{n (\ln n)(\ln (\ln n))^\alpha} $$

\end{itemize}


\noindent{\bf Solução:}
\\

\begin{itemize}
\item[a)] Aplicando o critério da razão, temos:

$$\lim_{n\rightarrow\infty} \frac{|a_{n+1}|}{|a_n|} = \lim_{n\rightarrow\infty} \frac{(3n+1) |x-3|^{n+1}}{(n+2)^2 \, 2^{n+2}}  \cdot \frac{(n+1)^2 \, 2^{n+1}}{(3n-2) |x-3|^n} = \lim_{n\rightarrow\infty} \Big( \frac{3n+1}{3n-2} \Big) \Big( \frac{n+1}{n+2} \Big)^2 \frac{|x-3|}{2}  $$
$$ = \lim_{n\rightarrow\infty} \Bigg( \frac{3+\frac{1}{n}}{3-\frac{2}{n}} \Bigg) \Bigg( \frac{1+\frac{1}{n}}{1+\frac{2}{n}} \Bigg)^2 \frac{|x-3|}{2} = \frac{|x-3|}{2} $$

Para $\frac{|x-3|}{2} < 1 \Leftrightarrow |x-3| < 2$: a série converge absolutamente, pelo critério da razão.

Para $\frac{|x-3|}{2} > 1 \Leftrightarrow |x-3| > 2$: o termo geral não vai a zero e, portanto, a série diverge.

Para $\frac{|x-3|}{2} = 1 \Rightarrow x = 1 \lor x=5$: o critério da razão é inconclusivo, outros critérios devem ser utilizados. \\

Para $x=5$, temos:

$$  \sum_{n=1}^\infty \frac{3n-2}{(n+1)^2 \, 2^{n+1}} (x-3)^n = \sum_{n=1}^\infty \frac{3n-2}{(n+1)^2 \, 2} $$

Para analisar a convergência desta série, utilizamos o critério da comparação no limite:

$$ L = \lim_{n\rightarrow\infty} \frac{\frac{3n-2}{(n+1)^2 \, 2}}{\frac{1}{n}} = \lim_{n\rightarrow\infty} \frac{3n^2-2n}{(n+1)^2 \, 2} = \lim_{n\rightarrow\infty} \frac{3-\frac{2}{n}}{(1+\frac{1}{n})^2 \, 2} = \frac{3}{2}  $$

Como $0 < L < \infty$, temos que série diverge, pois $\displaystyle\sum_{n=1}^\infty \frac{1}{n}$ é uma série divergente.

Para $x=1$, temos:

$$  \sum_{n=1}^\infty \frac{3n-2}{(n+1)^2 \, 2^{n+1}} (x-3)^n = \sum_{n=1}^\infty \frac{(-1)^n \, (3n-2)}{(n+1)^2 \, 2} $$

Esta série não apresenta convergência absoluta, pois já verificamos que $\displaystyle\sum_{n=1}^\infty \frac{3n-2}{(n+1)^2 \, 2}$ diverge. \\

Para verificar se há convergência condicional, utilizamos o critério das séries alternadas: 
    
    \begin{itemize}
    \item[$\bullet$] $\displaystyle\lim_{n \rightarrow \infty} |a_n| = \lim_{n \rightarrow \infty} \frac{3n-2}{(n+1)^2 \, 2} = \lim_{n \rightarrow \infty} \frac{3n-2}{(n^2 + 2n + 1) \, 2} = \lim_{n \rightarrow \infty} \frac{3-\frac{2}{n}}{(n + \frac{2}{n} + \frac{1}{n^2}) \, 2} = 0 $
    \item[$\bullet$] $|a_{n}| = f(n) = \displaystyle\frac{3n-2}{(n+1)^2 \, 2}  $
    
    $f'(x) = \displaystyle\frac{3 \cdot (x+1)^2 -  (3x-2) \cdot 2 (x+1)}{(x+1)^4} = \frac{7 - 3x}{(x+1)^3} < 0, \forall x > \frac{7}{3}$
    
    $\therefore |a_{n}|$ é decrescente $\forall n \geq 3$.  
    
    \end{itemize}
    
    Como o módulo do termo geral da série é descrescente e vai a zero, pode-se afirmar que, pelo critério das séries alternadas, a série é convergente. Portanto, a série converge condicionalmente. 

\item[b)] Para analisar a convergência da série, utilizamos o critério da integral, com $f(x) = \frac{1}{x (\ln x)(\ln (\ln x))^\alpha}$. Como $\alpha > 0$, o termo geral da série vai a zero e $f$ é claramente positiva e decrescente para $x \geq 3$.
    
    $$ \int_{3}^\infty f(x) \, dx = \int_{3}^\infty \frac{1}{x (\ln x)(\ln (\ln x))^\alpha}  \, dx \underset{u = \ln x \Rightarrow du = \frac{dx}{x} }{=} \int_{\ln 3}^\infty \frac{1}{ u(\ln u)^\alpha}  \, du = \underset{v = \ln u \Rightarrow dv = \frac{du}{u} }{=} \int_{\ln(\ln 3)}^\infty \frac{1}{ v^\alpha}  \, dv $$
    
    $$
    \Rightarrow \int_{3}^\infty f(x) \, dx =
    \begin{cases}
    \displaystyle \lim_{v\rightarrow\infty} \displaystyle\frac{v^{1-\alpha}}{1-\alpha} - \displaystyle\frac{(\ln(\ln 3))^{1-\alpha}}{1-\alpha}, \; \alpha \neq 1 \\
    \displaystyle \lim_{v\rightarrow\infty}\ln (v) - \ln (\ln(\ln 3)), \;\;\;\;\; \alpha = 1
    \end{cases}
    $$
    
    $$
    \lim_{v\rightarrow\infty} \displaystyle\frac{v^{1-\alpha}}{1-\alpha} =
    \begin{cases}
    \infty, \;\; 1-\alpha > 0 \\
    0, \;\;\;\; 1-\alpha < 0
    \end{cases}
    $$
    
        $$
    \therefore \int_{3}^\infty f(x) \, dx =
    \begin{cases}
    \infty, \;\;\;\;\;\;\;\;\;\;\;\;\;\;\;\;\;\;\;\;  0 < \alpha \leq 1 \\
    - \displaystyle\frac{(\ln(\ln 3))^{1-\alpha}}{1-\alpha}, \; \alpha > 1
    \end{cases}
    $$
    
    A série converge se e somente se a integral imprópria converge. Sendo assim, temos que a série converge para $\alpha > 1$ e diverge para $0 < \alpha \leq 1$.

\end{itemize}

\newpage
%%%%%%%%%%%%%%%%%%%%%%%%%%%%%%%%%%%%%%%%%%TURMA B%%%%%%%%%%%%%%%%%%%%%%%%%%%%%%%%%%%%%%%%%%%%%%%

\begin{center}
\textbf{Instituto de Matemática e Estatística da USP\\
MAT2455 - Cálculo Diferencial e Integral IV para Engenharia\\}
\textbf{1a. Prova - 2o. Semestre 2015 - 01/09/2015}
\end{center}

\noindent {\bf Turma B}

\noindent{\bf 1ª Questão:}
\begin{itemize}
	\item[a)] (1,0) Mostre que a seguência
	
	$$0,79 \;\;\;\; 0,798 \;\;\;\; 0,7989 \;\;\;\; 0,79898 \;\;\;\; 0,798989$$
	
	é convergente e calcule o valor de seu limite como razão entre dois inteiros.
	
	\item[b)] Considere a série
	
	$$\displaystyle\sum_{n=0}^\infty (-1)^n \, \frac{1}{(2n)!} = 1 - \frac{1}{2!} + \frac{1}{4!} - ...$$
	
    Prove que a série é convergente e calcule explicitamente uma aproximação do valor da soma da série com erro menor que $2 \cdot 10^{-3}$.
\end{itemize}

\noindent{\bf \\Solução:}
\begin{itemize}
    \item[a)] A sequencia dada pode ser descrita como:
    
    $$
    \begin{cases}
    a_1 = 0,79 \\
    a_{n+1} =
        \begin{cases}
        a_n + 8 \cdot 10^{-(n+2)}, \text{$n$ ímpar} \\
        a_n + 9 \cdot 10^{-(n+2)}, \text{$n$ par} \\
        \end{cases}  
    \end{cases}
    $$
    
    Assim, pode-se concluir que a sequência é crescente, pois $a_{n+1} > a_n, \forall n \geq 1$. 
    
    Como a sequência é crescente e claramente limitada superiormente por $0,8$, temos que a sequência é convergente.
    
    O convergência da sequência garante que todas as sub-sequências devem convergir para o mesmo valor. Sendo assim, definimos a seguinte sub-sequência:
    
    $$
    \begin{cases}
    a_1 = 0,79 \\
    a_{n+2} =  a_n + 89 \cdot 10^{-(n+3)}, \text{$n$ ímpar}
    \end{cases}
    $$
    
    A partir da expressão recursiva, é possível determinar uma expressão não recursiva para o termo geral da sub-sequência:
    
    \begin{align*} 
    n &= 1 \rightarrow a_3 = 0,79 + 89 \cdot 10^{-4} \\ 
    n &= 3 \rightarrow a_5 = a_3 + 89 \cdot 10^{-6} = 0,79 + 89 \cdot 10^{-4} + 89 \cdot 10^{-6} \\
    n &= 5 \rightarrow a_7 = a_5 + 89 \cdot 10^{-8} = 0,79 + 89 \cdot 10^{-4} + 89 \cdot 10^{-6} + 89 \cdot 10^{-8}\\
    \vdots \\
    n &= k-2 \rightarrow a_k = 0,79 + 89 \cdot 10^{-4}(1 + 10^{-2} + 10^{-4} + ... + 10^{-k+3}), \; \text{$k$ ímpar}
    \end{align*}
    
    Assim, podemos calcular o limite desta sub-sequência da seguinte maneira:
    
    $$ \lim_{k \rightarrow \infty} a_k = 0,79 + 89 \cdot 10^{-4} \sum_{m=0}^{\infty} (10^{-2})^m $$
    
    Como $\displaystyle\sum_{m=0}^{\infty} (10^{-2})^m$ é a soma de uma PG de razão positiva menor que $1$, temos:
    
    $$ \lim_{k \rightarrow \infty} a_k = 0,79 + 89 \cdot 10^{-4} \cdot \frac{1}{1 - 10^{-2}} = \Big(79 + \frac{89}{99} \Big) \frac{1}{100} = \frac{791}{990} $$ \\



    \item[b)] Para analisar a convergência da série, utilizamos o critério das séries alternadas:
    
    \begin{itemize}
    \item[$\bullet$] $\displaystyle\lim_{n \rightarrow \infty} |a_n| = \lim_{n \rightarrow \infty} \displaystyle\frac{1}{(2n)!} = 0 $
    \item[$\bullet$] $|a_{n+1}| = \displaystyle\frac{1}{(2n+2)!} = \displaystyle\frac{1}{(2n+2)(2n+1)(2n)!} = \frac{|a_n|}{(2n+2)(2n+1)} \\ \Rightarrow |a_{n+1}| < |a_n|, \forall n \geq 0  $
    \end{itemize}
    
    Como o módulo do termo geral da série é descrescente e vai a zero, pode-se afirmar que, pelo critério das séries alternadas, a série é convergente.
    
    Deseja-se aproximar $\displaystyle\sum_{n=0}^\infty (-1)^n \, \frac{1}{(2n)!}$ por $\displaystyle\sum_{n=0}^k (-1)^n \, \frac{1}{(2n)!}$  com $erro < \varepsilon = 2\cdot 10^{-3}$.
Para séries alternadas, o erro da aproximação respeita a seguinte relação:

$$ erro = \sum_{n=k+1}^\infty a_n < |a_{k+1}| $$

Repare que se encontrarmos um valor de $k$ que respeite $|a_{k+1}| < \epsilon$, temos que $erro < \epsilon$. Sendo assim, temos:

$$|a_{k+1}| = \frac{1}{(2k+2)!} < 2\cdot 10^{-3} \Rightarrow (2k+2)! > 0,5 \cdot 10^{3} $$

\begin{align*} 
    k &= 1 \rightarrow (2k+2)! = 4! = 24 < 500 \\ 
    k &= 2 \rightarrow (2k+2)! = 6! = 720 > 500 \\
\end{align*}

Portanto: 

$$ \sum_{n=0}^\infty (-1)^n \, \frac{1}{(2n)!}  \simeq \sum_{n=0}^2 (-1)^n \, \frac{1}{(2n)!} = 1 - \frac{1}{2!} + \frac{1}{4!} $$

Com $erro < 2 \cdot 10^{-3}$.
    

\end{itemize}
\ \

%---------------------------------------QUESTAO 2-----------------------------------------
\newpage

\noindent{\bf 2ª Questão:} Decidir se a série dada converge ou não. Em caso afirmativo, dizer se a convergência é absoluta ou condicional.

\begin{itemize}
\item[a)] (1,0) $\displaystyle\sum_{n=1}^\infty (-1)^{n-1} \, \Big( \frac{n}{2n-1} \Big)^{3n-1}$
\item[b)] (1,0) $\displaystyle\sum_{n=1}^\infty (-1)^{n-1} \, \frac{e^{3n} (n!)^3}{n^{3n}}$
\item[c)] (1,0) $\displaystyle\sum_{n=1}^\infty (-1)^{n-1} \, \frac{\ln \frac{n}{2}}{n}$
\end{itemize}

\noindent{\bf Solução:} \\

\begin{itemize}
\item[a)] Aplicando o critério da razão:

$$\lim_{n\rightarrow\infty} \frac{|a_{n+1}|}{|a_n|} = \lim_{n\rightarrow\infty} \Big( \frac{n+1}{2n+1} \Big)^{3n+2} \Big( \frac{2n-1}{n} \Big)^{3n-1} = \lim_{n\rightarrow\infty}  \Big( \frac{n+1}{2n+1} \Big)^3 \Big( \frac{n+1}{n} \Big)^{3n-1} \Big( \frac{2n-1}{2n+1} \Big)^{3n-1} $$

\begin{itemize}
\item[$\bullet$] $\displaystyle\lim_{n\rightarrow\infty}  \Big( \displaystyle\frac{n+1}{2n+1} \Big)^3 = \displaystyle\lim_{n\rightarrow\infty}  \Big( \displaystyle\frac{1+\frac{1}{n}}{2+\frac{1}{n}} \Big)^3 = \displaystyle\frac{1}{2^3} $
\item[$\bullet$] $\displaystyle\lim_{n\rightarrow\infty} \Big( \displaystyle\frac{n+1}{n} \Big)^{3n-1} = \displaystyle\lim_{n\rightarrow\infty} e^{(3n-1) \ln(\frac{n+1}{n})} = e^3$, pois:
\begin{itemize}
\item[--] $\displaystyle\lim_{x\rightarrow\infty} (3x-1) \ln\Big(\frac{x+1}{x}\Big) = \displaystyle\lim_{x\rightarrow\infty} \frac{ \ln\big(\frac{x+1}{x}\big) }{\frac{1}{3x-1}} \underset{u = \frac{1}{x}}{=} \displaystyle\lim_{u\rightarrow 0} \frac{ \ln(u+1) }{\frac{u}{3-u}}$

$\underset{\frac{0}{0}}{\overset{L'H}{=}} \displaystyle\lim_{u\rightarrow 0} \displaystyle \frac{\frac{1}{u+1}}{\frac{1\cdot(3-u)-u\cdot(-1)}{(3-u)^2}}=3$
\end{itemize}
\item[$\bullet$] $\displaystyle\lim_{n\rightarrow\infty} \Big( \displaystyle\frac{2n-1}{2n+1} \Big)^{3n-1} = \displaystyle\lim_{n\rightarrow\infty} e^{(3n-1) \ln(\frac{2n-1}{2n+1})} = e^{-3}$, pois:
\begin{itemize}
\item[--] $\displaystyle\lim_{x\rightarrow\infty} (3x-1) \ln\Big(\frac{2x-1}{2x+1}\Big) = \displaystyle\lim_{x\rightarrow\infty} \frac{ \ln\big(\frac{2x-1}{2x+1}\big) }{\frac{1}{3x-1}} \underset{u = \frac{1}{x}}{=} \displaystyle\lim_{u\rightarrow 0} \frac{ \ln\big(\frac{2-u}{2+u}\big) }{\frac{u}{3-u}}$

$\underset{\frac{0}{0}}{\overset{L'H}{=}} \displaystyle\lim_{u\rightarrow 0} \displaystyle \frac{\frac{2+u}{2-u} \cdot \frac{(-1)\cdot(2+u)-(2-u)\cdot 1}{(2+u)^2}}{\frac{1\cdot(3-u)-u\cdot(-1)}{(3-u)^2}}= -3$
\end{itemize}
\end{itemize}

$$\therefore \lim_{n\rightarrow\infty} \frac{|a_{n+1}|}{|a_n|} = \frac{1}{2^3} \cdot e^3 \cdot e^{-3} = \frac{1}{8} < 1 $$

Como $\displaystyle\lim_{n\rightarrow\infty} \frac{|a_{n+1}|}{|a_n|} < 1$, a série converge absolutamente.

\item[b)] Aplicando o critério da razão:

$$ \frac{|a_{n+1}|}{|a_n|} =  \frac{e^{3n+3}((n+1)!)^3}{(n+1)^{3n+3}} \cdot  \frac{n^{3n}}{e^{3n} (n!)^3} =  \frac{e^{3}(n+1)^3 (n!)^3}{(n+1)^{3} (n+1)^{3n}} \cdot  \frac{n^{3n}}{ (n!)^3} = \Bigg( \frac{e}{(\frac{n+1}{n})^n} \Bigg)^3 $$
$$\lim_{n\rightarrow\infty} \frac{|a_{n+1}|}{|a_n|} = \lim_{n\rightarrow\infty} \Bigg( \frac{e}{(\frac{n+1}{n})^n} \Bigg)^3 = 1 $$

O critério da razão não leva a nenhuma conclusão, pois $\displaystyle\lim_{n\rightarrow\infty} \frac{|a_{n+1}|}{|a_n|} = 1 $. Porém sabe-se que:

$$ \frac{|a_{n+1}|}{|a_n|} = \Bigg( \frac{e}{(\frac{n+1}{n})^n} \Bigg)^3 > 1, \forall n > 0 $$

porque sabemos que a sequência $(\frac{n+1}{n})^n$ é crescente e tende a $e$. Isso implica que o termo geral não vai a zero e, portanto, a série diverge.

\item[c)] Para analisar a convergência da série, utilizamos o critério da integral, com $f(x) = \frac{\ln \frac{x}{2}}{x}$. Para isso, precisamos verificar se o termo geral da série vai a zero e se $f$ é decrescente.
    
    \begin{itemize}
    \item[$\bullet$] $\displaystyle\lim_{n \rightarrow \infty} |a_n| = \lim_{x \rightarrow \infty} f(x) = \lim_{x \rightarrow \infty} \frac{\ln \frac{x}{2}}{x} \underset{\frac{\infty}{\infty}}{ \overset{L'H}{=}} \lim_{x \rightarrow \infty} \frac{\frac{2}{x} \cdot \frac{1}{2}}{1} = 0$
    \item[$\bullet$] $f'(x) = \frac{\frac{1}{x} \cdot x - \ln(\frac{x}{2}) \cdot 1}{x^2} = \frac{1 - \ln(\frac{x}{2})}{x^2} < 0, \forall x > 2e  \Rightarrow f$ é decrescente $\forall x > 2e$.
    \end{itemize}
    
    $$ \int_{6}^\infty f(x) \, dx = \int_{6}^\infty \frac{\ln \frac{x}{2}}{x}  \, dx \underset{u = \ln \frac{x}{2} \Rightarrow du = \frac{dx}{x} }{=} \int_{\ln 6}^\infty u  \, du = \lim_{u\rightarrow\infty} \frac{u^2}{2} - \frac{(\ln 6)^2}{2} = \infty $$
    
    Como a integral imprópria diverge, $\displaystyle\sum_{n=1}^\infty |a_n|$ diverge. Porém, a série alternada converge pelo critério das séries alternadas, pois, como foi visto anteriormente, o módulo do termo geral da série é descrescente e vai a zero. Sendo assim, a série converge condicionalmente.

\end{itemize}

%---------------------------------------QUESTAO 3-----------------------------------------

\newpage
\noindent{\bf 3ª Questão: }
\begin{itemize}
\item[a)] (2,5) Determinar os valores de $x \in \mathbb{R}$ para os quais a série de potências

$$ \sum_{n=1}^\infty \frac{3n-2}{(n+1)^2 2^{n+1}} (x-2)^n $$

é convergente.

\item[b)] (1,5) Estudar a convergência em termos de $\alpha > 0$:

$$ \sum_{n=3}^\infty \frac{1}{n (\ln n)(\ln (\ln n))^\alpha} $$

\end{itemize}


\noindent{\bf Solução:}
\\

\begin{itemize}
\item[a)] Aplicando o critério da razão, temos:

$$\lim_{n\rightarrow\infty} \frac{|a_{n+1}|}{|a_n|} = \lim_{n\rightarrow\infty} \frac{(3n+1) |x-2|^{n+1}}{(n+2)^2 \, 2^{n+2}}  \cdot \frac{(n+1)^2 \, 2^{n+1}}{(3n-2) |x-2|^n} = \lim_{n\rightarrow\infty} \Big( \frac{3n+1}{3n-2} \Big) \Big( \frac{n+1}{n+2} \Big)^2 \frac{|x-2|}{2}  $$
$$ = \lim_{n\rightarrow\infty} \Bigg( \frac{3+\frac{1}{n}}{3-\frac{2}{n}} \Bigg) \Bigg( \frac{1+\frac{1}{n}}{1+\frac{2}{n}} \Bigg)^2 \frac{|x-2|}{2} = \frac{|x-2|}{2} $$

Para $\frac{|x-2|}{2} < 1 \Leftrightarrow |x-2| < 2$: a série converge absolutamente, pelo critério da razão.

Para $\frac{|x-2|}{2} > 1 \Leftrightarrow |x-2| > 2$: o termo geral não vai a zero e, portanto, a série diverge.

Para $\frac{|x-2|}{2} = 1 \Rightarrow x = 0 \lor x=4$: o critério da razão é inconclusivo, outros critérios devem ser utilizados. \\

Para $x=4$, temos:

$$  \sum_{n=1}^\infty \frac{3n-2}{(n+1)^2 \, 2^{n+1}} (x-2)^n = \sum_{n=1}^\infty \frac{3n-2}{(n+1)^2 \, 2} $$

Para analisar a convergência desta série, utilizamos o critério da comparação no limite:

$$ L = \lim_{n\rightarrow\infty} \frac{\frac{3n-2}{(n+1)^2 \, 2}}{\frac{1}{n}} = \lim_{n\rightarrow\infty} \frac{3n^2-2n}{(n+1)^2 \, 2} = \lim_{n\rightarrow\infty} \frac{3-\frac{2}{n}}{(1+\frac{1}{n})^2 \, 2} = \frac{3}{2}  $$

Como $0 < L < \infty$, temos que série diverge, pois $\displaystyle\sum_{n=1}^\infty \frac{1}{n}$ é uma série divergente.

Para $x=0$, temos:

$$  \sum_{n=1}^\infty \frac{3n-2}{(n+1)^2 \, 2^{n+1}} (x-2)^n = \sum_{n=1}^\infty \frac{(-1)^n \, (3n-2)}{(n+1)^2 \, 2} $$

Esta série não apresenta convergência absoluta, pois já verificamos que $\displaystyle\sum_{n=1}^\infty \frac{3n-2}{(n+1)^2 \, 2}$ diverge. \\

Para verificar se há convergência condicional, utilizamos o critério das séries alternadas: 
    
    \begin{itemize}
    \item[$\bullet$] $\displaystyle\lim_{n \rightarrow \infty} |a_n| = \lim_{n \rightarrow \infty} \frac{3n-2}{(n+1)^2 \, 2} = \lim_{n \rightarrow \infty} \frac{3n-2}{(n^2 + 2n + 1) \, 2} = \lim_{n \rightarrow \infty} \frac{3-\frac{2}{n}}{(n + \frac{2}{n} + \frac{1}{n^2}) \, 2} = 0 $
    \item[$\bullet$] $|a_{n}| = f(n) = \displaystyle\frac{3n-2}{(n+1)^2 \, 2}  $
    
    $f'(x) = \displaystyle\frac{3 \cdot (x+1)^2 -  (3x-2) \cdot 2 (x+1)}{(x+1)^4} = \frac{7 - 3x}{(x+1)^3} < 0, \forall x > \frac{7}{3}$
    
    $\therefore |a_{n}|$ é decrescente $\forall n \geq 3$.  
    
    \end{itemize}
    
    Como o módulo do termo geral da série é descrescente e vai a zero, pode-se afirmar que, pelo critério das séries alternadas, a série é convergente. Portanto, a série converge condicionalmente. 

\item[b)] Para analisar a convergência da série, utilizamos o critério da integral, com $f(x) = \frac{1}{x (\ln x)(\ln (\ln x))^\alpha}$. Como $\alpha > 0$, o termo geral da série vai a zero e $f$ é claramente positiva e decrescente para $x \geq 3$.
    
    $$ \int_{3}^\infty f(x) \, dx = \int_{3}^\infty \frac{1}{x (\ln x)(\ln (\ln x))^\alpha}  \, dx \underset{u = \ln x \Rightarrow du = \frac{dx}{x} }{=} \int_{\ln 3}^\infty \frac{1}{ u(\ln u)^\alpha}  \, du = \underset{v = \ln u \Rightarrow dv = \frac{du}{u} }{=} \int_{\ln(\ln 3)}^\infty \frac{1}{ v^\alpha}  \, dv $$
    
    $$
    \Rightarrow \int_{3}^\infty f(x) \, dx =
    \begin{cases}
    \displaystyle \lim_{v\rightarrow\infty} \displaystyle\frac{v^{1-\alpha}}{1-\alpha} - \displaystyle\frac{(\ln(\ln 3))^{1-\alpha}}{1-\alpha}, \; \alpha \neq 1 \\
    \displaystyle \lim_{v\rightarrow\infty}\ln (v) - \ln (\ln(\ln 3)), \;\;\;\;\; \alpha = 1
    \end{cases}
    $$
    
    $$
    \lim_{v\rightarrow\infty} \displaystyle\frac{v^{1-\alpha}}{1-\alpha} =
    \begin{cases}
    \infty, \;\; 1-\alpha > 0 \\
    0, \;\;\;\; 1-\alpha < 0
    \end{cases}
    $$
    
        $$
    \therefore \int_{3}^\infty f(x) \, dx =
    \begin{cases}
    \infty, \;\;\;\;\;\;\;\;\;\;\;\;\;\;\;\;\;\;\;\;  0 < \alpha \leq 1 \\
    - \displaystyle\frac{(\ln(\ln 3))^{1-\alpha}}{1-\alpha}, \; \alpha > 1
    \end{cases}
    $$
    
    A série converge se e somente se a integral imprópria converge. Sendo assim, temos que a série converge para $\alpha > 1$ e diverge para $0 < \alpha \leq 1$.

\end{itemize}

\end{document} 
