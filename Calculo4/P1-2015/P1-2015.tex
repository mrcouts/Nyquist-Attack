\documentclass[12pt,a4paper]{article}
\usepackage[utf8x]{inputenc}
\usepackage{ucs}
\usepackage{amsmath}
\usepackage{amsfonts}
\usepackage{amssymb}
\usepackage{graphicx}
\usepackage{grffile}
\usepackage{float}
\usepackage{multicol}
\usepackage[portuguese]{babel}
\title{MAT2456 - P2-2014}
\author{André Garnier Coutinho}
\setlength{\textwidth}{17cm}
\setlength{\textheight}{24cm}
\addtolength{\topmargin}{-2cm}
\addtolength{\oddsidemargin}{-2cm}

\newcommand{\re}{\mathbb{R}}

\newcommand{\sen}{\mbox{\,sen}}

\begin{document}
%%%%%%%%%%%%%%%%%%%%%%%%%%%%%%%%%%%%%%%%%%TURMA A%%%%%%%%%%%%%%%%%%%%%%%%%%%%%%%%%%%%%%%%%%%%%%%
\begin{center}
\textbf{Instituto de Matemática e Estatística da USP\\
MAT2455 - Cálculo Diferencial e Integral IV para Engenharia\\}
\textbf{1a. Prova - 2o. Semestre 2015 - 01/09/2015}
\end{center}

\noindent {\bf Turma B}

\noindent{\bf 1ª Questão:}
\begin{itemize}
	\item[a)] (1,0) Mostre que a seguência
	
	$$0,79 \;\;\;\; 0,798 \;\;\;\; 0,7989 \;\;\;\; 0,79898 \;\;\;\; 0,798989$$
	
	é convergente e calcule o valor de seu limite como razão entre dois inteiros.
	
	\item[b)] Considere a série
	
	$$\displaystyle\sum_{n=0}^\infty (-1)^n \, \frac{1}{(2n)!} = 1 - \frac{1}{2!} + \frac{1}{4!} - ...$$
	
    Prove que a série é convergente e calcule explicitamente uma aproximação do valor da soma da série com erro menor que $2 \cdot 10^{-3}$.
\end{itemize}


\noindent{\bf \\ \\Solução:}
\begin{itemize}
    \item[a)] A sequencia dada pode ser descrita como:
    
    $$
    \begin{cases}
    a_1 = 0,79 \\
    a_{n+1} =
        \begin{cases}
        a_n + 8 \cdot 10^{-(n+2)}, \text{$n$ ímpar} \\
        a_n + 9 \cdot 10^{-(n+2)}, \text{$n$ par} \\
        \end{cases}  
    \end{cases}
    $$
    
    Assim, pode-se concluir que a sequência é crescente, pois $a_{n+1} > a_n, \forall n \geq 1$. 
    
    Como a sequência é crescente e claramente limitada superiormente por $0,8$, temos que a sequência é convergente.
    
    O convergência da sequência garante que todas as sub-sequências devem convergir para o mesmo valor. Sendo assim, definimos a seguinte sub-sequência:
    
    $$
    \begin{cases}
    a_1 = 0,79 \\
    a_{n+2} =  a_n + 89 \cdot 10^{-(n+3)}, \text{$n$ ímpar}
    \end{cases}
    $$
    
    A partir da expressão recursiva, é possível determinar uma expressão não recursiva para o termo geral da sub-sequência:
    
    \begin{align*} 
    n &= 1 \rightarrow a_3 = 0,79 + 89 \cdot 10^{-4} \\ 
    n &= 3 \rightarrow a_5 = a_3 + 89 \cdot 10^{-6} = 0,79 + 89 \cdot 10^{-4} + 89 \cdot 10^{-6} \\
    n &= 5 \rightarrow a_7 = a_5 + 89 \cdot 10^{-8} = 0,79 + 89 \cdot 10^{-4} + 89 \cdot 10^{-6} + 89 \cdot 10^{-8}\\
    \vdots \\
    n &= k-2 \rightarrow a_k = 0,79 + 89 \cdot 10^{-4}(1 + 10^{-2} + 10^{-4} + ... + 10^{-k+3}), \; \text{$k$ ímpar}
    \end{align*}
    
    Assim, podemos calcular o limite desta sub-sequência da seguinte maneira:
    
    $$ \lim_{k \rightarrow \infty} a_k = 0,79 + 89 \cdot 10^{-4} \sum_{m=0}^{\infty} (10^{-2})^m $$
    
    Como $\displaystyle\sum_{m=0}^{\infty} (10^{-2})^m$ é a soma de uma PG de razão positiva menor que $1$, temos:
    
    $$ \lim_{k \rightarrow \infty} a_k = 0,79 + 89 \cdot 10^{-4} \cdot \frac{1}{1 - 10^{-2}} = \Big(79 + \frac{89}{99} \Big) \frac{1}{100} = \frac{791}{990} $$ \\



    \item[b)] Para analisar a convergência da série, utilizamos o critério das séries alternadas:
    
    \begin{itemize}
    \item[--] $\displaystyle\lim_{n \rightarrow \infty} |a_n| = \lim_{n \rightarrow \infty} \displaystyle\frac{1}{(2n)!} = 0 $
    \item[--] $|a_{n+1}| = \displaystyle\frac{1}{(2n+2)!} = \displaystyle\frac{1}{(2n+2)(2n)!} = \frac{|a_n|}{2n+2} \Rightarrow |a_{n+1}| < |a_n|, \forall n \geq 0  $
    \end{itemize}
    
    Como o módulo do termo geral da série é descrescente e vai a zero, pode-se afirmar que, pelo critério das séries alternadas, a série é convergente.
    
    Deseja-se aproximar $\displaystyle\sum_{n=0}^\infty (-1)^n \, \frac{1}{(2n)!}$ por $\displaystyle\sum_{n=0}^k (-1)^n \, \frac{1}{(2n)!}$  com $erro < \varepsilon = 10^{-3}$.
Para séries alternadas, o erro da aproximação respeita a seguinte relação:

$$ erro = \sum_{n=k+1}^\infty a_n < |a_{k+1}| $$

Repare que se encontrarmos um valor de $k$ que respeite $|a_{k+1}| < \epsilon$, temos que $erro < \epsilon$. Sendo assim, temos:

$$|a_{k+1}| = \frac{1}{(2k+2)!} < 10^{-3} \Rightarrow (2k+2)! > 10^{3} $$

\begin{align*} 
    k &= 1 \rightarrow (2k+2)! = 4! = 24 < 10^3 \\ 
    k &= 2 \rightarrow (2k+2)! = 6! = 6 \cdot 5 \cdot 24 = 720 < 10^3 \\ 
    k &= 3 \rightarrow (2k+2)! = 8! = 8 \cdot 7 \cdot 720  > 10^3 \\ 
\end{align*}

Portanto: 

$$ \sum_{n=0}^\infty (-1)^n \, \frac{1}{(2n)!}  \simeq \sum_{n=0}^3 (-1)^n \, \frac{1}{(2n)!} = 1 - \frac{1}{2!} + \frac{1}{4!} - \frac{1}{6!}  $$

Com $erro < 10^{-3}$.
    

\end{itemize}
\ \

%---------------------------------------QUESTAO 2-----------------------------------------
\newpage

\noindent{\bf 2ª Questão:} Decidir se a série dada converge ou não. Em caso afirmativo, dizer se a convergência é absoluta ou condicional.

\begin{itemize}
\item[a)] (1,0) $\displaystyle\sum_{n=1}^\infty (-1)^{n-1} \, \Big( \frac{n}{2n-1} \Big)^{3n-1}$
\item[b)] (1,0) $\displaystyle\sum_{n=1}^\infty (-1)^{n-1} \, \frac{e^{3n} (n!)^3}{n^{3n}}$
\item[c)] (1,0) $\displaystyle\sum_{n=1}^\infty (-1)^{n-1} \, \frac{\ln \frac{n}{2}}{n}$
\end{itemize}

\noindent{\bf Solução:} \\

\begin{itemize}
\item[a)] Aplicando o critério da razão:

$$\lim_{n\rightarrow\infty} \frac{|a_{n+1}|}{|a_n|} = \lim_{n\rightarrow\infty} \Big( \frac{n+1}{2n+1} \Big)^{3n+2} \Big( \frac{2n-1}{n} \Big)^{3n-1} = \lim_{n\rightarrow\infty}  \Big( \frac{n+1}{2n+1} \Big)^3 \Big( \frac{n+1}{n} \Big)^{3n-1} \Big( \frac{2n-1}{2n+1} \Big)^{3n-1} $$

\begin{itemize}
\item[--] $\displaystyle\lim_{n\rightarrow\infty}  \Big( \displaystyle\frac{n+1}{2n+1} \Big)^3 = \displaystyle\lim_{n\rightarrow\infty}  \Big( \displaystyle\frac{1+\frac{1}{n}}{2+\frac{1}{n}} \Big)^3 = \displaystyle\frac{1}{2^3} $
\item[--] $\displaystyle\lim_{n\rightarrow\infty} \Big( \displaystyle\frac{n+1}{n} \Big)^{3n-1} = \displaystyle\lim_{n\rightarrow\infty} e^{(3n-1) \ln(\frac{n+1}{n})} = e^3$, pois:
\begin{itemize}
\item[--] $\displaystyle\lim_{x\rightarrow\infty} (3x-1) \ln\Big(\frac{x+1}{x}\Big) = \displaystyle\lim_{x\rightarrow\infty} \frac{ \ln\big(\frac{x+1}{x}\big) }{\frac{1}{3x-1}} \underset{u = \frac{1}{x}}{=} \displaystyle\lim_{u\rightarrow 0} \frac{ \ln(u+1) }{\frac{u}{3-u}}$

$\underset{\frac{0}{0}}{\overset{L'H}{=}} \displaystyle\lim_{u\rightarrow 0} \displaystyle \frac{\frac{1}{u+1}}{\frac{1\cdot(3-u)-u\cdot(-1)}{(3-u)^2}}=3$
\end{itemize}
\item[--] $\displaystyle\lim_{n\rightarrow\infty} \Big( \displaystyle\frac{2n-1}{2n+1} \Big)^{3n-1} = \displaystyle\lim_{n\rightarrow\infty} e^{(3n-1) \ln(\frac{2n-1}{2n+1})} = e^{-3}$, pois:
\begin{itemize}
\item[--] $\displaystyle\lim_{x\rightarrow\infty} (3x-1) \ln\Big(\frac{2x-1}{2x+1}\Big) = \displaystyle\lim_{x\rightarrow\infty} \frac{ \ln\big(\frac{2x-1}{2x+1}\big) }{\frac{1}{3x-1}} \underset{u = \frac{1}{x}}{=} \displaystyle\lim_{u\rightarrow 0} \frac{ \ln\big(\frac{2-u}{2+u}\big) }{\frac{u}{3-u}}$

$\underset{\frac{0}{0}}{\overset{L'H}{=}} \displaystyle\lim_{u\rightarrow 0} \displaystyle \frac{\frac{2+u}{2-u} \cdot \frac{(-1)\cdot(2+u)-(2-u)\cdot 1}{(2+u)^2}}{\frac{1\cdot(3-u)-u\cdot(-1)}{(3-u)^2}}= -3$
\end{itemize}
\end{itemize}

$$\therefore \lim_{n\rightarrow\infty} \frac{|a_{n+1}|}{|a_n|} = \frac{1}{2^3} \cdot e^3 \cdot e^{-3} = \frac{1}{8} < 1 $$

Como $\displaystyle\lim_{n\rightarrow\infty} \frac{|a_{n+1}|}{|a_n|} < 1$, a série converge absolutamente.

\item[b)] Aplicando o critério da razão:

$$ \frac{|a_{n+1}|}{|a_n|} =  \frac{e^{3n+3}((n+1)!)^3}{(n+1)^{3n+3}} \cdot  \frac{n^{3n}}{e^{3n} (n!)^3} =  \frac{e^{3}(n+1)^3 (n!)^3}{(n+1)^{3} (n+1)^{3n}} \cdot  \frac{n^{3n}}{ (n!)^3} = \Bigg( \frac{e}{(\frac{n+1}{n})^n} \Bigg)^3 $$
$$\lim_{n\rightarrow\infty} \frac{|a_{n+1}|}{|a_n|} = \lim_{n\rightarrow\infty} \Bigg( \frac{e}{(\frac{n+1}{n})^n} \Bigg)^3 = 1 $$

O critério da razão não leva a nenhuma conclusão, pois $\displaystyle\lim_{n\rightarrow\infty} \frac{|a_{n+1}|}{|a_n|} = 1 $. Porém sabe-se que:

$$ \frac{|a_{n+1}|}{|a_n|} = \Bigg( \frac{e}{(\frac{n+1}{n})^n} \Bigg)^3 > 1, \forall n > 0 $$

porque sabemos que a sequência $(\frac{n+1}{n})^n$ é crescente e tende a $e$. Isso implica que o termo geral não vai a zero e, portanto, a série diverge.

\item[c)] Para analisar a convergência da série, utilizamos o critério da integral, com $f(x) = \frac{\ln \frac{x}{2}}{x}$. Para isso, precisamos verificar se o termo geral da série vai a zero e se $f$ é decrescente.
    
    \begin{itemize}
    \item[--] $\displaystyle\lim_{n \rightarrow \infty} |a_n| = \lim_{x \rightarrow \infty} f(x) = \lim_{x \rightarrow \infty} \frac{\ln \frac{x}{2}}{x} \underset{\frac{\infty}{\infty}}{ \overset{L'H}{=}} \lim_{x \rightarrow \infty} \frac{\frac{2}{x} \cdot \frac{1}{2}}{1} = 0$
    \item[--] $f'(x) = \frac{\frac{1}{x} \cdot x - \ln(\frac{x}{2}) \cdot 1}{x^2} = \frac{1 - \ln(\frac{x}{2})}{x^2} < 0, \forall x > 2e  \Rightarrow f$ é decrescente $\forall x > 2e$.
    \end{itemize}
    
    $$ \int_{6}^\infty f(x) \, dx = \int_{6}^\infty \frac{\ln \frac{x}{2}}{x}  \, dx \underset{u = \ln \frac{x}{2} \Rightarrow du = \frac{dx}{x} }{=} \int_{\ln 6}^\infty u  \, du = \lim_{u\rightarrow\infty} \frac{u^2}{2} - \frac{(\ln 6)^2}{2} = \infty $$
    
    Como a integral imprópria diverge, $\displaystyle\sum_{n=1}^\infty |a_n|$ diverge. Porém, a série alternada converge pelo critério das séries alternadas, pois, como foi visto anteriormente, o módulo do termo geral da série é descrescente e vai a zero. Sendo assim, a série converge condicionalmente.

\end{itemize}

%---------------------------------------QUESTAO 3-----------------------------------------

\newpage
\noindent{\bf 3ª Questão: }
\begin{itemize}
\item[a)] (2,0 pontos) Seja

\begin{center}
$ f(x) = \displaystyle\begin{cases} 0 \, , \,\,\,\,\,\,\, $ se $ x \in \Big[-\pi, -\frac{\pi}{2} \Big[ \,$ ou $x \in \Big]\frac{\pi}{2}, \pi \Big] \\\\ 2|x| \,,$ se $ x \in \Big[-\frac{\pi}{2}, \frac{\pi}{2} \Big] \end{cases}$
\end{center}

Encontre a série de Fourier de $f$.

\item[b)] (1,5 pontos) Se $S(x)$ é a soma da série encontrada em $a)$, esboce o gráfico de $S$ no intervalo $[-3\pi,3\pi]$, calcule $S\Big(39 \displaystyle\frac{\pi}{2}\Big)$ e $S\Big(1223 \displaystyle\frac{\pi}{8}\Big)$.

\end{itemize}


\noindent{\bf Solução:}
\\

\begin{itemize}
\item[a)]

$$ a_0 = \frac{1}{\pi} \int_{-\pi}^\pi f(x) \, dx = \frac{1}{\pi} \int_{-\frac{\pi}{2}}^{\frac{\pi}{2}} 2 |x| \, dx = \frac{2}{\pi} \int_{0}^{\frac{\pi}{2}} 2 x \, dx = \frac{2}{\pi} x^2 \Big|_0^{\frac{\pi}{2}} = \frac{\pi}{2} $$

$$ a_n = \frac{1}{\pi} \int_{-\pi}^\pi f(x) \cos (n x) \, dx = \frac{1}{\pi} \int_{-\frac{\pi}{2}}^{\frac{\pi}{2}} 2 |x| \cos (n x) \, dx = \frac{2}{\pi} \int_{0}^{\frac{\pi}{2}} 2 x \cos (n x) \, dx $$

$$ \int_{0}^{\frac{\pi}{2}}  x \cos (n x) \, dx =  \frac{x \sen (nx)}{n} \Big|_0^\frac{\pi}{2} -  \int_{0}^{\frac{\pi}{2}} \frac{\sen(nx)}{n} \,dx = \frac{\pi \sen( \frac{n \pi}{2})}{2 n} + \frac{\cos (nx)}{n^2} \Big|_0^\frac{\pi}{2}    $$

$$ = \frac{\pi \sen( \frac{n \pi}{2})}{2 n} + \frac{ ( \cos (\frac{n \pi}{2}) -1)}{n^2} $$

$$ \therefore a_n =  \frac{ 2 \sen( \frac{n \pi}{2})}{ n} +  \frac{ 4 ( \cos (\frac{n \pi}{2}) -1)}{\pi n^2} $$

\begin{center}
$b_n = \displaystyle\frac{1}{\pi} \displaystyle\int_{-\pi}^\pi f(x) \sen (n x) \, dx = 0 \,\,$ (função ímpar)
\end{center}

$$ \therefore S(x) = \frac{a_0}{2} + \sum_{n=1}^\infty \Big( a_n \cos(nx) + b_n \sen(nx) \Big) = \frac{\pi}{4} + \sum_{n=1}^\infty \Big( \frac{ 2 \sen( \frac{n \pi}{2})}{ n} +  \frac{ 4 ( \cos (\frac{n \pi}{2}) -1)}{\pi n^2} \Big) \cos (nx) $$

\item[b)] Pelo toerema da convergência da série de Fourier, $S(x)$ converge para:

\begin{itemize}
\item[$\bullet$] Para $x \in  (-\pi,\pi)$:
	\begin{itemize}
	\item[-] $f(x), \,$ onde f(x) é contínua
	\item[-] A média dos limites laterais, onde f(x) é descontínua
	\end{itemize}
\item[$\bullet$] $\displaystyle\frac{f(\pi)+f(-\pi)}{2}, \, $ para $ x = \pi $ ou $ x = -\pi $ 
\item[$\bullet$] Para $x \notin [ -\pi,\pi]$: repete-se periodicamente.
\end{itemize}

Sendo assim:
$$ S\Big(\frac{39 \pi}{2}\Big) = S\Big(17\pi + \frac{\pi}{2}\Big) = S\Big(\pi + \frac{\pi}{2}\Big) = S\Big(- \frac{\pi}{2}\Big) = \frac{1}{2} ( \lim_{x \rightarrow - \frac{\pi}{2}^+ } f(x) + \lim_{x \rightarrow - \frac{\pi}{2}^- } f(x) ) = \frac{1}{2} ( \pi + 0) = \frac{\pi}{2} $$

$$ S\Big(1223 \cdot \frac{\pi}{8}\Big) = S\Big(152\pi + \frac{7 \pi}{8}\Big) = S\Big(\frac{7\pi}{8}\Big) = f\Big(\frac{7\pi}{8}\Big) = 0 $$

\end{itemize}

\newpage
%%%%%%%%%%%%%%%%%%%%%%%%%%%%%%%%%%%%%%%%%%TURMA B%%%%%%%%%%%%%%%%%%%%%%%%%%%%%%%%%%%%%%%%%%%%%%%
\begin{center}
\textbf{Instituto de Matemática e Estatística da USP\\
MAT2455 - Cálculo Diferencial e Integral IV para Engenharia\\}
\textbf{2a. Prova - 2o. Semestre 2014 - 13/10/2014}
\end{center}

\noindent {\bf Turma B}

\noindent{\bf 1ª Questão:}
\begin{itemize}
	\item[a)] (1,0 ponto) Seja $ f(x) = \displaystyle\frac{1}{1+3x^4}$. Calcule $f^{(40)}(0)$.
	
	\item[b)] Obtenha uma expressão para a soma da série $\displaystyle\sum_{n=1}^\infty (-1)^n \, 3^n \, 16 n^2 \, x^{4n}$ 
	\item[c)] Encontre um valor para a soma do item $b)$, quando $x = \displaystyle\frac{1}{3}$.
\end{itemize}


\noindent{\bf \\ \\Solução:}
\begin{itemize}
    \item[a)] Sabe-se que $ \displaystyle\frac{1}{1-x} = \displaystyle\sum_{n=0}^\infty x^n $, para $|x|<1$ (soma da PG). Sendo assim:
    
    $$\frac{1}{1 + 3x^4} = \sum_{n=0}^\infty (-3 x^4), |-3x^4| < 1$$
    $$ = \sum_{n=0}^\infty (-1)^n  \, 3^n \, x^{4n}, |x| < \frac{1}{\sqrt[4]{3}} $$
    
    
    Sabe-se que para uma série de potencias positivas $ \, \displaystyle\sum_{k=0}^\infty a_k (x-x_0)^k $, $a_k$ é dado pelo coeficiênte de Taylor: $a_k = \displaystyle\frac{f^{(k)}(x_0)}{k!}$.
    
    Como a série encontrada foi expandida em torno de $x_0 = 0$, temos que: $a_{40} = \displaystyle\frac{f^{(40)}(0)}{40!}$, o qual é coeficiente de $x^{40}$. \\
    
    O termo geral da série obtida é dado por $ \bar{a}_n = (-1)^n  \, 3^n \, x^{4n}$.
    
    Para $n=10$, temos $\bar{a}_{10} = 3^{10} x^{40}$, o que significa que o coeficiente de $x^{40}$ na série é $3^{10}$.
    
    Sendo assim, temos: $ \displaystyle\frac{f^{(40)}(0)}{40!} = 3^{10} $
    
    $$ \therefore f^{(40)} = 3^{10} \, 40! $$


    \item[b)] Deseja-se encontrar uma expressão para $\displaystyle\sum_{n=1}^\infty (-1)^n \, 3^n \, 16 n^2 x^{4n}$.
    
     Pode-se observar que há uma certa semelhança entre os termos gerais desta série e da série do exercicio anterior. Repare que derivando em x, multiplicando por x, derivando mais uma vez e multplicando por x mais uma vez, chegamos na mesma expressão. Sendo assim:
     
     $$\frac{1}{1 + 3x^4} =  \sum_{n=0}^\infty (-1)^n  \, 3^n \, x^{4n}, |x| < \frac{1}{\sqrt[4]{3}} $$
     
     Derivando em x:
     
     $$\frac{-3 \cdot 4 x^3}{(1 + 3x^4)^2} =  \sum_{n=0}^\infty (-1)^n  \, 3^n \, 4 n \, x^{4n-1}, |x| < \frac{1}{\sqrt[4]{3}} $$
     
     $$ \Rightarrow \frac{-12 x^4}{(1 + 3x^4)^2} =  \sum_{n=0}^\infty (-1)^n  \, 3^n \, 4 n \, x^{4n}, |x| < \frac{1}{\sqrt[4]{3}} $$
     
     Derivando mais uma vez:
     
     $$ \frac{-12 [ (4 x^3) (1 + 3x^4)^2 - x^4 \cdot 2 (1 + 3 x^4) 12 x^3 ] }{(1 + 3x^4)^4} =  \sum_{n=0}^\infty (-1)^n  \, 3^n \, 16 n^2 \, x^{4n-1}, |x| < \frac{1}{\sqrt[4]{3}} $$
     
     $$ \frac{-12 [ 4 x^3 - 12 x^7 ] }{(1 + 3x^4)^3} =  \sum_{n=0}^\infty (-1)^n  \, 3^n \, 16 n^2 \, x^{4n-1}, |x| < \frac{1}{\sqrt[4]{3}} $$
     
     $$ \therefore \sum_{n=0}^\infty (-1)^n  \, 3^n \, 16 n^2 \, x^{4n} = \frac{-36 x^3 (1 - 3 x^4)}{(1 + 3 x^4)^3} , |x| < \frac{1}{\sqrt[4]{3}} $$
     
     \item[c)] Como $ x = \frac{1}{3}$ está dentro do intervalo de convergência da série do item $b)$, temos que:
     
     $$  \sum_{n=0}^\infty   \frac{(-1)^n  \, 16 n^2}{27^n} = \frac{-36 \cdot \frac{1}{27} (1 - 3 \cdot \frac{1}{81})}{(1 + 3  \cdot \frac{1}{81})^3} = -\frac{3^5 \, 13}{2^3 \, 7^3} = -\frac{3159}{2744} $$
    

\end{itemize}
\ \

%---------------------------------------QUESTAO 2-----------------------------------------
\newpage

\noindent{\bf 2ª Questão:}

\begin{itemize}
\item[a)] (1,5 pontos) Seja $ f(x) = \begin{cases} \displaystyle\frac{e^{2x}-1}{x} \, , \, $ se $ x \neq 0 \\ 2 \, , \,\,\,\,\,\,\,\,\,\,\,\,\,\,\,\,\, $ se $x = 0 \end{cases} $

\begin{itemize}
\item[a1)] Encontre uma série numérica cuja soma seja igual a $ \displaystyle\int_0^{1/4} \, f(x) \, dx$.
\item[a2)] Encontre um valor aproximado para $ \displaystyle\int_0^{1/4} \, f(x) \, dx$, com erro, em módulo, menor que $10^{-4}$. 
\end{itemize}

\item[b)] (1,5 pontos) Sabendo que a série de Fourier de senos de $g(x) = x(\pi - x)$, em $[0, \pi]$ é

\begin{center}
$ \displaystyle\frac{8}{\pi} \Big( \sen x + \displaystyle\frac{\sen (3x)}{3^3} + \displaystyle\frac{\sen (5x)}{5^3} + ... \Big) \,$,
\end{center} 

calcule $ \displaystyle\sum_{n=0}^{\infty} \, \Big( \displaystyle\frac{1}{2n + 1} \Big)^6 $.

\end{itemize}

\noindent{\bf Solução:} \\

\begin{itemize}
\item[a1)] Sabe-se que:

$$ e^x  = \sum_{n=0}^\infty \frac{x^n}{n!} \, , \forall x \in \mathbb{R}$$

Sendo assim:

$$ e^{2x} - 1 = \sum_{n=0}^\infty \frac{(2x)^n}{n!} - 1 =  \sum_{n=1}^\infty \frac{2^n \, x^n}{n!} \, , \forall x \in \mathbb{R} $$
$$ \therefore \frac{e^{2x} - 1}{x} = \sum_{n=1}^\infty \frac{2^n \, x^{n-1}}{n!} \, , \forall x \neq 0 $$

Para x = 0:

$$ \sum_{n=1}^\infty \frac{2^n \, x^{n-1}}{n!} = 2 $$

$$ \therefore f(x) = \sum_{n=1}^\infty \frac{2^n \, x^{n-1}}{n!} \, , \forall x \in \mathbb{R} $$

Assim:

$$ \int_{0}^{x'} f(x) \, dx = \sum_{n=1}^\infty \frac{2^n \, x^{n}}{n \cdot n!} \Big|_{0}^{x'} = \sum_{n=1}^\infty \frac{2^n \, (x')^{n}}{n \cdot n!} \, , \forall x \in \mathbb{R} $$

Para $ x' = 1/4$:

$$ \int_{0}^{1/4} f(x) \, dx =  \sum_{n=1}^\infty \frac{2^n}{ 4^n \, n \cdot n!} = \sum_{n=1}^\infty \frac{1}{ 2^n \, n \cdot n!} $$

\item[a2)] Deseja-se calcular $ \displaystyle\int_{0}^{1/4} f(x) \, dx$ com $erro < \varepsilon = 10^{-4}$. Do item anterior, sabe-se que:

$$ \int_{0}^{1/4} f(x) \, dx = \sum_{n=1}^\infty \frac{1}{ 2^n \, n \cdot n!} \simeq \sum_{n=1}^k \frac{1}{ 2^n \, n \cdot n!}  $$

O erro da aproximação é dado por:

$$ erro = \sum_{n=k+1}^\infty \frac{1}{ 2^n \, n \cdot n!} $$

Rapare que se não tivessemos $ n \cdot n!$ multiplicando $2^n$, teríamos a soma de uma PG de razão $1/2$, a qual é fácil de calcular o valor exato da soma.

Repare também que o termo $ \frac{1}{n \cdot n!} $ é sempre decrescente, ou seja:

$$ \frac{1}{n \cdot n!} \leq \frac{1}{ (k+1) \cdot (k+1)!} ,\ \forall n \geq k+1 $$

Sendo assim:

$$ erro = \sum_{n=k+1}^\infty \frac{1}{ 2^n \, n \cdot n!} \leq \frac{1}{(k+1)(k+1)!} \sum_{n=k+1}^\infty \frac{1}{ 2^n} < \varepsilon $$

$$ \frac{1}{(k+1)(k+1)!} \frac{ \frac{1}{2^{k+1}} }{1 - \frac{1}{2}} = \frac{1}{(k+1)(k+1)!} \frac{1}{2^k} < \varepsilon  $$

$$ \therefore 2^k (k+1)(k+1)! > \frac{1}{\varepsilon} = 10^4 $$

Para $k = 5$:

$$ 2^k (k+1)(k+1)! = 32 \cdot 6 \cdot 720 = 192 \cdot 720 > 10^4 $$

Portanto: 

$$ \int_{0}^{1/4} f(x) \, dx  \simeq \sum_{n=1}^5 \frac{1}{ 2^n \, n \cdot n!}  $$

Com $erro < 10^{-4}$.

\item[b)] Seja $\tilde{g}(x)$ a extensão ímpar de $g(x)$. A série de Fourier de $\tilde{g}(x)$ é a série de senos de $g(x)$. Sabe-se que os coeficientes da série de Fourier de $\tilde{g}(x)$ são:

$$ a_0 = a_n = 0 $$
$$ b_{2n} = 0 $$
$$ b_{2n+1} = \frac{8}{\pi (2n + 1)^3} $$

Aplicando a identidade de Parceval:

$$ \frac{a_0}{2} + \sum_{n=1}^\infty \Big( a_n^2 + b_n^2 \Big) =\frac{1}{\pi} \int_{-\pi}^{\pi} \tilde{g}^2(x) \, dx $$

$$ \Rightarrow  \sum_{n=1}^\infty  b_{2n}^2 + \sum_{n=0}^\infty  b_{2n+1}^2  =\frac{1}{\pi} \int_{-\pi}^{\pi} \tilde{g}^2(x) \, dx $$

Como $\tilde{g}(x)$ é impar, $\tilde{g}^2(x)$ é par. Além disso, como $\tilde{g}(x)$ é extensão ímpar de $g(x)$, $\tilde{g}(x) = g(x)$ para $x \in [0, \pi]$. Assim, temos:

$$ \sum_{n=0}^\infty  \frac{64}{\pi^2 (2n + 1)^6}  =\frac{2}{\pi} \int_{0}^{\pi} g^2(x) \, dx $$.

$$ \int_{0}^{\pi} g^2(x) \, dx = \int_{0}^{\pi} x^2 \pi^2 - 2 \pi x^3 + x^4 \, dx = \Big( \frac{\pi^2 x^3}{3} - \frac{ 2\pi x^4 }{4} + \frac{x^5}{5}  \Big)_0^\pi = \frac{\pi^5}{30} $$

$$ \therefore \sum_{n=0}^\infty  \frac{1}{ (2n + 1)^6} = \frac{\pi^2}{64} \cdot \frac{2}{\pi} \cdot \frac{\pi^5}{30} = \frac{\pi^6}{960}  $$

\end{itemize}

%---------------------------------------QUESTAO 3-----------------------------------------

\newpage
\noindent{\bf 3ª Questão: }
\begin{itemize}
\item[a)] (2,0 pontos) Seja

\begin{center}
$ f(x) = \displaystyle\begin{cases} 0 \, , \,\,\,\,\,\,\, $ se $ x \in \Big[-\pi, -\frac{\pi}{2} \Big[ \,$ ou $x \in \Big]\frac{\pi}{2}, \pi \Big] \\\\ 3|x| \,,$ se $ x \in \Big[-\frac{\pi}{2}, \frac{\pi}{2} \Big] \end{cases}$
\end{center}

Encontre a série de Fourier de $f$.

\item[b)] (1,5 pontos) Se $S(x)$ é a soma da série encontrada em $a)$, esboce o gráfico de $S$ no intervalo $[-3\pi,3\pi]$, calcule $S\Big(39 \displaystyle\frac{\pi}{2}\Big)$ e $S\Big(1223 \displaystyle\frac{\pi}{8}\Big)$.

\end{itemize}


\noindent{\bf Solução:}
\\

\begin{itemize}
\item[a)]

$$ a_0 = \frac{1}{\pi} \int_{-\pi}^\pi f(x) \, dx = \frac{1}{\pi} \int_{-\frac{\pi}{2}}^{\frac{\pi}{2}} 3 |x| \, dx = \frac{2}{\pi} \int_{0}^{\frac{\pi}{2}} 3 x \, dx = \frac{3}{\pi} x^2 \Big|_0^{\frac{\pi}{2}} = \frac{3\pi}{4} $$

$$ a_n = \frac{1}{\pi} \int_{-\pi}^\pi f(x) \cos (n x) \, dx = \frac{1}{\pi} \int_{-\frac{\pi}{2}}^{\frac{\pi}{2}} 3 |x| \cos (n x) \, dx = \frac{2}{\pi} \int_{0}^{\frac{\pi}{2}} 3 x \cos (n x) \, dx $$

$$ \int_{0}^{\frac{\pi}{2}}  x \cos (n x) \, dx =  \frac{x \sen (nx)}{n} \Big|_0^\frac{\pi}{2} -  \int_{0}^{\frac{\pi}{2}} \frac{\sen(nx)}{n} \,dx = \frac{\pi \sen( \frac{n \pi}{2})}{2 n} + \frac{\cos (nx)}{n^2} \Big|_0^\frac{\pi}{2}    $$

$$ = \frac{\pi \sen( \frac{n \pi}{2})}{2 n} + \frac{ ( \cos (\frac{n \pi}{2}) -1)}{n^2} $$

$$ \therefore a_n =  \frac{ 3 \sen( \frac{n \pi}{2})}{ n} +  \frac{ 6 ( \cos (\frac{n \pi}{2}) -1)}{\pi n^2} $$

\begin{center}
$b_n = \displaystyle\frac{1}{\pi} \displaystyle\int_{-\pi}^\pi f(x) \sen (n x) \, dx = 0 \,\,$ (função ímpar)
\end{center}

$$ \therefore S(x) = \frac{a_0}{2} + \sum_{n=1}^\infty \Big( a_n \cos(nx) + b_n \sen(nx) \Big) = \frac{3\pi}{8} + \sum_{n=1}^\infty \Big( \frac{ 3 \sen( \frac{n \pi}{2})}{ n} +  \frac{ 6 ( \cos (\frac{n \pi}{2}) -1)}{\pi n^2} \Big) \cos (nx) $$

\item[b)] Pelo toerema da convergência da série de Fourier, $S(x)$ converge para:

\begin{itemize}
\item[$\bullet$] Para $x \in  (-\pi,\pi)$:
	\begin{itemize}
	\item[-] $f(x), \,$ onde f(x) é contínua
	\item[-] A média dos limites laterais, onde f(x) é descontínua
	\end{itemize}
\item[$\bullet$] $\displaystyle\frac{f(\pi)+f(-\pi)}{2}, \, $ para $ x = \pi $ ou $ x = -\pi $ 
\item[$\bullet$] Para $x \notin [ -\pi,\pi]$: repete-se periodicamente.
\end{itemize}

Sendo assim:
$$ S\Big(\frac{39 \pi}{2}\Big) = S\Big(17\pi + \frac{\pi}{2}\Big) = S\Big(\pi + \frac{\pi}{2}\Big) = S\Big(- \frac{\pi}{2}\Big) = \frac{1}{2} ( \lim_{x \rightarrow - \frac{\pi}{2}^+ } f(x) + \lim_{x \rightarrow - \frac{\pi}{2}^- } f(x) ) = \frac{1}{2} ( \frac{3\pi}{2} + 0) = \frac{3\pi}{4} $$

$$ S\Big(1223 \cdot \frac{\pi}{8}\Big) = S\Big(152\pi + \frac{7 \pi}{8}\Big) = S\Big(\frac{7\pi}{8}\Big) = f\Big(\frac{7\pi}{8}\Big) = 0 $$

\end{itemize}


\end{document} 
